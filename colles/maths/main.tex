%! suppress = Unicode
\documentclass[10pt]{article}

% Math related packages.
\usepackage{amsfonts, amsthm, amsmath, amssymb, amstext}
\usepackage{mathtools}
\usepackage{physics}
\usepackage{cancel, textcomp}
\usepackage[mathscr]{euscript}
\usepackage[nointegrals]{wasysym}
\usepackage[a4paper, left=2cm, right=2cm, top=2.3cm, bottom=2cm]{geometry}

\usepackage{esvect}
\usepackage{IEEEtrantools}

% Font related packages.
\usepackage[french]{babel}
\usepackage[unicode]{hyperref}
\usepackage[fontsize = 10pt]{scrextend}
\usepackage[T1]{fontenc}
\usepackage[utf8]{inputenc}
\usepackage[finemath]{kotex}
\usepackage{dhucs-nanumfont}
\usepackage{mathpazo}
\usepackage{FiraMono}
\usepackage{mathrsfs}

% Other
\usepackage{tabularx}

% A more colorful world.
\usepackage{xcolor}

% Display Source Code
\usepackage{listings}
\lstset{
    language=C,
    basicstyle=\small\ttfamily\mdseries,
    numberstyle=\color{gray},
    stringstyle=\color[HTML]{933797},
    commentstyle=\color[HTML]{228B22},
    emph={[2]from,with,import,as,pass,return,and,or,not},
    emphstyle={[2]\color[HTML]{DD52F0}},
    emph={[3]range,format,enumerate,print},
    emphstyle={[3]\color[HTML]{D17032}},
    emph={[4]if,elif,else,for,while,in,def,lambda,int,float,all,len},
    emphstyle={[4]\color{blue}},
    emph={[5]abs},
    emphstyle={[5]\color{black}},
    showstringspaces=false,
    breaklines=true,
    prebreak=\mbox{{\color{gray}\tiny$\searrow$}},
    numbers=left,
    xleftmargin=15pt
}

% OPTIONS
\everymath\expandafter{\the\everymath\displaystyle}

% COMMANDS
\newcommand{\f}[1]{\texttt{#1}}
\newcommand{\chpt}[1]{\newpage\begin{flushright}\huge\textbf{#1}\end{flushright}}
\newcommand{\urlsymbol}{\kern1pt\vbox to .5ex{}\raise.10ex\hbox{\pdfliteral{%
    q .8 0 0 .8 0 0 cm
    2.5 5 m 1 j 1 J .8 w
    1 5 l 0 5 0 4 y 0 1 l 0 0 1 0 y 4 0 l 5 0 5 1 y 5 2.5 l S
    3 3 m 6 6 l S 4 6 m 6 6 l 6 4 l S
Q}}\kern5pt}


\def\N{\mathbb N}
\def\Z{\mathbb Z}
\def\Q{\mathbb Q}
\def\R{\mathbb R}
\def\Rpe{\mathbb R_+^*}
\def\C{\mathbb C}
\def\K{\mathbb K}

\def\ssi{\Leftrightarrow}
\def\Ssi{\Longleftrightarrow}
\def\implique{\Longrightarrow}

\def\O{\mathcal{O}}

\newcommand{\q}[1]{\paragraph{Question #1}}

\title{\Huge\textbf{Khôlles de Mathématiques en MP2I.}}
\author{Amar AHMANE.\\ MP2I}
\date{}

\begin{document}

    \maketitle

    \subsection*{Semaine 1}
    Pas de khôlle de maths.
    \subsection*{Semaine 2}
    \subsubsection*{Question de cours}
    \paragraph{Énoncé} Soit $n\in\Z$. Montrer que \[n\text{ est pair}\Ssi n^2\text{ est pair}\]

    \subsubsection*{Exercices}
    \paragraph{Exercice} Soit $f:\R\to\R$ la fonction définie par \[\forall x\in\R,\quad f(x)=x\exp(x)\]
    \begin{enumerate}
        \item Étudier les variations de la fonction $f$ et donner l'équation de sa tangente en $0$.
        \item Montrer que $f$ induit une bijection de $[-1,+\infty[$ sur un intervalle à préciser.
        \item On note $w$ la bijection réciproque de la restriction de $f$ à $[-1,+\infty[$. Montrer que $w$ est dérivable sur $]-1/e,+\infty[$ et que \[\forall y\in]-1/e,+\infty[\textbackslash\lbrace 0\rbrace,\quad w^\prime(y)=\frac{w(y)}{y(1+w(y))}\]
    \end{enumerate}

    \subsection*{Semaine 3}
    \subsubsection*{Question de cours}
    \paragraph{Énoncé} Énoncer puis démonter la formule du binôme de Newton.

    \subsubsection*{Exercices}
    \paragraph{Exercice 1} Calculer \(\sum_{n=0}^{+\infty}\sum_{k=n}^{+\infty}\frac1{k!}\).

    \paragraph{Exercice 1} Soit $n\in\N$. Montrer que \(\sum_{0\leq i,j \leq n}ij=\sum_{k=0}^nk^3\).
\end{document}