% Preamble
\documentclass[10pt]{article}

% Math related packages.
\usepackage{amsfonts, amsthm, amsmath, amssymb, amstext}
\usepackage{mathtools}
\usepackage{physics}
\usepackage{cancel, textcomp}
\usepackage[mathscr]{euscript}
\usepackage[nointegrals]{wasysym}
\usepackage[left=3cm, right=3cm, top=1.5cm, bottom=1.5cm]{geometry}

\usepackage{esvect}
\usepackage{IEEEtrantools}

% Font related packages.
\usepackage[french]{babel}
\usepackage[unicode]{hyperref}
\usepackage[fontsize = 10pt]{scrextend}
\usepackage[T1]{fontenc}
\usepackage[utf8]{inputenc}
\usepackage[finemath]{kotex}
\usepackage{dhucs-nanumfont}
\usepackage{mathpazo}
\usepackage{FiraMono}
\usepackage{mathrsfs}

% Other
\usepackage{enumerate}
\usepackage[shortlabels]{enumitem}
\usepackage{tabularx}
\usepackage[object=vectorian]{pgfornament}
\usepackage{pgf}
\usepackage{pgfpages}


% A more colorful world.
\usepackage{xcolor}

% Display Source Code
\usepackage{listings}
\lstset{
    language=C,
    basicstyle=\small\ttfamily\mdseries,
    numberstyle=\color{gray},
    stringstyle=\color[HTML]{933797},
    commentstyle=\color[HTML]{228B22},
    emph={[2]from,with,import,as,pass,return,and,or,not},
    emphstyle={[2]\color[HTML]{DD52F0}},
    emph={[3]range,format,enumerate,print},
    emphstyle={[3]\color[HTML]{D17032}},
    emph={[4]if,elif,else,for,while,in,def,lambda,int,float,all,len},
    emphstyle={[4]\color{blue}},
    emph={[5]abs},
    emphstyle={[5]\color{black}},
    showstringspaces=false,
    breaklines=true,
    prebreak=\mbox{{\color{gray}\tiny$\searrow$}},
    numbers=left,
    xleftmargin=15pt
}

% OPTIONS
\everymath\expandafter{\the\everymath\displaystyle}

% COMMANDS
\newcommand{\f}[1]{\texttt{#1}}
\newcommand{\sct}[1]{
	\begin{center}
		\Large\textbf{#1}
	\end{center}
}
\newcommand{\subsct}[1]{
	\begin{center}
		\large\textbf{#1}
	\end{center}
}
\newcommand{\q}[1]{\textbf{#1.}\quad}
\newcommand{\urlsymbol}{\kern1pt\vbox to .5ex{}\raise.10ex\hbox{\pdfliteral{%
    q .8 0 0 .8 0 0 cm
    2.5 5 m 1 j 1 J .8 w
    1 5 l 0 5 0 4 y 0 1 l 0 0 1 0 y 4 0 l 5 0 5 1 y 5 2.5 l S
    3 3 m 6 6 l S 4 6 m 6 6 l 6 4 l S
Q}}\kern5pt}


\def\N{\mathbb N}
\def\Z{\mathbb Z}
\def\Q{\mathbb Q}
\def\R{\mathbb R}
\def\Rpe{\mathbb R_+^*}
\def\C{\mathbb C}
\def\K{\mathbb K}
\def\L{\mathbb L}

\def\P{\mathscr{P}}

\def\ssi{\Leftrightarrow}
\def\Ssi{\Longleftrightarrow}
\def\implique{\Longrightarrow}

% CUSTOM TITLE
\makeatletter
\def\@maketitle{%
	\newpage
	%  \null% DELETED
	%  \vskip 2em% DELETED
	\begin{center}%
		\let \footnote \thanks
		{\LARGE \@title \par}%
		\vskip 1em%
		{\large
			\lineskip .5em%
			\begin{tabular}[t]{c}%
				\@author
			\end{tabular}\par}%
		\vskip 1em%
		{\large \@date}%
	\end{center}%
	\par
	\vskip -1em}
\makeatother

\title{\textbf{TD 4 : Calculs de complexité}}
\author{Amar AHMANE \\ MP2I}
\date{}


% DOCUMENT
\begin{document}
	\maketitle
	\subsct{Exercice 3 : Couleur Majoritaire}
	Soit $n\in\N^*$. On pose $E=[\![1, n]\!]$. Soit $\L$ un ensemble dont les éléments sont des couleurs, on note alors $R$ l'application définie par 


	\[R:\begin{array}{c c c}E&\to& \L\\ x&\mapsto& R(x)\end{array}\]

	On fait alors quelques définitions :
	\begin{enumerate}[(i)]
		\item Soient $x\in E$, $c\in\L$. On dit que $x$ est de couleur $c$ si et seulement si $R(x)=c$.
		\item On dit que deux entiers $i$ et $j$ de $E$ sont "de même couleur" si et seulement si $R(i)=R(j)$. 
		\item Soit $c\in\L$, on note $E_c$ l'ensemble des entiers dans $E$ de couleur $c$. Alors \[E_c=\lbrace x\in E\ |\ R(x)=c\rbrace\]
		\item Soit $c\in\L$, on dit que $c$ est majoritaire dans $E$ si et seulement si $|E_c|>\frac{|E|}2$.
	\end{enumerate}

	On peut considérer que cette portion de code est écrite en début de fichier, et que cette structure peut être utilisée dans les fonctions que l'on programmera dans la suite :
	\begin{lstlisting}
	struct ec {
		int couleur;
	}
	\end{lstlisting}

	\vspace{2mm}

	\begin{enumerate}[start=9,label={\bfseries \arabic*}]
		\item $n$ désigne la taille du pointeur sur \f{ec}.


		\begin{lstlisting}
		int count(struc ec x, struct ec* ecs, int n){
			int result = 0;
			for(int i = 0; i < n; i++){
				if(ecs[i].couleur == x.couleur) result++;
			}
			return result;
		}
		\end{lstlisting}
		Il s'agit d'un algorithme en $\Theta(n)$.
		\item O
		\begin{lstlisting}
		bool has_max(struct ec* ecs, int n, int l){
			int* colours = (int*) malloc(l*sizeof(int));
			for(int i = 0; i < l; i++){
				colours[i] = -1;
			}
			for(int i = 0; i < n; i++){
				if(colours[ecs[i].couleur] == -1){
					int occ_in_i = count(ecs[i], ecs, n);
					if(occ_in_i > n/2) return true;
				}
			}
			return false;
		}
		\end{lstlisting}
		\item Soient $c\in\L$, $k\in\N$ et $(E_i)_{i\in [\![1, k]\!]}$ une partition de $E$. Supposons que $c$ est majoritaire dans $E$, montrons, par l'absurde, qu'il existe $i\in[\![1, k]\!]$ tel que $c$ est majoritaire dans $E_i$.\\
		Supposons alors que pour tout $i\in[\![1, k]\!]$, $c$ n'est pas majoritaire dans $E_i$, i.e $E_{ic}\leq \frac{|E_i|}2$. En ce cas \[\sum_{i=1}^k|E_{ic}|\leq \sum_{i=1}^k \frac{|E_i|}2 \tag{1}\]
		Or
		\begin{align*}
			\sum_{i=1}^k \frac{|E_i|}2 &= \frac12\sum_{i=1}^k\sum_{x\in E_i}1\\
									   &=\footnotemark[1] \frac12\sum_{x\in E}1\\
									   &= \frac{|E|}2
		\end{align*}
		Montrons que $(E_{ic})_{i\in[\![1, k]\!]}$ est un recouvrement disjoint de $E_c$.
		\begin{enumerate}[(i)]
			\item Soit $(i, j)\in[\![1, k]\!]^2$. On a $E_{ic}\subset E_i$ et $E_{jc}\subset E_j$, or $E_i\cap E_j=\emptyset$, donc $E_{ic}\cap E_{jc}=\emptyset$.
			\item \[\bigcup_{i=1}^kE_{ic}=\bigcup_{i=1}^kE_i\cap E_c=E_c\cap\bigcup_{i=1}^kE_i=E_c\cap E= E_c\]
		\end{enumerate}
		Donc, on a 
		\begin{align*}
			\sum_{i=1}^k |E_{ic}| &= \sum_{i=1}^k\sum_{x\in E_{ic}}1\\
						          &=\footnotemark[2] \sum_{x\in E_c}1\\
							      &= |E_c|
		\end{align*}
		Donc d'après (1) \[|E_c|\leq \frac{|E|}2\]
		Autrement dit $c$ n'est pas majoritaire dans $E$, ce qui est absurde.
		\item L'idée est la suivante pour cet algorithme : les couleurs majoritaires dans chacun des $E_i$ sont des couleurs candidates, du fait de l'implication prouvée plus haut. Le travail de l'algorithme sera ainsi d'établir une liste d'au plus $k$ candidats, puis ensuite de vérifier s'il y en a un qui est majortaire dans $E$. Le tout se fait en $\mathcal O(n\sqrt{n})$.
		\begin{lstlisting}
		bool has_max_bis(struct ec* ecs, int k, int* bornes, int l){
			int* pmax = (int*) malloc(k*sizeof(int));
			int* colours = (int*) malloc(l*sizeof(int));
			int occ_in_i;
			int p = 0;
			for(int i = 0; i < k; i++){
				for(int j = 0; j < l; j++){
					colours[i] = -1;
				}
				for(int j = bornes[i]; j <= bornes[i + 1]; j++){
					if(colours[ecs[j].couleur] == -1){
						occ_in_i = 0;
						for(int k = bornes[i]; k <= bornes[i + 1; k++){
							if(ecs[k].couleur == ecs[j].couleur) occ_in_i++;
						}
						if(occ_in_i > k/2) {
							pmax[i] = ecs[j];
							p++;
							break;
						}
					}
				}
			}
			for(int i = 0; i < p; i++){
				if(count(pmax[i], ecs, bornes[k] + 1) > (bornes[k] + 1)/2) return true;
			}
			return false;
		}
		\end{lstlisting}
		\item On continue d'utiliser l'implication prouvée en question 11 dans cette partie. La seule différence est que l'on possède ici, à chaque fois, au plus deux candidats. À chaque instance, les lignes \f{4} et \f{5} nous les fournissent, il suffit alors de faire la vérification. Celle-ci se faisant en temps $n$, on a que $T(n)\leq T(1)+n\log_2(n)$, donc $T(n)=\mathcal O(n\log n)$
		\begin{lstlisting}
		struct ec max_dicho(struc ec* ecs, int first, int last){
			struct ec default_x;
			int mid = (first + last)/2;
			int occ = 0;
			struct ec x1 = max_dicho(ecs, first, mid);
			struct ec x2 = max_dicho(ecs, mid + 1, last);
			default_x.couleur = -1;
			if(last == first){
				return ecs[first];
			} else if(last > first || first > last){
				return default_x;
			}
			if(x1.couleur != -1){
				for(int i = first; i <= last; i++){
					if(ecs[i].couleur == x1.couleur) occ++;
				}
				if(occ > (first - last)/2) return x1;
				occ = 0;
			}
			if(x2.couleur != -1){
				for(int i = first; i <= last; i++){
					if(ecs[i].couleur == x2.couleur) occ++;
				}
				if(occ > (first - last)/2) return x2;
				free(occ);
			}
			return default_x;
		}
		\end{lstlisting}
		\item Supposons que $n\geq 3$. Soient $x,y\in E$ deux entiers distincts de couleurs distinctes. Soit $c\in\L$, supposons que $c$ est majoritaire dans $E$. Les ensembles $E\backslash\lbrace x,y\rbrace$ et $\lbrace x,y\rbrace$ forment une partition de $E$. Or, il n'existe pas de couleur majoritaire dans $\lbrace x,y\rbrace$, donc $c$ est majoritaire dans $E\backslash\lbrace x,y\rbrace$.
	\end{enumerate}
	\footnotetext[1]{$(E_i)_{i\in [\![1, k]\!]}$ est une partition de $E$.}
	\footnotetext[1]{$(E_{ic})_{i\in [\![1, k]\!]}$ est un recouvrement disjoint de $E_c$.}
\end{document}