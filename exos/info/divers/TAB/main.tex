\documentclass[a4paper,12pt,reqno]{amsart}
\usepackage{latexsym}
\usepackage{amsmath,amsthm,amssymb,amscd}
\usepackage{fullpage}
\usepackage{xcolor}
\usepackage{csquotes}
%\usepackage{eucal}
\usepackage{parskip}
\usepackage{mathtools}
\usepackage{algorithm}
\usepackage[noend]{algpseudocode}
\mathtoolsset{showonlyrefs}

\usepackage[activate={true,nocompatibility},final,tracking=true,kerning=true,spacing=true]{microtype}
\usepackage[pdfencoding=unicode,pdftex,bookmarks=true,bookmarksnumbered]{hyperref}
\definecolor{citecolour}{rgb}{0.0, 0.0, 0.8}
\colorlet{linkcolour}{green!50!black}
\hypersetup{colorlinks,breaklinks,
		linkcolor=linkcolour,citecolor=citecolour,
		filecolor=linkcolour, urlcolor=linkcolour}
\usepackage{version}


\newtheorem{prevtheorem}{Théorème}
\renewcommand*{\theprevtheorem}{\Roman{prevtheorem}}
\newtheorem{prevcorollary}{Corollary}
\renewcommand*{\theprevcorollary}{\Roman{prevcorollary}}
\newtheorem{prevlemma}{Lemma}
\renewcommand*{\theprevlemma}{\Roman{prevlemma}}

\newtheorem{theorem}{Theorem}
\newtheorem{proposition}[theorem]{Proposition}
\newtheorem{propdef}[theorem]{Proposition-Définition}
\newtheorem*{propositionno}{Proposition}
\newtheorem*{corollaryno}{Corollary}
\newtheorem{claim}[theorem]{Claim}
\newtheorem{property}{Property}
\newtheorem{lemma}[theorem]{Lemma}
\newtheorem{corollary}[theorem]{Corollary}
\newtheorem*{conjectureno}{Conjecture}
\newtheorem*{theoremno}{Theorem}


\theoremstyle{remark}
\newtheorem{example}[theorem]{Example}
\newtheorem{exs}[theorem]{Examples}
\newtheorem{définition}[theorem]{Définition}
\newtheorem{remark}[theorem]{Remark}
\newtheorem{remarks}[theorem]{Remarks}
\newtheorem{problem}[theorem]{Problem}

\numberwithin{equation}{section}
\usepackage[shortlabels]{enumitem}
\begingroup
 \makeatletter
 \@for\theoremstyle:=definition,remark,plain\do{%
 \expandafter\g@addto@macro\csname th@\theoremstyle\endcsname{%
 \addtolength\thm@preskip\parskip
 }%
 }

\DeclareMathOperator{\set}{\mathfrak{F}}
\DeclareMathOperator{\X}{\mathfrak{X}}
\DeclareMathOperator{\Y}{\mathfrak{Y}}
\DeclareMathOperator{\N}{\mathbf{N}}
\DeclareMathOperator{\Z}{\mathbf{Z}}
\DeclareMathOperator{\Syl}{Syl}
\DeclareMathOperator{\ab}{ab}
\DeclareMathOperator{\Hall}{Hall}
\DeclareMathOperator{\cart}{Cart}
\DeclareMathOperator{\cent}{\mathbf{C}}
\DeclareMathOperator{\soc}{\mathbf{Soc}}
\DeclareMathOperator{\nsup}{nsup}
\DeclareMathOperator{\norm}{\mathrel{\unlhd}}
\DeclareMathOperator{\aut}{\mathbf{Aut}}
\DeclareMathOperator{\gl}{GL}
\DeclareMathOperator{\zent}{\mathbf{Z}}
\DeclareMathOperator{\fratt}{\mathbf{\Phi}}


\newcommand{\Q}{\mathbb{Q}}
\newcommand{\R}{\mathbb{R}}
\newcommand{\K}{\mathbb{K}}
\newcommand{\z}{\mathfrak{z}}
\newcommand{\gensub}[1]{\langle{#1}\rangle}

\includeversion{comment}
\renewcommand{\leq}{\leqslant}
\renewcommand{\geq}{\geqslant}
\newcommand{\md}[1]{\,\left(\textnormal{mod}\ #1\right)}
\renewcommand\qedsymbol{$\blacksquare$}
\renewenvironment{proof}{{\textit{Preuve} }}{\hfill $\square$ \\}
\newenvironment{proofofA}{{\bf {Proof of Theorem \ref{Thm:A}.} }}{\hfill $\square$ \\}
\renewcommand\qedsymbol{$\blacksquare$}
\newenvironment{proofofC}{{\bf {Proof of Theorem \ref{Thm:C}.} }}{\hfill $\blacksquare$ \\}
\newenvironment{proofof}{{\bf {Proof.} }}{\hfill $\blacksquare$ \\}




\begin{document}
\title{Généralisation d'un résultat de complexité à une classe plus large d'algorithmes récursifs}
\author{Amar AHMANE \& Pierre-Gabriel Berlureau}


\begin{abstract}
Si on note $\mathcal A$ un algorithme prenant en entrée un seul paramètre de taille $n$. Dans l'hypothèse où $\mathcal A$ est algorithme récursif utilisant la stratégie \textit{diviser pour régner}, on montrera dans le Théorème~\ref{Thm:A} que, dans des conditions particulières, la complexité de l'algorithme est quasi-linéaire et on précisera la constante.
\end{abstract}
\maketitle

\section{Introduction}
Ce qui a motivé la recherche et la preuve du petit résultat généralisé qui sera exposé plus tard est l'exercice de Khôlle qui a été posé à Louis Thevenet. On considère l'algorithme suivant :

\begin{algorithm}
\caption{}
\begin{algorithmic}[1]
\Procedure{foo}{$lst$}
	\If{\Call{Length}{$lst$} = 0}
		\State \Return 1
	\EndIf
	\State $m \gets \Call{Length}{lst} / 3$
	\State \Return \Call{foo}{$lst$[:$m$]} + \Call{foo}{$lst$[$m$:]}
\EndProcedure
\end{algorithmic}
\end{algorithm}
On arrive très bien à dénombrer les différentes tailles qui apparaissent à un certain étage de l'arbre d'appel de cet algorithme récursif. On a alors le résultat suivant

\begin{proposition}
L'Algorithme 1 a une complexité en $\mathcal O\left(n\log_{\frac 32}(n)\right)$.
\end{proposition}

L'idée était alors de considérer le cas général décrit ci-dessous.

\section{Résultat généralisé}
Dans toute la suite, on notera $\mathcal A$ un algorithme récursif utilisant la stratégie \textit{diviser pour régner}. Un premier résultat que l'on va énoncer et démontrer ici consistera à dénombrer des objets d'une certaine forme dans un type d'arbres particulier. En réalité, on s'intéressera à des arbres à (au plus) $m$ fils par noeud, avec $m\geq 2$ un entier naturel, définis par récurrence d'une façon qui sera précisée plus tard. Mais avant de s'y attaquer, précisons quelques notations et définitions locales.

\begin{définition}
Soit $E$ un ensemble. On appelle arbre sur $E$ un arbre dont les noeuds sont des éléments de $E$. On notera $\mathfrak A_E$ l'ensemble des arbres sur $E$.
\end{définition}

\begin{définition}
Soit $E$, $F$ deux ensembles et $f\in F^E$. Alors, lorsque $A\in \mathfrak A_E$, on note $\tilde{f}:\mathfrak A_E\to\mathfrak A_F$ l'application qui à $A$ associe un arbre sur $F$ dont les noeuds sont les noeuds de $A$ auxquels on aura appliqué $f$.
\end{définition}

\begin{propdef}
Soit $m\geq 2$, $(x_1,\dots,x_m)\in(\R^*)^m$. On définit par récurrence un arbre parfait infini $A$ sur $\R^m$ de la façon suivante :
\begin{itemize}
	\item La racine de $A$ est $0_{\R^m}$.
	\item Pour tout noeud $N=(n_1,\dots,n_m)$ de $A$, $N$ possède exactement $m$ fils et pour tout $i\in[\![1,m]\!]$, le $i$ème fils est le noeud $N_i=(n_1,\dots,n_i+x_i,\dots,n_m)$.
\end{itemize}
Ainsi, pour tout $k\in\N$, pour tout $(i_1,\dots,i_m)\in\N^m$ tel que $i_1+\dots+i_m=k$, le nombre de noeuds de valeur $(i_1x_1,i_2x_2,\dots,i_mx_m)$ à l'étage $k$ est \[\frac{k!}{i_1!\dots i_m!}\]
Note : l'étage $0$ correspond à celui de la racine.
\end{propdef}

\begin{proof}
On montre le résultat par récurrence sur $k$.
\begin{itemize}
	\item Pour $k=0$, l'étage $0$ correspond à celui de la racine, de valeur $0_{\R^m}$, qui apparaît bien $1=\frac{0!}{0!\dots 0!}$, $0_{\R^m}$ étant le seul $m$-uplet de somme $0$.
	\item 
\end{itemize}
\end{proof}

\begin{prevtheorem}\label{Thm:A}
	Soit $m\in\N^*$, $m\geq 2$, $p\in \N^*\backslash\lbrace m\rbrace$ et $(q_1,\dots,q_m)\in(\N^*)^m$ un $m$-uplet d'entiers non tous égaux tels que \[\sum_{k=1}^mq_k=p\]
	Soit $\mathcal A$ un algorithme récursif prenant un seul paramètre et utilisant la stratégie \text{diviser pour régner} et tel que pour une entrée de taille $n$ \[T(n)=\sum_{k=1}^mT\left(\frac{q_kn}p\right)+\mathcal O(1)\]
	où $T(n)$ représente le nombre d'opérations élémentaires effectuées par $\mathcal A$ pour une entrée de taille $n$. Alors \[T(n)=\mathcal O\left(n\log_{\frac{p}{\max(q_1,\dots,q_m)}}(n)\right)\]
\end{prevtheorem}


\end{document}