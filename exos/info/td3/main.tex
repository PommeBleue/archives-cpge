%! suppress = Unicode
\documentclass[10pt]{article}

% Math related packages.
\usepackage{amsfonts, amsthm, amsmath, amssymb, amstext}
\usepackage{mathtools}
\usepackage{physics}
\usepackage{cancel, textcomp}
\usepackage[mathscr]{euscript}
\usepackage[nointegrals]{wasysym}
\usepackage[a4paper, left=2cm, right=2cm, top=2.3cm, bottom=2cm]{geometry}

\usepackage{esvect}
\usepackage{IEEEtrantools}

% Font related packages.
\usepackage[french]{babel}
\usepackage[unicode]{hyperref}
\usepackage[fontsize = 10pt]{scrextend}
\usepackage[T1]{fontenc}
\usepackage[utf8]{inputenc}
\usepackage[finemath]{kotex}
\usepackage{dhucs-nanumfont}
\usepackage{mathpazo}
\usepackage{FiraMono}
\usepackage{mathrsfs}

% Other
\usepackage{tabularx}

% A more colorful world.
\usepackage{xcolor}

% Display Source Code
\usepackage{listings}
\lstset{
    language=C,
    basicstyle=\small\ttfamily\mdseries,
    numberstyle=\color{gray},
    stringstyle=\color[HTML]{933797},
    commentstyle=\color[HTML]{228B22},
    emph={[2]from,with,import,as,pass,return,and,or,not},
    emphstyle={[2]\color[HTML]{DD52F0}},
    emph={[3]range,format,enumerate,print},
    emphstyle={[3]\color[HTML]{D17032}},
    emph={[4]if,elif,else,for,while,in,def,lambda,int,float,all,len},
    emphstyle={[4]\color{blue}},
    emph={[5]abs},
    emphstyle={[5]\color{black}},
    showstringspaces=false,
    breaklines=true,
    prebreak=\mbox{{\color{gray}\tiny$\searrow$}},
    numbers=left,
    xleftmargin=15pt
}

% OPTIONS
\everymath\expandafter{\the\everymath\displaystyle}

% COMMANDS
\newcommand{\f}[1]{\texttt{#1}}
\newcommand{\chpt}[1]{\newpage\begin{flushright}\huge\textbf{#1}\end{flushright}}
\newcommand{\urlsymbol}{\kern1pt\vbox to .5ex{}\raise.10ex\hbox{\pdfliteral{%
    q .8 0 0 .8 0 0 cm
    2.5 5 m 1 j 1 J .8 w
    1 5 l 0 5 0 4 y 0 1 l 0 0 1 0 y 4 0 l 5 0 5 1 y 5 2.5 l S
    3 3 m 6 6 l S 4 6 m 6 6 l 6 4 l S
Q}}\kern5pt}


\def\N{\mathbb N}
\def\Z{\mathbb Z}
\def\Q{\mathbb Q}
\def\R{\mathbb R}
\def\Rpe{\mathbb R_+^*}
\def\C{\mathbb C}
\def\K{\mathbb K}

\def\ssi{\Leftrightarrow}
\def\Ssi{\Leftrightarrow}
\def\implique{\Longrightarrow}

\def\O{\mathcal{O}}

\newcommand{\q}[1]{\paragraph{Question #1}}

\title{\Huge\textbf{TD3 : Ordres de grandeur et complexités simples.}}
\author{Amar AHMANE.\\ MP2I}
\date{}

\begin{document}

    % Defining column types
    \newcolumntype{C}[1]{>{\centering\arraybackslash}p{#1}}

    \maketitle
    \subsubsection*{1 – Classement}
    Étant donné les conditions données sur les classes, on peut en former 7 avc les fonctions données. Tout est résumé dans ce joli petit tableau :\\
    \begin{center}
        \begin{tabular}{| C{2cm} | C{1.5cm} | C{2.5cm} | C{2cm} | C{2.5cm} | C{2cm} | C{2cm} |}
            \hline
            \(\Theta_1\) & \(\Theta_2\) & \(\Theta_3\) & \(\Theta_4\) & \(\Theta_5\) & \(\Theta_6\) & \(\Theta_7\)\\
            \hline
            \(n\mapsto \log n\) & \(n\mapsto \sqrt{n} \) & \(n\mapsto n^2+\sqrt{n}\) & \(n\mapsto n^2\sqrt{n}\) & \(n\mapsto 2n^2+n^3\) & \(n\mapsto 2^n \) & \(n\mapsto 3^n\)\\
            \(n\mapsto \log n^2\) & – & \(n\mapsto 3n^2-5n\) & – & – & \(n\mapsto 2^{n+2} \) & –\\
            \(n\mapsto \log 2n\) & – & – & –  & – & – & –\\
            \hline
        \end{tabular} 
    \end{center}  

    \subsubsection*{2 – Min et Max}
    Soient $f,g\in(\R_+^*)^\N$.
    \paragraph{Question 1} Montrons que $f=\Theta(g)\ssi g=\Theta(f)$.\\


    $\boxed{\Rightarrow}$\quad Supposons que $f=\Theta(g)$. Il existe alors $c,d\in\R_+^*$ et $n_0\in\N$ tels que \[\forall n\in\N,\quad n\geq n_0\implique dg(n) \leq f(n)\leq cg(n)\]
    Soit $n\in\N$. Supposons que $n\geq n_0$. D'une part \[f(n)\leq cg(n) \ssi \frac1c f(n) \leq g(n)\]
    D'autre part \[dg(n)\leq f(n) \ssi g(n)\leq\frac1d f(n) \]
    Finalement, \[\frac1c f(n)\leq g(n)\leq\frac1d f(n)\]
    En posant $(c^\prime, d^\prime)=\left(\frac1d, \frac1c \right)$, on a \[\boxed{d^\prime f(n)\leq g(n)\leq c^\prime f(n)}\]

    $\boxed{\Leftarrow}$\quad Le problème est symétrique.\\


    \paragraph{Question 2} Montrons que $\max(f,g)=\Theta(f+g)$.\\
    Soit $n\in\N$. On a, puisque on a toujours soit $\max(f,g)=f$ soit $\max(f,g)=g$, et comme $f$ et $g$ sont à valeurs dans $\Rpe$ : \[\max(f,g)(n)\leq f(n)+g(n)\]
    Mais aussi, comme \[f(n)\leq \max(f,g)(n)\quad \text{et} g(n)\leq \max(f,g)(n)\]
    Il en découle que \[f(n)+g(n)\leq 2\max(f,g)\ssi \frac{f(n)+g(n)}2\leq \max(f,g)(n)\]
    Donc, finalement, \[\frac{f(n)+g(n)}2\leq\max(f,g)(n) \leq f(n)+g(n)\]

    \paragraph{Question 3} Tout ce que l'on peut dire de $\min(f, g)$ est que \[\min(f,g)=\O(f)\ \text{(ou }\O(g)\text{ ou }\O(f+g)\text{ )}\]

    \subsubsection*{3 – Boucles}
    \paragraph{Question 4} 
    \begin{lstlisting}
    for(int i = 0; i < n; i++){
        for(int j = 0; j < n; j++){
            for(int k = 0; k < n; k++){
                // cout constant
            }
        }
    }
    \end{lstlisting}
    En supposant que l'on fasse $p$ opérations de coût constant après être entré dans la troisième boucle, on fera ainsi le nombre d'opérations suivant :
    \begin{align*}
        \sum_{i=0}^{n-1}\sum_{j=0}^{n-1}\sum_{k=0}^{n-1}1 &= \sum_{i=0}^{n-1}\sum_{j=0}^{n-1}np\\
                                                    &= \sum_{i=0}^{n-1}n\times np\\
                                                    &= n\times n^2p=n^3p
    \end{align*}
    Ainsi, en notant $T_1(n)$ le nombre d'opérations total de coût constant effectuées à la fin du programme, on a que $T_1(n)=\O(n^3)$.\\

    \begin{lstlisting}
    for(int i = 0; i < n; i++){
        for(int j = 0; j < i; j++){
        }
    }
    \end{lstlisting}
    Pour ce second programme, on supposera encore que l'on est en train d'effectuer $p$ opérations de coût constant après entre entré dans la seconde boucle. On fera ainsi le nombre d'opérations suivant :
    \begin{align*}
        \sum_{i=0}^{n-1}\sum_{j=0}^{i-1}p &= \sum_{i=0}^{n-1}ip\\
                                      &= p\frac{n(n-1)}2
    \end{align*}
    Ainsi, en notant $T_2(n)$ le nombre d'opérations de coût constant total effectuées à la fin du programme, on a que $T_2(n)=\O(n^2)$.

    \paragraph{Question 6} $\Theta((f\cdot g)(n))$.

    \subsubsection*{4 – Exponentiation rapide}
    
\end{document}