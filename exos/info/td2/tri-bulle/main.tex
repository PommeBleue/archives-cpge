%! suppress = Unicode
\documentclass[10pt]{article}

% Math related packages.
\usepackage{amsfonts, amsthm, amsmath, amssymb, amstext}
\usepackage{mathtools}
\usepackage{physics}
\usepackage{cancel, textcomp}
\usepackage[mathscr]{euscript}
\usepackage[nointegrals]{wasysym}
\usepackage[a4paper, left=1cm, right=1.2cm, top=2.3cm, bottom=2cm]{geometry}

\usepackage{esvect}
\usepackage{IEEEtrantools}

% Font related packages.
\usepackage[french]{babel}
\usepackage[unicode]{hyperref}
\usepackage[fontsize = 10pt]{scrextend}
\usepackage[T1]{fontenc}
\usepackage[utf8]{inputenc}
\usepackage[finemath]{kotex}
\usepackage{dhucs-nanumfont}
\usepackage{mathpazo}
\usepackage{FiraMono}
\usepackage{mathrsfs}

% A more colorful world.
\usepackage{xcolor}

% Display Source Code
\usepackage{listings}
\lstset{
    language=C,
    basicstyle=\small\ttfamily\mdseries,
    numberstyle=\color{gray},
    stringstyle=\color[HTML]{933797},
    commentstyle=\color[HTML]{228B22},
    emph={[2]from,with,import,as,pass,return,and,or,not},
    emphstyle={[2]\color[HTML]{DD52F0}},
    emph={[3]range,format,enumerate,print},
    emphstyle={[3]\color[HTML]{D17032}},
    emph={[4]if,elif,else,for,while,in,def,lambda,int,float,all,len},
    emphstyle={[4]\color{blue}},
    emph={[5]abs},
    emphstyle={[5]\color{black}},
    showstringspaces=false,
    breaklines=true,
    prebreak=\mbox{{\color{gray}\tiny$\searrow$}},
    numbers=left,
    xleftmargin=15pt
}

% OPTIONS
\everymath\expandafter{\the\everymath\displaystyle}

% COMMANDS
\newcommand{\f}[1]{\texttt{#1}}
\newcommand{\chpt}[1]{\newpage\begin{flushright}\huge\textbf{#1}\end{flushright}}
\newcommand{\urlsymbol}{\kern1pt\vbox to .5ex{}\raise.10ex\hbox{\pdfliteral{%
    q .8 0 0 .8 0 0 cm
    2.5 5 m 1 j 1 J .8 w
    1 5 l 0 5 0 4 y 0 1 l 0 0 1 0 y 4 0 l 5 0 5 1 y 5 2.5 l S
    3 3 m 6 6 l S 4 6 m 6 6 l 6 4 l S
Q}}\kern5pt}


\def\N{\mathbb N}
\def\Z{\mathbb Z}
\def\Q{\mathbb Q}
\def\R{\mathbb R}
\def\C{\mathbb C}
\def\K{\mathbb K}

\title{\Huge\textbf{TD2 : Preuves de terminaison et de correciton.}}
\author{Amar AHMANE.\\ MP2I}
\date{}

\begin{document}
    \maketitle
    \subsection*{2 – Tri bulle}
        \begin{lstlisting}
        /**
         * entree : un tableau d'entiers et sa taille
         * sirtue : le tableau est trie en place
         */
         void tri_bulle(int *tab, int n){
            bool permutation = true;
            while(permutation){ // si le dernier passage a fait au moins une modification   
                permutation = false;
                for(int i = 0; i < n-1; i++){
                    if(tab[i] > tab[i+1]){
                        int tmp = tab[i];
                        tab[i] = tab[i+1];
                        tab[i+1] = tmp;
                        permutation = true;
                    }
                }
                n = n - 1;
            }
         }
        \end{lstlisting}
        Ce code servira seulement de référence pour les démonstrations qui vont suivre, on aura notamment besoin de spécifier le numéro des lignes pour montrer précisément de quels changements dans les variables on parle.
\end{document}