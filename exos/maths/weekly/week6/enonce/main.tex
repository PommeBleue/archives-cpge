% Preamble
\documentclass[17pt]{article}

% Math related packages.
\usepackage{amsfonts, amsthm, amsmath, amssymb, amstext}
\usepackage{mathtools}
\usepackage{physics}
\usepackage{cancel, textcomp}
\usepackage[mathscr]{euscript}
\usepackage[nointegrals]{wasysym}
\usepackage[left=3cm, right=3cm, top=1.5cm, bottom=1.5cm]{geometry}

\usepackage{esvect}
\usepackage{IEEEtrantools}

% Font related packages.
\usepackage[french]{babel}
\usepackage[unicode]{hyperref}
\usepackage[fontsize = 10pt]{scrextend}
\usepackage[T1]{fontenc}
\usepackage[utf8]{inputenc}
\usepackage[finemath]{kotex}
\usepackage{dhucs-nanumfont}
\usepackage{mathpazo}
\usepackage{FiraMono}
\usepackage{mathrsfs}

% Other
\usepackage{enumerate}
\usepackage[shortlabels]{enumitem}
\usepackage{tabularx}
\usepackage[object=vectorian]{pgfornament}
\usepackage{pgf}
\usepackage{pgfpages}
\usepackage[european,straightvoltages]{circuitikz}
\usepackage{scalerel}
\usepackage{stackengine}


% A more colorful world.
\usepackage{xcolor}

% Display Source Code
\usepackage{listings}
\lstset{
    language=C,
    basicstyle=\small\ttfamily\mdseries,
    numberstyle=\color{gray},
    stringstyle=\color[HTML]{933797},
    commentstyle=\color[HTML]{228B22},
    emph={[2]from,with,import,as,pass,return,and,or,not},
    emphstyle={[2]\color[HTML]{DD52F0}},
    emph={[3]range,format,enumerate,print},
    emphstyle={[3]\color[HTML]{D17032}},
    emph={[4]if,elif,else,for,while,in,def,lambda,int,float,all,len},
    emphstyle={[4]\color{blue}},
    emph={[5]abs},
    emphstyle={[5]\color{black}},
    showstringspaces=false,
    breaklines=true,
    prebreak=\mbox{{\color{gray}\tiny$\searrow$}},
    numbers=left,
    xleftmargin=15pt
}

% OPTIONS
\everymath\expandafter{\the\everymath\displaystyle}

% COMMANDS
\newcommand{\f}[1]{\texttt{#1}}
\newcommand{\sct}[1]{
	\begin{center}
		\Large\textbf{#1}
	\end{center}
}
\newcommand{\subsct}[1]{
	\begin{center}
		\large\textbf{#1}
	\end{center}
}
\newcommand{\inl}[2]{[\![#1, #2]\!]}
\newcommand{\q}[1]{\textbf{#1.}\quad}
\newcommand{\urlsymbol}{\kern1pt\vbox to .5ex{}\raise.10ex\hbox{\pdfliteral{%
    q .8 0 0 .8 0 0 cm
    2.5 5 m 1 j 1 J .8 w
    1 5 l 0 5 0 4 y 0 1 l 0 0 1 0 y 4 0 l 5 0 5 1 y 5 2.5 l S
    3 3 m 6 6 l S 4 6 m 6 6 l 6 4 l S
Q}}\kern5pt}

\newcounter{iloop}
\newcommand\openbigstar[1][0.7]{%
  \scalerel*{%
    \stackinset{c}{-.125pt}{c}{}{\scalebox{#1}{\color{white}{$\bigstar$}}}{%
      $\bigstar$}%
  }{\bigstar}
}
\newcommand{\Stars}[1]{\ensuremath{%
\pgfmathtruncatemacro{\imax}{ifthenelse(int(#1)==#1,#1-1,#1)}%
\pgfmathsetmacro{\xrest}{0.9*(1-#1+\imax)}%
\setcounter{iloop}{0}%
\loop\stepcounter{iloop}\ifnum\value{iloop}<\the\numexpr\imax+1
\bigstar\repeat
\openbigstar[\xrest]%
\setcounter{iloop}{0}%
\loop\stepcounter{iloop}\ifnum\value{iloop}<\the\numexpr5-\imax\relax
\openbigstar[.9]\repeat}}


\def\N{\mathbb N}
\def\Z{\mathbb Z}
\def\Q{\mathbb Q}
\def\R{\mathbb R}
\def\Rpe{\mathbb R_+^*}
\def\C{\mathbb C}
\def\K{\mathbb K}
\def\L{\mathbb L}

\def\P{\mathscr{P}}

\def\ssi{\Leftrightarrow}
\def\Ssi{\Longleftrightarrow}
\def\implique{\Longrightarrow}

\def\sep{\noindent\makebox[\linewidth]{\rule{\paperwidth}{0.4pt}}}

% CUSTOM TITLE
\makeatletter
\def\@maketitle{%
	\newpage
	%  \null% DELETED
	%  \vskip 2em% DELETED
	\begin{center}%
		\let \footnote \thanks
		{\LARGE\bfseries \@title \par}%
		\vskip 1em%
		{\large
			\lineskip .5em%
			\begin{tabular}[t]{c}%
				\@author
			\end{tabular}\par}%
		\vskip 1em%
		{\large \@date}%
	\end{center}%
	\par
	\vskip -1em}
\makeatother

\title{Semaine 6}
\author{Pomme Bleue}

\begin{document}
	\maketitle
	\begin{center}
		\bfseries\Large\textsc{Pierre-Gabriel Berlureau}
	\end{center}
	\subsection*{Congruences modulo un sous-groupe et Théorème de Lagrange}
	Soit $G$ un groupe noté multiplicativement et $H$ un sous-groupe de $G$.
	\paragraph*{Définition}
	\begin{itemize}
		\item Les classes à droite modulo $H$ sont les ensembles $xH$ où $x$ parcourt $G$, qui ne sont pas des groupes (sauf si $x\in H$).
		\item Les classes à gauche modulo $H$ sont les ensembles $Hx$ où $x$ parcourt $G$, qui ne sont pas des groupes (sauf si $x\in H$).
	\end{itemize}
	Montrer les points suivants : 
	\begin{enumerate}[i)]
		\item Pour tout $x\in G$, $|Hx|=|H|$.
		\item $\lbrace Hx;\ x\in G\rbrace$ est une partition de $G$.
	\end{enumerate}
	On admettra qu'une propriété similaire est valable pour les classes à gauche.\\
	On suppose à présent que $G$ est fini. On appelle alors \textit{ordre} de $G$ son cardinal. Montrer alors le Théorème de Lagrande : i.e que l'ordre de $H$ divise l'ordre de $G$.
	
	\begin{center}
		\bfseries\Large\textsc{Matteo Delfour}
	\end{center}
	\subsection*{Morphismes de $\Q$ dans $\Z$}
		Trouver tous les morphismes de groupes de $(\Q,+)$ dans $(\Z,+)$.
	\begin{center}
		\bfseries\Large\textsc{Yanis Grigy}
	\end{center}
	\subsection*{Petit Lemme}
	Soit $G$ un groupe (multiplicatif) tel que pour tout $x\in G$, $x^2=1_G$. Montrer que $G$ est abélien et que $|G|$ est une puissance de $2$.
		
	\begin{center}
		\bfseries\Large\textsc{Louis Marchal}
	\end{center}
	\subsection*{Groupes dont l'ensemble des sous-groupes est fini}
		Caractériser les groupes dont l'ensemble des sous-groupes est fini.

	\begin{center}
		\bfseries\Large\textsc{Louis Thevenet}
	\end{center}
	\subsection*{Cas particulier du Lemme de Cauchy}
		Soit $G$ un groupe (multiplicatif) de neutre $1$. Soit $g\in G$, l'ordre de $g$ est par définition : \[\text{ord}(g)=\min\lbrace n\in\N^*\ | \ x^n=1\rbrace\]
		Cet ordre peut être $+\infty$ par convention si l'ensemble ci-dessus est vide.\\
		On suppose que $G$ est de cardinal $2p$ avec $p$ premier et on admet le Théorème de Lagrange que Pierre-Gabriel doit démontrer dans son exercice et le Lemme que Yanis doit démontrer. Montrer alors que $G$ contient un élément d'ordre $p$.

	\newpage
	\begin{center}
		\bfseries\Large\textsc{Armand sans nom de famille}
	\end{center}
	\subsection*{Existence d'un idempotent}
	Soit $E$ un ensemble fini muni d'une loi de composition interne associative. Montrer que $E$ contient un élément idempotent.

	\begin{center}
		\bfseries\Large\textsc{Shems}
	\end{center}
	\subsection*{Neutre à droite et inverse à droite}
	Soit $G$ un ensemble non vide muni d'une loi associative notée multiplicativement admettant un neutre à droite $e$ : \[\forall g\in G,\ ge=g\]
	et telle que tout élément admette un inverse à droite : \[\forall g\in G,\ \exists g'\in G,\ gg'=e\]
	Montrer que cette loi définit une structure de groupe.
	\newpage
	\begin{center}
		\Large\bfseries Indications
	\end{center}
	\paragraph{Pierre-Gabriel BERLUREAU} Le point i) résulte de la bijection induite par la multiplication par $a$. Le point ii) nécessaite de vérifier que $Ha$ est la classe  de $a$ pour la relation d'équivalence. Pour le Théorème de Lagrange, remarquer que les parts de la partition sont toutes de même taille.

	\paragraph{Matteo DELFOUR} Analyse-Synthèse où on utilisera à un moment le fait que les sous-groupes de $(\Z, +)$ sont les $n\Z$.

	\paragraph{Yanis GRIGY} La commutativité est simple à montrer. Pour ce qui est du reste, faire une récurrence sur $|G|$.

	\paragraph{Louis MARCHAL} Viens me voir.

	\paragraph{Louis THEVENET} Par l'absurde, montrer que pour tout $x\in G$, $x^2=1$, utiliser le Lemme de l'exercice de Yanis et en déduire une absurdité.

	\paragraph{Armand sans nom de famille} Considérer, pour $x\in G$, $f_x:n\in N\mapsto a^{2^n}$. Est-elle injective ?

	\paragraph{Shems} Pour $g\in G$, il existe un inverse à droite pour $g$ noté $g'$ et un inverse à droite pour $g'$ noté $g''$. Se débrouiller avec ça.
\end{document}