% Preamble
\documentclass[17pt]{article}

% Math related packages.
\usepackage{amsfonts, amsthm, amsmath, amssymb, amstext}
\usepackage{mathtools}
\usepackage{physics}
\usepackage{cancel, textcomp}
\usepackage[mathscr]{euscript}
\usepackage[nointegrals]{wasysym}
\usepackage[left=3cm, right=3cm, top=1.5cm, bottom=1.5cm]{geometry}

\usepackage{esvect}
\usepackage{IEEEtrantools}

% Font related packages.
\usepackage[french]{babel}
\usepackage[unicode]{hyperref}
\usepackage[fontsize = 10pt]{scrextend}
\usepackage[T1]{fontenc}
\usepackage[utf8]{inputenc}
\usepackage[finemath]{kotex}
\usepackage{dhucs-nanumfont}
\usepackage{mathpazo}
\usepackage{FiraMono}
\usepackage{mathrsfs}

% Other
\usepackage{enumerate}
\usepackage[shortlabels]{enumitem}
\usepackage{tabularx}
\usepackage[object=vectorian]{pgfornament}
\usepackage{pgf}
\usepackage{pgfpages}
\usepackage[european,straightvoltages]{circuitikz}
\usepackage{scalerel}
\usepackage{stackengine}


% A more colorful world.
\usepackage{xcolor}

% Display Source Code
\usepackage{listings}
\lstset{
    language=C,
    basicstyle=\small\ttfamily\mdseries,
    numberstyle=\color{gray},
    stringstyle=\color[HTML]{933797},
    commentstyle=\color[HTML]{228B22},
    emph={[2]from,with,import,as,pass,return,and,or,not},
    emphstyle={[2]\color[HTML]{DD52F0}},
    emph={[3]range,format,enumerate,print},
    emphstyle={[3]\color[HTML]{D17032}},
    emph={[4]if,elif,else,for,while,in,def,lambda,int,float,all,len},
    emphstyle={[4]\color{blue}},
    emph={[5]abs},
    emphstyle={[5]\color{black}},
    showstringspaces=false,
    breaklines=true,
    prebreak=\mbox{{\color{gray}\tiny$\searrow$}},
    numbers=left,
    xleftmargin=15pt
}

% OPTIONS
\everymath\expandafter{\the\everymath\displaystyle}

% COMMANDS
\newcommand{\f}[1]{\texttt{#1}}
\newcommand{\sct}[1]{
	\begin{center}
		\Large\textbf{#1}
	\end{center}
}
\newcommand{\subsct}[1]{
	\begin{center}
		\large\textbf{#1}
	\end{center}
}
\newcommand{\inl}[2]{[\![#1, #2]\!]}
\newcommand{\q}[1]{\textbf{#1.}\quad}
\newcommand{\urlsymbol}{\kern1pt\vbox to .5ex{}\raise.10ex\hbox{\pdfliteral{%
    q .8 0 0 .8 0 0 cm
    2.5 5 m 1 j 1 J .8 w
    1 5 l 0 5 0 4 y 0 1 l 0 0 1 0 y 4 0 l 5 0 5 1 y 5 2.5 l S
    3 3 m 6 6 l S 4 6 m 6 6 l 6 4 l S
Q}}\kern5pt}

\newcounter{iloop}
\newcommand\openbigstar[1][0.7]{%
  \scalerel*{%
    \stackinset{c}{-.125pt}{c}{}{\scalebox{#1}{\color{white}{$\bigstar$}}}{%
      $\bigstar$}%
  }{\bigstar}
}
\newcommand{\Stars}[1]{\ensuremath{%
\pgfmathtruncatemacro{\imax}{ifthenelse(int(#1)==#1,#1-1,#1)}%
\pgfmathsetmacro{\xrest}{0.9*(1-#1+\imax)}%
\setcounter{iloop}{0}%
\loop\stepcounter{iloop}\ifnum\value{iloop}<\the\numexpr\imax+1
\bigstar\repeat
\openbigstar[\xrest]%
\setcounter{iloop}{0}%
\loop\stepcounter{iloop}\ifnum\value{iloop}<\the\numexpr5-\imax\relax
\openbigstar[.9]\repeat}}


\def\N{\mathbb N}
\def\Z{\mathbb Z}
\def\Q{\mathbb Q}
\def\R{\mathbb R}
\def\Rpe{\mathbb R_+^*}
\def\C{\mathbb C}
\def\K{\mathbb K}
\def\L{\mathbb L}

\def\P{\mathscr{P}}

\def\ssi{\Leftrightarrow}
\def\Ssi{\Longleftrightarrow}
\def\implique{\Longrightarrow}

\def\sep{\noindent\makebox[\linewidth]{\rule{\paperwidth}{0.4pt}}}

% CUSTOM TITLE
\makeatletter
\def\@maketitle{%
	\newpage
	%  \null% DELETED
	%  \vskip 2em% DELETED
	\begin{center}%
		\let \footnote \thanks
		{\LARGE \@title \par}%
		\vskip 1em%
		{\large
			\lineskip .5em%
			\begin{tabular}[t]{c}%
				\@author
			\end{tabular}\par}%
		\vskip 1em%
		{\large \@date}%
	\end{center}%
	\par
	\vskip -1em}
\makeatother

\title{Semaine 2}

\begin{document}
	\maketitle
	\subsubsection*{\textsc{Pierre-Gabriel Berlureau} \Stars{2}}
	Montrer que pour tout entier naturel $n$, si $n$ n'est pas un carré parfait, alors $\sqrt{n}\in\R\backslash\Q$.
	\subsubsection*{\textsc{Matteo Delfour} \Stars{4}}
		Montrer que la propriété fondamentale de $\N$ est équivalente à l'axiome de récurrence.\\
		\textbf{Reformulation : } Soit $\mathcal P$ une propriété un entier $n\in\N$. On montre que ces deux phrases sont équivalentes
		\begin{enumerate}[i)]
		 \item $(\mathcal{P}(0)\wedge\ (\forall n\in\N,\ \mathcal{P}(n)\implique \mathcal{P}(n+1)))\implique (\forall n\in\N,\ \mathcal{P}(n))$
		 \item Toute partie non vide et majorée de $\N$ admet un plus grand élément.
		\end{enumerate}
	\subsubsection*{\textsc{Yanis Grigy} \Stars{4}}
		Soit $n\in\N$, $X=(x_1,\dots,x_n)\in(\R_+^*)^n$. Montrer que \[\sum_{k=1}^n\frac{x_k}n\geq \sqrt[n]{\prod_{k=1}^nx_k}\]
	\subsubsection*{\textsc{Louis Marchal} \Stars{4}}
		Soient $n\in\N$, $I\in\mathcal{P}(\N^n)$ un ensemble fini de $n-uplets$. Montrer qu'il existe des entiers $w_1,\dots,w_n$ tels que l'application $f:(m_1,\dots,m_n)\in I\mapsto \sum_{k=1}^nm_kw_k$ est injective.
	\subsubsection*{\textsc{Shems} \Stars{4}}
		\textit{Note : la difficulté de l'exercice vient seulement du fait que les notions sont nouvelles, en réalité, l'exercice ne mérite pas plus de deux étoiles, d'ailleurs, je serais plus d'avis à lui en donner 1.69.}\\



		Soient $I$ et $J$ deux ensembles non vides et $(u_{i,j})_{(i,j)\in I\times J}$ une famille de réels. Montrer que cette famille est majorée si et seulement si pour tout $i\in I$, la famille $(u_{i,j})_{j\in J}$ est majorée et la famille $\left(\sup_{i\in J} u_{i,j}\right)_{i\in I}$ est majorée. On a donc\[\sup_{(i,j)\in I\times J}u_{i,j}=\sup_{i\in J}\left(\sup_{j\in J} u_{i,j}\right)\]

\end{document}