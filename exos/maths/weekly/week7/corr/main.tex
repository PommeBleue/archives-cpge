% Preamble
\documentclass[17pt]{article}

% Math related packages.
\usepackage{amsfonts, amsthm, amsmath, amssymb, amstext}
\usepackage{mathtools}
\usepackage{physics}
\usepackage{cancel, textcomp}
\usepackage[mathscr]{euscript}
\usepackage[nointegrals]{wasysym}
\usepackage[left=3cm, right=3cm, top=1.5cm, bottom=1.5cm]{geometry}

\usepackage{esvect}
\usepackage{IEEEtrantools}

% Font related packages.
\usepackage[french]{babel}
\usepackage[unicode]{hyperref}
\usepackage[fontsize = 10pt]{scrextend}
\usepackage[T1]{fontenc}
\usepackage[utf8]{inputenc}
\usepackage[finemath]{kotex}
\usepackage{dhucs-nanumfont}
\usepackage{mathpazo}
\usepackage{FiraMono}
\usepackage{mathrsfs}

% Other
\usepackage{enumerate}
\usepackage[shortlabels]{enumitem}
\usepackage{tabularx}
\usepackage[object=vectorian]{pgfornament}
\usepackage{pgf}
\usepackage{pgfpages}
\usepackage[european,straightvoltages]{circuitikz}
\usepackage{scalerel}
\usepackage{stackengine}


% A more colorful world.
\usepackage{xcolor}

% Display Source Code
\usepackage{listings}
\lstset{
    language=C,
    basicstyle=\small\ttfamily\mdseries,
    numberstyle=\color{gray},
    stringstyle=\color[HTML]{933797},
    commentstyle=\color[HTML]{228B22},
    emph={[2]from,with,import,as,pass,return,and,or,not},
    emphstyle={[2]\color[HTML]{DD52F0}},
    emph={[3]range,format,enumerate,print},
    emphstyle={[3]\color[HTML]{D17032}},
    emph={[4]if,elif,else,for,while,in,def,lambda,int,float,all,len},
    emphstyle={[4]\color{blue}},
    emph={[5]abs},
    emphstyle={[5]\color{black}},
    showstringspaces=false,
    breaklines=true,
    prebreak=\mbox{{\color{gray}\tiny$\searrow$}},
    numbers=left,
    xleftmargin=15pt
}

% OPTIONS
\everymath\expandafter{\the\everymath\displaystyle}

% COMMANDS
\newcommand{\f}[1]{\texttt{#1}}
\newcommand{\sct}[1]{
	\begin{center}
		\Large\textbf{#1}
	\end{center}
}
\newcommand{\subsct}[1]{
	\begin{center}
		\large\textbf{#1}
	\end{center}
}
\newcommand{\inl}[2]{[\![#1, #2]\!]}
\newcommand{\q}[1]{\textbf{#1.}\quad}
\newcommand{\urlsymbol}{\kern1pt\vbox to .5ex{}\raise.10ex\hbox{\pdfliteral{%
    q .8 0 0 .8 0 0 cm
    2.5 5 m 1 j 1 J .8 w
    1 5 l 0 5 0 4 y 0 1 l 0 0 1 0 y 4 0 l 5 0 5 1 y 5 2.5 l S
    3 3 m 6 6 l S 4 6 m 6 6 l 6 4 l S
Q}}\kern5pt}

\newcounter{iloop}
\newcommand\openbigstar[1][0.7]{%
  \scalerel*{%
    \stackinset{c}{-.125pt}{c}{}{\scalebox{#1}{\color{white}{$\bigstar$}}}{%
      $\bigstar$}%
  }{\bigstar}
}
\newcommand{\Stars}[1]{\ensuremath{%
\pgfmathtruncatemacro{\imax}{ifthenelse(int(#1)==#1,#1-1,#1)}%
\pgfmathsetmacro{\xrest}{0.9*(1-#1+\imax)}%
\setcounter{iloop}{0}%
\loop\stepcounter{iloop}\ifnum\value{iloop}<\the\numexpr\imax+1
\bigstar\repeat
\openbigstar[\xrest]%
\setcounter{iloop}{0}%
\loop\stepcounter{iloop}\ifnum\value{iloop}<\the\numexpr5-\imax\relax
\openbigstar[.9]\repeat}}


\def\N{\mathbb N}
\def\Z{\mathbb Z}
\def\Q{\mathbb Q}
\def\R{\mathbb R}
\def\Rpe{\mathbb R_+^*}
\def\C{\mathbb C}
\def\K{\mathbb K}
\def\L{\mathbb L}

\def\P{\mathscr{P}}

\def\Ker{\text{Ker}}
\def\ord{\text{ord}}

\def\ssi{\Leftrightarrow}
\def\Ssi{\Longleftrightarrow}
\def\implique{\Longrightarrow}

\def\sep{\noindent\makebox[\linewidth]{\rule{\paperwidth}{0.4pt}}}

% CUSTOM TITLE
\makeatletter
\def\@maketitle{%
	\newpage
	%  \null% DELETED
	%  \vskip 2em% DELETED
	\begin{center}%
		\let \footnote \thanks
		{\LARGE\bfseries \@title \par}%
		\vskip 1em%
		{\large
			\lineskip .5em%
			\begin{tabular}[t]{c}%
				\@author
			\end{tabular}\par}%
		\vskip 1em%
		{\large \@date}%
	\end{center}%
	\par
	\vskip -1em}
\makeatother

\title{Semaine 7\\ Exercices}
\author{Amar AHMANE}

\begin{document}
	\maketitle
	\begin{center}
		\textit{Je m'en allais, les poings dans mes poches crevées ;}\\
		\textit{Mon paletot aussi devenait idéal ;}\\
		\textit{J'allais sous le ciel, Muse ! et j'étais ton féal ;}\\
		\textit{Oh ! là ! là ! que d'amours splendides j'ai rêvées !}\\
	\end{center}
	\begin{flushright}
		\textit{Ma bohème}, Arthur Rimbaud
	\end{flushright}
	Les rappels : 
	\begin{itemize}
		\item[\bfseries Définition 1] Lorsque $G$ est un groupe fini, on appelle ordre de $G$ son cardinal.
		\item[\bfseries Définition 2] Soit $G$ un groupe multiplicatif. Soit $g\in G$, alors on définit \[\ord(g)=\min\lbrace n\in\N^*\ | \ g^n=1\rbrace\]
		Par convention, cette quantité est $+\infty$ si l'ensemble considéré ci-dessus est vide.
		\item[\bfseries Définition 3] Soit $n\in\N^*$. On note $\mathbb P$ l'ensemble des nombres premiers. Cet entier admet une décomposition nombres premiers (résultat admis) que l'on choisit de noter de la sorte : \[n=\prod_{p\in\mathbb P}p^{v_p(n)}\]
		où $v_p(n)$ est l'exposant de $p$ dans la décomposition en nombre premiers de $n$. Dans ce cas, le pgcd de deux entiers $m$ et $n$ non nuls est l'entier \[n\wedge m=\prod_{p\in\mathbb P}p^{\min(v_p(n),v_p(m))}\]
		\item[\bfseries Définition 4] On dit que deux entiers sont premiers entre eux si leur pgcd est égal à 1.
	\end{itemize}
	Quelques résultats utiles :
	\begin{itemize}
		\item Théorème de Lagrange : l'ordre de chaque sous-groupe d'un groupe $G$ divise l'ordre de $G$.
		\item Lemme de Gauss : lorsque $p$ et $q$ sont deux entiers premiers entre eux, et lorsque $p|mq$ où $m$ est un entier, on a que $p|m$.
	\end{itemize}
	\subsection*{Exercice : ordre du produit de deux éléments dont les ordres sont premiers entre eux (source : Oraux X-ENS, Algèbre 1, Cassini)}
	Évidemment, l'ordre de $y^m$ est $n$, et l'ordre de $x^m$ est $n$. Or, on a $y^m\in <xy>$ et $x^n\in <xy>$ donc, d'après Lagrange, $O(x^n)=O(x)|O(xy)$ et $O(y^m)=O(y)|O(xy)$, mais il est aussi clair que $(xy)^{mn}=1$ donc $O(xy)|O(x)O(y)$ et $O(x)O(y)|O(xy)$; or, tous les entiers avec lesquels ont travaille ici sont dans $\N$, mais $|$ est une relation d'ordre sur $\N$, donc par antisymétrie $O(xy)=O(x)O(y)=mn$.
	\subsection*{Exercice : quelques exemples de sous-groupes (source : Les maths en tête, Xavier Gourdon)}
	Soit $G$ un groupe, $H_1$ et $H_2$ deux sous-groupes de $G$.
	\begin{enumerate}
		\item On suppose que $H_1\cup H_2$ est un sous-groupe de $G$ et on suppose par l'absurde que que $H_1\not\subset H_2$ et $H_2\not\subset H_1$. Il existe alors $h_1\in H_1$ tel que $h_1\notin H_2$ et $h_2\in H_2$ tel que $h_2\notin H_2$. Alors $h_1h_2\in H_1\cup H_2$ puisque $H_1\cup H_2$ est un sous-groupe : si $h_1h2\in H_1$, alors $h_2=h_1^{-1}(h_1h_2)\in H_1$ ce qui est absurde, sinon on a que $h_1h_2\in H_2$ et $h_1=(h_1h_2)h_2^{-1}\in H_2$ ce qui est aussi absurde.
		\item $H_1\cap H_2$ est un sous-groupe de $H_1$, donc son ordre divise celui de $H_1$, de même, c'est un sous-groupe de $H_2$ donc son ordre divise celui de $H_2$. Ainsi, l'ordre de $H_1\cap H_2$ est un diviseur commun de l'ordre de $H_1$ et celui de $H_2$, donc l'ordre de $H_1\cap H_2$ est fatalement $1$, donc $H_1\cap H_2=\lbrace 1_G\rbrace$.
	\end{enumerate}
	\subsection*{Exercice : cardinal d'un groupe fini et Im et Ker. (source : Oraux X-ENS, Algèbre 1, Cassini)}
	Soit $G$ un groupe fini, et $f$ un morphisme de $G$ dans lui-même.
	\begin{enumerate}
		\item C'est du cours.
		\item Il y a autant de classes d'équivalences que d'images par le morphisme $f$.
		\item Vérifications faciles, le neutre est évidemment $\bar{1}$.
		\item Découle directement de l'équivalence $f(x)=f(y)\ssi x^{-1}y\in \Ker f$.
		\item Conséquence de ce que PG a du démontrer la semaine dernière : les classes d'équivalences sont deux à deux disjontes et de même cardinal et leur union disjointe est égale à $G$, il suffit alors de passer au cardinal.
		\item C'était la question la plus difficile de la semaine : il fallait se convaincre que $(\Ker f=\Ker f^2\ssi \Im f=\Im f^2)\ssi (|\Ker f|=|\Ker f^2|\ssi |\Im f|=|\Im f^2|)$; en effet, si $x\in \Ker f$, on a $f\circ f(x)=f(e)=e$ donc $x\in \Ker f^2$ donc $\Ker f\subset \Ker f^2$, d'autre part si $y\in \Im f^2$, alors il existe $x\in G$ tel que $y=f\circ f(x)\in \Im f$ donc $\Im f^2\subset \Im f$. Comme tous ces ensembles sont finis, il est clair que $\Im f=\Im f^2\ssi |\Im f|=|\Im f^2|$, et de même $\Ker f=\Ker f^2\ssi |\Ker f|=|\Ker f^2|$. L'exercice devient très simple puisque l'on sait que $|G|=|\Ker f|\times |\Im f|$, et comme $f^2$ est aussi un homomorphisme, $|G|=|\Ker f^2|\times |\Im f^2|$, je vous laisse conclure...
	\end{enumerate}
\end{document}