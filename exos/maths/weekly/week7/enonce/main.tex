% Preamble
\documentclass[17pt]{article}

% Math related packages.
\usepackage{amsfonts, amsthm, amsmath, amssymb, amstext}
\usepackage{mathtools}
\usepackage{physics}
\usepackage{cancel, textcomp}
\usepackage[mathscr]{euscript}
\usepackage[nointegrals]{wasysym}
\usepackage[left=3cm, right=3cm, top=1.5cm, bottom=1.5cm]{geometry}

\usepackage{esvect}
\usepackage{IEEEtrantools}

% Font related packages.
\usepackage[french]{babel}
\usepackage[unicode]{hyperref}
\usepackage[fontsize = 10pt]{scrextend}
\usepackage[T1]{fontenc}
\usepackage[utf8]{inputenc}
\usepackage[finemath]{kotex}
\usepackage{dhucs-nanumfont}
\usepackage{mathpazo}
\usepackage{FiraMono}
\usepackage{mathrsfs}

% Other
\usepackage{enumerate}
\usepackage[shortlabels]{enumitem}
\usepackage{tabularx}
\usepackage[object=vectorian]{pgfornament}
\usepackage{pgf}
\usepackage{pgfpages}
\usepackage[european,straightvoltages]{circuitikz}
\usepackage{scalerel}
\usepackage{stackengine}


% A more colorful world.
\usepackage{xcolor}

% Display Source Code
\usepackage{listings}
\lstset{
    language=C,
    basicstyle=\small\ttfamily\mdseries,
    numberstyle=\color{gray},
    stringstyle=\color[HTML]{933797},
    commentstyle=\color[HTML]{228B22},
    emph={[2]from,with,import,as,pass,return,and,or,not},
    emphstyle={[2]\color[HTML]{DD52F0}},
    emph={[3]range,format,enumerate,print},
    emphstyle={[3]\color[HTML]{D17032}},
    emph={[4]if,elif,else,for,while,in,def,lambda,int,float,all,len},
    emphstyle={[4]\color{blue}},
    emph={[5]abs},
    emphstyle={[5]\color{black}},
    showstringspaces=false,
    breaklines=true,
    prebreak=\mbox{{\color{gray}\tiny$\searrow$}},
    numbers=left,
    xleftmargin=15pt
}

% OPTIONS
\everymath\expandafter{\the\everymath\displaystyle}

% COMMANDS
\newcommand{\f}[1]{\texttt{#1}}
\newcommand{\sct}[1]{
	\begin{center}
		\Large\textbf{#1}
	\end{center}
}
\newcommand{\subsct}[1]{
	\begin{center}
		\large\textbf{#1}
	\end{center}
}
\newcommand{\inl}[2]{[\![#1, #2]\!]}
\newcommand{\q}[1]{\textbf{#1.}\quad}
\newcommand{\urlsymbol}{\kern1pt\vbox to .5ex{}\raise.10ex\hbox{\pdfliteral{%
    q .8 0 0 .8 0 0 cm
    2.5 5 m 1 j 1 J .8 w
    1 5 l 0 5 0 4 y 0 1 l 0 0 1 0 y 4 0 l 5 0 5 1 y 5 2.5 l S
    3 3 m 6 6 l S 4 6 m 6 6 l 6 4 l S
Q}}\kern5pt}

\newcounter{iloop}
\newcommand\openbigstar[1][0.7]{%
  \scalerel*{%
    \stackinset{c}{-.125pt}{c}{}{\scalebox{#1}{\color{white}{$\bigstar$}}}{%
      $\bigstar$}%
  }{\bigstar}
}
\newcommand{\Stars}[1]{\ensuremath{%
\pgfmathtruncatemacro{\imax}{ifthenelse(int(#1)==#1,#1-1,#1)}%
\pgfmathsetmacro{\xrest}{0.9*(1-#1+\imax)}%
\setcounter{iloop}{0}%
\loop\stepcounter{iloop}\ifnum\value{iloop}<\the\numexpr\imax+1
\bigstar\repeat
\openbigstar[\xrest]%
\setcounter{iloop}{0}%
\loop\stepcounter{iloop}\ifnum\value{iloop}<\the\numexpr5-\imax\relax
\openbigstar[.9]\repeat}}


\def\N{\mathbb N}
\def\Z{\mathbb Z}
\def\Q{\mathbb Q}
\def\R{\mathbb R}
\def\Rpe{\mathbb R_+^*}
\def\C{\mathbb C}
\def\K{\mathbb K}
\def\L{\mathbb L}

\def\P{\mathscr{P}}

\def\Ker{\text{Ker}}
\def\ord{\text{ord}}

\def\ssi{\Leftrightarrow}
\def\Ssi{\Longleftrightarrow}
\def\implique{\Longrightarrow}

\def\sep{\noindent\makebox[\linewidth]{\rule{\paperwidth}{0.4pt}}}

% CUSTOM TITLE
\makeatletter
\def\@maketitle{%
	\newpage
	%  \null% DELETED
	%  \vskip 2em% DELETED
	\begin{center}%
		\let \footnote \thanks
		{\LARGE\bfseries \@title \par}%
		\vskip 1em%
		{\large
			\lineskip .5em%
			\begin{tabular}[t]{c}%
				\@author
			\end{tabular}\par}%
		\vskip 1em%
		{\large \@date}%
	\end{center}%
	\par
	\vskip -1em}
\makeatother

\title{Semaine 7\\ Exercices}
\author{Amar AHMANE}

\begin{document}
	\maketitle
	\begin{center}
		\textit{Je m'en allais, les poings dans mes poches crevées ;}\\
		\textit{Mon paletot aussi devenait idéal ;}\\
		\textit{J'allais sous le ciel, Muse ! et j'étais ton féal ;}\\
		\textit{Oh ! là ! là ! que d'amours splendides j'ai rêvées !}\\
	\end{center}
	\begin{flushright}
		\textit{Ma bohème}, Arthur Rimbaud
	\end{flushright}
	Quelques définitions qui vont servir pour la résolution des exercices : 
	\begin{itemize}
		\item[\bfseries Définition 1] Lorsque $G$ est un groupe fini, on appelle ordre de $G$ son cardinal.
		\item[\bfseries Définition 2] Soit $G$ un groupe multiplicatif. Soit $g\in G$, alors on définit \[\ord(g)=\min\lbrace n\in\N^*\ | \ g^n=1\rbrace\]
		Par convention, cette quantité est $+\infty$ si l'ensemble considéré ci-dessus est vide.
		\item[\bfseries Définition 3] Soit $n\in\N^*$. On note $\mathbb P$ l'ensemble des nombres premiers. Cet entier admet une décomposition nombres premiers (résultat admis) que l'on choisit de noter de la sorte : \[n=\prod_{p\in\mathbb P}p^{v_p(n)}\]
		où $v_p(n)$ est l'exposant de $p$ dans la décomposition en nombre premiers de $n$. Dans ce cas, le pgcd de deux entiers $m$ et $n$ non nuls est l'entier \[n\wedge m=\prod_{p\in\mathbb P}p^{\min(v_p(n),v_p(m))}\]
		\item[\bfseries Définition 4] On dit que deux entiers sont premiers entre eux si leur pgcd est égal à 1.
	\end{itemize}
	Quelques résultats utiles :
	\begin{itemize}
		\item Théorème de Lagrange : l'ordre de chaque sous-groupe d'un groupe $G$ divise l'ordre de $G$.
		\item Lemme de Gauss : lorsque $p$ et $q$ sont deux entiers premiers entre eux, et lorsque $p|mq$ où $m$ est un entier, on a que $p|m$.
	\end{itemize}
	\subsection*{Exercice : ordre du produit de deux éléments dont les ordres sont premiers entre eux (source : Oraux X-ENS, Algèbre 1, Cassini)}
	Soit $G$ un groupe abélien fini. Pour tout $x\in G$, on note $\ord(x)$ l'ordre de $x$.
	Soit $(x,y)\in G^2$, $m=\ord(x)$, $n=\ord(y)$. On suppose que $m$ et $n$ sont premiers entre eux. Montrer que $\ord(xy)=mn$. \textit{On se servira du lemme de Gauss.}
	\subsection*{Exercice : quelques exemples de sous-groupes (source : Les maths en tête, Xavier Gourdon)}
	Soit $G$ un groupe, $H_1$ et $H_2$ deux sous-groupes de $G$.
	\begin{enumerate}
		\item On suppose que $H_1\cup H_2$ est un groupe de $G$, montrer alors que $H_1\subset H_2$ ou $H_2\subset H_1$.
		\item Si les ordres de $H_1$ et $H_2$ sont finis et premiers entre eux, que dire de $H_1\cap H_2$ ?
	\end{enumerate}
	\subsection*{Exercice : cardinal d'un groupe fini et Im et Ker. (source : Oraux X-ENS, Algèbre 1, Cassini)}
	Soit $G$ un groupe fini, et $f$ un morphisme de $G$ dans lui-même.
	\begin{enumerate}
		\item Montrer que la relation $\mathcal R$ définie par $\forall x,y\in G,\quad x\mathcal R y\ssi f(x)=f(y)$ est une relation d'équivalence. 
		\item On note $G/\mathcal R$ l'ensemble des classes d'équivalence des éléments de $G$. Montrer que $|G/\mathcal R|=|\Im f|$.
		\item On note $G/\Ker f=\lbrace x\Ker f;\ x\in G\rbrace$. Pour $x\in G$, on note $\overline x=x\Ker f$. On munit $G/\Ker f$ de la loi \[\overline x\cdot\overline y=\overline{x\cdot y}\]
		Montrer que cette loi munit $G/\Ker f$ d'une structure de groupe.
		\item Montrer que $G/\Ker f=G/\mathcal R$.
		\item Montrer que pour tout $x\in G$, $|\overline x|=|\Ker f|$. En déduire que $|G|=|G/\mathcal R||\Ker f|$.
		\item En déduire que $\Ker f^2=\Ker f\ssi \Im f=\Im f^2$.
	\end{enumerate}
\end{document}