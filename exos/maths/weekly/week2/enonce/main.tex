% Preamble
\documentclass[10pt]{article}

% Math related packages.
\usepackage{amsfonts, amsthm, amsmath, amssymb, amstext}
\usepackage{mathtools}
\usepackage{physics}
\usepackage{cancel, textcomp}
\usepackage[mathscr]{euscript}
\usepackage[nointegrals]{wasysym}
\usepackage[left=3cm, right=3cm, top=1.5cm, bottom=1.5cm]{geometry}

\usepackage{esvect}
\usepackage{IEEEtrantools}

% Font related packages.
\usepackage[french]{babel}
\usepackage[unicode]{hyperref}
\usepackage[fontsize = 10pt]{scrextend}
\usepackage[T1]{fontenc}
\usepackage[utf8]{inputenc}
\usepackage[finemath]{kotex}
\usepackage{dhucs-nanumfont}
\usepackage{mathpazo}
\usepackage{FiraMono}
\usepackage{mathrsfs}

% Other
\usepackage{enumerate}
\usepackage[shortlabels]{enumitem}
\usepackage{tabularx}
\usepackage[object=vectorian]{pgfornament}
\usepackage{pgf}
\usepackage{pgfpages}
\usepackage[european,straightvoltages]{circuitikz}
\usepackage{scalerel}
\usepackage{stackengine}


% A more colorful world.
\usepackage{xcolor}

% Display Source Code
\usepackage{listings}
\lstset{
    language=C,
    basicstyle=\small\ttfamily\mdseries,
    numberstyle=\color{gray},
    stringstyle=\color[HTML]{933797},
    commentstyle=\color[HTML]{228B22},
    emph={[2]from,with,import,as,pass,return,and,or,not},
    emphstyle={[2]\color[HTML]{DD52F0}},
    emph={[3]range,format,enumerate,print},
    emphstyle={[3]\color[HTML]{D17032}},
    emph={[4]if,elif,else,for,while,in,def,lambda,int,float,all,len},
    emphstyle={[4]\color{blue}},
    emph={[5]abs},
    emphstyle={[5]\color{black}},
    showstringspaces=false,
    breaklines=true,
    prebreak=\mbox{{\color{gray}\tiny$\searrow$}},
    numbers=left,
    xleftmargin=15pt
}

% OPTIONS
\everymath\expandafter{\the\everymath\displaystyle}

% COMMANDS
\newcommand{\f}[1]{\texttt{#1}}
\newcommand{\sct}[1]{
	\begin{center}
		\Large\textbf{#1}
	\end{center}
}
\newcommand{\subsct}[1]{
	\begin{center}
		\large\textbf{#1}
	\end{center}
}
\newcommand{\inl}[2]{[\![#1, #2]\!]}
\newcommand{\q}[1]{\textbf{#1.}\quad}
\newcommand{\urlsymbol}{\kern1pt\vbox to .5ex{}\raise.10ex\hbox{\pdfliteral{%
    q .8 0 0 .8 0 0 cm
    2.5 5 m 1 j 1 J .8 w
    1 5 l 0 5 0 4 y 0 1 l 0 0 1 0 y 4 0 l 5 0 5 1 y 5 2.5 l S
    3 3 m 6 6 l S 4 6 m 6 6 l 6 4 l S
Q}}\kern5pt}

\newcounter{iloop}
\newcommand\openbigstar[1][0.7]{%
  \scalerel*{%
    \stackinset{c}{-.125pt}{c}{}{\scalebox{#1}{\color{white}{$\bigstar$}}}{%
      $\bigstar$}%
  }{\bigstar}
}
\newcommand{\Stars}[1]{\ensuremath{%
\pgfmathtruncatemacro{\imax}{ifthenelse(int(#1)==#1,#1-1,#1)}%
\pgfmathsetmacro{\xrest}{0.9*(1-#1+\imax)}%
\setcounter{iloop}{0}%
\loop\stepcounter{iloop}\ifnum\value{iloop}<\the\numexpr\imax+1
\bigstar\repeat
\openbigstar[\xrest]%
\setcounter{iloop}{0}%
\loop\stepcounter{iloop}\ifnum\value{iloop}<\the\numexpr5-\imax\relax
\openbigstar[.9]\repeat}}


\def\N{\mathbb N}
\def\Z{\mathbb Z}
\def\Q{\mathbb Q}
\def\R{\mathbb R}
\def\Rpe{\mathbb R_+^*}
\def\C{\mathbb C}
\def\K{\mathbb K}
\def\L{\mathbb L}

\def\P{\mathscr{P}}

\def\ssi{\Leftrightarrow}
\def\Ssi{\Longleftrightarrow}
\def\implique{\Longrightarrow}

\def\sep{\noindent\makebox[\linewidth]{\rule{\paperwidth}{0.4pt}}}

% CUSTOM TITLE
\makeatletter
\def\@maketitle{%
	\newpage
	%  \null% DELETED
	%  \vskip 2em% DELETED
	\begin{center}%
		\let \footnote \thanks
		{\LARGE \@title \par}%
		\vskip 1em%
		{\large
			\lineskip .5em%
			\begin{tabular}[t]{c}%
				\@author
			\end{tabular}\par}%
		\vskip 1em%
		{\large \@date}%
	\end{center}%
	\par
	\vskip -1em}
\makeatother

\title{Semaine 1}

\begin{document}
	\maketitle
	\subsubsection*{\textsc{Louis Marchal} \Stars{4}}
	Soient $a$ et $b$ deux fonctions continues telles que $a\geq 1$ sur $\R$.
	\begin{enumerate}
		\item Montrer que si $\lim_{x\to +\infty}b(x)=0$, alors toute solution de l'équation différentielle $y'+a(x)y=b(x)$ vérifie $\lim_{x\to +\infty}y(x)=0$.
		\item Montrer que si $b\geq 0$ et $\lim_{x\to -\infty}b(x)=0$, alors il existe une et une seule solution de l'équation $y'+a(x)y=b(x)$ telle que $\lim_{x\to -\infty}y(x)=0$. Que peut-on dire des limites des autres solutions ?
	\end{enumerate}
	\subsubsection*{\textsc{Pierre-Gabriel Berlureau} \Stars{3}}
	En remplaçant $\tan x$ par $\frac{\sin x}{\cos x}$, calculer l'intégrale suivante \[I=\int_0^{\frac\pi4}\ln(1+\tan x)\dd x\]
	\subsubsection*{\textsc{Yanis Grigy}\Stars{3}}
	Calculer
	\begin{tabular}{c c c}
		\textbf{a)} $I_n=\int_0^{\frac\pi4} \tan^n x\dd x$ & \textbf{b)} $I_n=\int_0^{\frac\pi4} \frac{\dd x}{\cos^n(x)}$ & \textbf{c)} $I_n=\int_1^e \log^n x\dd x$
	\end{tabular}.
	\subsubsection*{\textsc{Shems} \Stars{4}}
	\begin{enumerate}
		\item Montrer que \[\forall x\in\R,\ \forall n\in\N,\quad \cos((n+1)x)=2\cos(nx)\cos(x)-\cos((n-1)x).\]
		\item En déduire que pour tout $n\in\N$, il existe un polynôme, que l'on note $T_n$, tel que \[\forall \theta\in\R,\quad 2\cos(n\theta)=T_n(2\cos(\theta)).\]
		\item Soit la fonction polynômiale $P:x\in\R\mapsto\sum_{i=0}^ka_ix^i\in\R$, où $k\in\N^*$ et $a_0,\dots,a_{k-1}\in\Z$ et $a_k=1$.\\

		Soit $a$ une racine rationnelle de $P$. En écrivant a sous la forme $a=\frac pq$ avec $(p,q)\in\Z\times\N^*$, $p$ et $q$ premiers entre eux, montrer que $a\in\Z$.
		\item (En bonus) En déduire le théorème de Niven :\\

		Soit $\theta\in[0,\frac\pi2]$ tel que $\frac\theta\pi\in\Q$. Montrer que si $\cos\theta\in\Q$, alors $\cos\theta\in\left\lbrace0,\frac12,1\right\rbrace$.
	\end{enumerate}
	\subsubsection*{\textsc{Matteo Delfour} \Stars{2}}
	À l'aide d'un changement de variable judicieux, calculer \[\int^x\frac{dt}{\sqrt{1+t}-\sqrt[3]{1+t}}\]

\end{document}