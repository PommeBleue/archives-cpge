% Preamble
\documentclass[10pt]{article}

% Math related packages.
\usepackage{amsfonts, amsthm, amsmath, amssymb, amstext}

\usepackage{mathtools}
\usepackage{physics}
\usepackage{cancel, textcomp}
\usepackage[mathscr]{euscript}
\usepackage[nointegrals]{wasysym}
\usepackage[left=3cm, right=3cm, top=1.5cm, bottom=1.5cm]{geometry}

\usepackage{esvect}
\usepackage{IEEEtrantools}

% Font related packages.
\usepackage[french]{babel}
\usepackage[unicode]{hyperref}
\usepackage[fontsize = 10pt]{scrextend}
\usepackage[T1]{fontenc}
\usepackage[utf8]{inputenc}
\usepackage[finemath]{kotex}
\usepackage{dhucs-nanumfont}
\usepackage{mathpazo}
\usepackage{FiraMono}
\usepackage{mathrsfs}

% Other
\usepackage{enumerate}
\usepackage[shortlabels]{enumitem}
\usepackage{tabularx}
\usepackage[object=vectorian]{pgfornament}
\usepackage{pgf}
\usepackage{pgfpages}
\usepackage[european,straightvoltages]{circuitikz}
\usepackage{scalerel}
\usepackage{stackengine}


% A more colorful world.
\usepackage{xcolor}

% Display Source Code
\usepackage{listings}
\lstset{
    language=C,
    basicstyle=\small\ttfamily\mdseries,
    numberstyle=\color{gray},
    stringstyle=\color[HTML]{933797},
    commentstyle=\color[HTML]{228B22},
    emph={[2]from,with,import,as,pass,return,and,or,not},
    emphstyle={[2]\color[HTML]{DD52F0}},
    emph={[3]range,format,enumerate,print},
    emphstyle={[3]\color[HTML]{D17032}},
    emph={[4]if,elif,else,for,while,in,def,lambda,int,float,all,len},
    emphstyle={[4]\color{blue}},
    emph={[5]abs},
    emphstyle={[5]\color{black}},
    showstringspaces=false,
    breaklines=true,
    prebreak=\mbox{{\color{gray}\tiny$\searrow$}},
    numbers=left,
    xleftmargin=15pt
}

% OPTIONS
\everymath\expandafter{\the\everymath\displaystyle}

% COMMANDS
\newcommand{\f}[1]{\texttt{#1}}
\newcommand{\sct}[1]{
	\begin{center}
		\Large\textbf{#1}
	\end{center}
}
\newcommand{\subsct}[1]{
	\begin{center}
		\large\textbf{#1}
	\end{center}
}
\newcommand{\inl}[2]{[\![#1, #2]\!]}
\newcommand{\q}[1]{\textbf{#1.}\quad}
\newcommand{\urlsymbol}{\kern1pt\vbox to .5ex{}\raise.10ex\hbox{\pdfliteral{%
    q .8 0 0 .8 0 0 cm
    2.5 5 m 1 j 1 J .8 w
    1 5 l 0 5 0 4 y 0 1 l 0 0 1 0 y 4 0 l 5 0 5 1 y 5 2.5 l S
    3 3 m 6 6 l S 4 6 m 6 6 l 6 4 l S
Q}}\kern5pt}

\newcounter{iloop}
\newcommand\openbigstar[1][0.7]{%
  \scalerel*{%
    \stackinset{c}{-.125pt}{c}{}{\scalebox{#1}{\color{white}{$\bigstar$}}}{%
      $\bigstar$}%
  }{\bigstar}
}
\newcommand{\Stars}[1]{\ensuremath{%
\pgfmathtruncatemacro{\imax}{ifthenelse(int(#1)==#1,#1-1,#1)}%
\pgfmathsetmacro{\xrest}{0.9*(1-#1+\imax)}%
\setcounter{iloop}{0}%
\loop\stepcounter{iloop}\ifnum\value{iloop}<\the\numexpr\imax+1
\bigstar\repeat
\openbigstar[\xrest]%
\setcounter{iloop}{0}%
\loop\stepcounter{iloop}\ifnum\value{iloop}<\the\numexpr5-\imax\relax
\openbigstar[.9]\repeat}}


\def\N{\mathbb N}
\def\Z{\mathbb Z}
\def\Q{\mathbb Q}
\def\R{\mathbb R}
\def\Rpe{\mathbb R_+^*}
\def\C{\mathbb C}
\def\K{\mathbb K}
\def\L{\mathbb L}

\def\P{\mathscr{P}}

\def\ssi{\Leftrightarrow}
\def\Ssi{\Longleftrightarrow}
\def\implique{\Longrightarrow}

\def\sep{\noindent\makebox[\linewidth]{\rule{\paperwidth}{0.4pt}}}

% CUSTOM TITLE
\makeatletter
\def\@maketitle{%
	\newpage
	%  \null% DELETED
	%  \vskip 2em% DELETED
	\begin{center}%
		\let \footnote \thanks
		{\LARGE \@title \par}%
		\vskip 1em%
		{\large
			\lineskip .5em%
			\begin{tabular}[t]{c}%
				\@author
			\end{tabular}\par}%
		\vskip 1em%
		{\large \@date}%
	\end{center}%
	\par
	\vskip -1em}
\makeatother

\title{Semaine 1}

\begin{document}
	\maketitle
	\subsubsection*{\textsc{Louis Marchal} \Stars{4}}
	\begin{enumerate}
		\item On exprime la solution générale comme somme d'une solution de l'équation homogène et d'une solution particulière. L'une tend vers $0$ en $+\infty$ puisque $A\geq 0$, il suffit de justifier que la solution particulière tend vers $0$ en $+\infty$. Pour rappel : on obtient une solution particulière grâce à la méthode de la variation de la constante, c'est une primitive de $x\mapsto e^{A(x)}b(x)$, que l'on obtient grâce au $TFA$.
		\item Même type d'argument, s'assurer de la convergence de la solution générale.
	\end{enumerate}
	\subsubsection*{\textsc{Pierre-Gabriel Berlureau} \Stars{3}}
	Premièrement, on a
	\begin{align*}
		\int_0^\frac\pi4 \ln(1+\tan x)\dd x &= \int_0^\frac\pi4 \ln(\cos(x)+\sin(x))-\int_0^\frac\pi4\cos(x)\dd x\\
											&= \int_0^\frac\pi4 \ln(\sqrt{2}\cos(x-\frac\pi4))\dd x-\int_0^\frac\pi4\cos(x)\dd x\\
											&=\ln(\sqrt{2})\frac\pi4+\int_0^\frac\pi4\cos(x-\frac\pi4)-\int_0^\frac\pi4\cos(x)\dd x
	\end{align*}
	Or, en posant le changement de variable $t=\frac\pi4 - x$, on a $\int_0^\frac\pi4\cos(x-\frac\pi4)=\int_\frac\pi4^0-\cos t\dd t=\int_0^\frac\pi4\cos t\dd t$.\\
	Finalement, $\boxed{I=\ln(\sqrt{2})\frac\pi4}$.
	\subsubsection*{\textsc{Yanis Grigy}\Stars{3}}
	Ton exercice consistait en un calcul d'intégrales. La petite difficulté était ici introduite par le paramètre entier naturel $n$. Mais rassure ta mère (je ne sais pas pourquoi j'ai dit ça, j'allais écrire rassure-toi et j'ai pensé à la vidéo de Cyprien), ce n'est pas si difficile que ça si on fait bien les choses !
	\begin{enumerate}[a)]
		\item On commence ici par chercher une relation de récurrence, i.e on veut exprimer $I_n$ en fonction des termes précédents. Pour cela, le plus évident est de recourir aux intégrations par parties, on a
		\begin{align*}
			I_n = \int_0^\frac\pi4 \tan^n (x) \dd x &=\int_0^\frac\pi4 \tan^2(x)(\tan^{n-2}(x))\dd x\\
													&= \int_0^\frac\pi4 (1+\tan^2(x)-1)(\tan^{n-2}(x))\dd x\\
													&=\left[\tan(x)\tan^{n-2}(x)\right]_0^\frac\pi4-\int_0^\frac\pi4\tan(x)(1+\tan^2(x))(n-2)\tan(x)^{n-3}\dd x -I_{n-2}\\
													&=1-(n-2)[I_{n-2}+I_n]-I_{n-2}\\
													&=1-(n-1)I_{n-2}-(n-2)I_n
		\end{align*}
		Finalement, on a $\boxed{I_n=\frac1{n-1}-I_{n-2}}$.\\
		Ainsi, si $n$ est impair, on a \[I_n=\sum_{1\leq 2k+1\leq n-2}\frac{(-1)^k}{n-(2k+1)}+(-1)^{\lfloor\frac n2\rfloor}\ln\sqrt2\]
		Et, sinon \[I_n=\sum_{1\leq 2k+1\leq n-1}\frac{(-1)^k}{n-(2k+1)}+(-1)^{\frac n2}\frac\pi4\]
		\item L'idée est exactement la même, notre relation de récurrence ici est : $\boxed{I_n=\frac{(\sqrt{2})^{n-2}}{n-1}+\frac{n-2}{n-1}I_{n-2}}$.\\
		Ainsi, si $n$ est impair, on a \[I_n=\sum_{k=0}^{\frac{n-1}2-1}\frac{(\sqrt{2})^{n-2(k+1)}\prod_{i=0}^{k-1}(n-2(i+1))}{\prod_{i=0}^k(n-(2k+1))}+\frac{\prod_{i=1}^{\frac{n-1}2}(n-2i)}{\prod_{i=0}^{\frac{n-1}2-1}(n-(2i+1))}\ln\left(\tan\left(\frac{3\pi}8\right)\right)\]
		Et, sinon \[I_n=\sum_{k=0}^{\frac{n-1}2-1}\frac{(\sqrt{2})^{n-2(k+1)}\prod_{i=0}^{k-1}(n-2(i+1))}{\prod_{i=0}^k(n-(2k+1))}\]
		\item Ici, notre relation de récurrence est $I_n=\left[x\ln^nx\right]_1^e-n\int_1^e\ln^{n-1}(x)\dd x=e-nI_{n-1}$.\\
		Finalement, \[I_n=\sum_{k=0}^{n-1}(-1)^kn^ke+(-1)^n n^n(e-1)\]

	\end{enumerate}
	\subsubsection*{\textsc{Shems} \Stars{4}}
	\begin{enumerate}
		\item Soient $n\in\N$ et $x\in\R$. On a alors \[\cos((n+1)x)+\cos((n-1)x)=2\cos\frac{2n}2x\cos\frac{2x}2=2\cos(nx)\cos x\]
		Donc $\boxed{\cos(n+1)x=2\cos(nx)\cos x-\cos(n-1)x}$.
		\item On le montre par récurrence double. Pour tout $n\in\N$, on note \[\mathcal P_n:"\exists T_n\in\Z[X],\ \forall \theta\in\R,\  2\cos(n\theta)=T_n(2\cos\theta)"\]
		\begin{itemize}
			\item On pose $T_0=2$, $T_1=X$, d'où $\forall \theta\in\R,\ \widetilde{T_0}(2\cos(0\times \theta))=2=2\cos(0\times \theta)$ et $\forall \theta\in\R,\ \widetilde{T_1}(2\cos(x))=2\cos(x)$. Donc $\mathcal P_0$ et $\mathcal P_1$.
			\item Soit $n\in\N$ tel que $\mathcal P_n$ et $\mathcal P_{n+1}$. Montrons $\mathcal P_{n+2}$.\\
			On a montré que \[\forall\theta\in\R, \cos((n+2)\theta)=2\cos((n+1)\theta)\cos\theta-\cos(n\theta)\]
			D'où \[\forall\theta\in\R,\ 2\cos((n+2)\theta)=2\cos((n+1)\theta)2\cos\theta-2\cos(n\theta)\]
			Ainsi, on a que \[\forall\theta\in\R,\ 2\cos((n+2)\theta)=2\cos\theta\widetilde{T_{n+1}}(2\cos\theta)-\widetilde{T_n}(2\cos\theta)\]
			On construit alors $T_{n+2}$ grâce à $T_{n+1}$ et $T_n$ : \[T_{n+1}=XT_{n+1}-T_n\]
			$\mathcal P_{n+2}$.
		\item OK.
		\end{itemize}
		\item Soit $a$ une racine rationnelle de $P$, il existe alors $(p,q)\in\Z\times\N^*$ tel que $a=\frac pq$. Ainsi, on a \[\widetilde{P}(a)=0\]
		\def\a{\left(\frac pq\right)}
		Donc \[\a^k+a_{k-1}\a^{k-1}+\dots+a_1\a+a_0=0\]
		D'où \[\a^k=-a_{k-1}\a^{k-1}-\dots-a_1\a-a_0\]
		Donc, en multipliant par $q^k$, \[p^k=-q\left(a_{k-1}\a^{k-1}q^{k-1}+\dots+a_1\a q^{k-1}+a_0q^{k-1}\right)\]
		Ainsi, $q$ divise $p^k$, d'où $q$ divise $p$, donc $a\in\Z$.
		\item Il existe $(p_1,q_1)\in\Z\times\N^*$ tel que $\theta=\frac{p_1\pi}{q_1}$. On a \[2\cos(q_1\theta)=\widetilde{T_{q_1}}(2\cos\theta)\]
		Or, $2\cos(q_1\theta)=2\cos(p_1\pi)\in\Z$. Donc $2\cos\theta$ est un rationnel racine d'un polynôme de degré $q_1$ à coefficients entiers, donc $2\cos\theta\in\Z$. Or, comme $\theta\in\left[0,\frac\pi2\right]$, on a $2\cos\theta\in\lbrace0,1,2\rbrace$, d'où $\cos\theta\in\left\lbrace0,\frac12,1\right\rbrace$.
	\end{enumerate}
	\subsubsection*{\textsc{Matteo Delfour} \Stars{2}}
	En posant le changement de variable $u=\sqrt[6]{1+t}$, d'où $6u^5\dd u=\dd t$, on a 
	\begin{align*}
		\int^x\frac{\dd t}{\sqrt{1+t}-\sqrt[3]{1+t}}&=\int^{\sqrt[6]{1+x}}\frac{6u^5\dd u}{u^3-u^2}\\
													&=\int^{\sqrt[6]{1+x}}\frac{6u^3\dd u}{u-1}\\
													&=\int^{\sqrt[6]{1+x}}\frac{6u^2(u-1+1)\dd u}{u-1}\\
													&=\int^{\sqrt[6]{1+x}}\left(6u^2+\frac{6u(u-1+1)}{u-1}\right)\dd u\\
													&=\int^{\sqrt[6]{1+x}}\left(6u^2+6u+\frac{6(u-1+1)}{u-1}\right)\dd u\\
													&=\int^{\sqrt[6]{1+x}}\left(6u^2+6u+6+6\frac1{u-1}\right)\dd u\\
													&=2\sqrt{1+x}+3\sqrt[3]{1+x}+6\sqrt[6]{1+x}+\ln|\sqrt[6]{1+x}-1|+C,\ \ C\in\R
	\end{align*}

\end{document}