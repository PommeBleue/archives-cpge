% Preamble
\documentclass[17pt]{article}

% Math related packages.
\usepackage{amsfonts, amsthm, amsmath, amssymb, amstext}
\usepackage{mathtools}
\usepackage{physics}
\usepackage{cancel, textcomp}
\usepackage[mathscr]{euscript}
\usepackage[nointegrals]{wasysym}
\usepackage[left=3cm, right=3cm, top=1.5cm, bottom=1.5cm]{geometry}

\usepackage{esvect}
\usepackage{IEEEtrantools}

% Font related packages.
\usepackage[french]{babel}
\usepackage[unicode]{hyperref}
\usepackage[fontsize = 10pt]{scrextend}
\usepackage[T1]{fontenc}
\usepackage[utf8]{inputenc}
\usepackage[finemath]{kotex}
\usepackage{dhucs-nanumfont}
\usepackage{mathpazo}
\usepackage{FiraMono}
\usepackage{mathrsfs}

% Other
\usepackage{enumerate}
\usepackage[shortlabels]{enumitem}
\usepackage{tabularx}
\usepackage[object=vectorian]{pgfornament}
\usepackage{pgf}
\usepackage{pgfpages}
\usepackage[european,straightvoltages]{circuitikz}
\usepackage{scalerel}
\usepackage{stackengine}


% A more colorful world.
\usepackage{xcolor}

% Display Source Code
\usepackage{listings}
\lstset{
    language=C,
    basicstyle=\small\ttfamily\mdseries,
    numberstyle=\color{gray},
    stringstyle=\color[HTML]{933797},
    commentstyle=\color[HTML]{228B22},
    emph={[2]from,with,import,as,pass,return,and,or,not},
    emphstyle={[2]\color[HTML]{DD52F0}},
    emph={[3]range,format,enumerate,print},
    emphstyle={[3]\color[HTML]{D17032}},
    emph={[4]if,elif,else,for,while,in,def,lambda,int,float,all,len},
    emphstyle={[4]\color{blue}},
    emph={[5]abs},
    emphstyle={[5]\color{black}},
    showstringspaces=false,
    breaklines=true,
    prebreak=\mbox{{\color{gray}\tiny$\searrow$}},
    numbers=left,
    xleftmargin=15pt
}

% OPTIONS
\everymath\expandafter{\the\everymath\displaystyle}

% COMMANDS
\newcommand{\f}[1]{\texttt{#1}}
\newcommand{\sct}[1]{
	\begin{center}
		\Large\textbf{#1}
	\end{center}
}
\newcommand{\subsct}[1]{
	\begin{center}
		\large\textbf{#1}
	\end{center}
}
\newcommand{\inl}[2]{[\![#1, #2]\!]}
\newcommand{\q}[1]{\textbf{#1.}\quad}
\newcommand{\urlsymbol}{\kern1pt\vbox to .5ex{}\raise.10ex\hbox{\pdfliteral{%
    q .8 0 0 .8 0 0 cm
    2.5 5 m 1 j 1 J .8 w
    1 5 l 0 5 0 4 y 0 1 l 0 0 1 0 y 4 0 l 5 0 5 1 y 5 2.5 l S
    3 3 m 6 6 l S 4 6 m 6 6 l 6 4 l S
Q}}\kern5pt}

\newcounter{iloop}
\newcommand\openbigstar[1][0.7]{%
  \scalerel*{%
    \stackinset{c}{-.125pt}{c}{}{\scalebox{#1}{\color{white}{$\bigstar$}}}{%
      $\bigstar$}%
  }{\bigstar}
}
\newcommand{\Stars}[1]{\ensuremath{%
\pgfmathtruncatemacro{\imax}{ifthenelse(int(#1)==#1,#1-1,#1)}%
\pgfmathsetmacro{\xrest}{0.9*(1-#1+\imax)}%
\setcounter{iloop}{0}%
\loop\stepcounter{iloop}\ifnum\value{iloop}<\the\numexpr\imax+1
\bigstar\repeat
\openbigstar[\xrest]%
\setcounter{iloop}{0}%
\loop\stepcounter{iloop}\ifnum\value{iloop}<\the\numexpr5-\imax\relax
\openbigstar[.9]\repeat}}


\def\N{\mathbb N}
\def\Z{\mathbb Z}
\def\Q{\mathbb Q}
\def\R{\mathbb R}
\def\Rpe{\mathbb R_+^*}
\def\C{\mathbb C}
\def\K{\mathbb K}
\def\L{\mathbb L}

\def\P{\mathscr{P}}

\def\ord{\text{ord}}
\def\Ker{\text{Ker}}

\def\ssi{\Leftrightarrow}
\def\Ssi{\Longleftrightarrow}
\def\implique{\Longrightarrow}

\def\sep{\noindent\makebox[\linewidth]{\rule{\paperwidth}{0.4pt}}}

% CUSTOM TITLE
\makeatletter
\def\@maketitle{%
	\newpage
	%  \null% DELETED
	%  \vskip 2em% DELETED
	\begin{center}%
		\let \footnote \thanks
		{\LARGE\bfseries \@title \par}%
		\vskip 1em%
		{\large
			\lineskip .5em%
			\begin{tabular}[t]{c}%
				\@author
			\end{tabular}\par}%
		\vskip 1em%
		{\large \@date}%
	\end{center}%
	\par
	\vskip -1em}
\makeatother

\title{Semaine 6}
\author{Pomme Bleue}

\begin{document}
	\maketitle
	\begin{center}
		\bfseries\Large\textsc{Pierre-Gabriel Berlureau}
	\end{center}
	\subsection*{Congruences modulo un sous-groupe et Théorème de Lagrange}
	On a 
	\begin{enumerate}[i)]
		\item L'application $f:h\in H\mapsto xh$ est une bijection. En effet, il est clair que $\Ker f = \lbrace e\rbrace$, donc $f$ est injective; la surjectivité est claire, d'où que $f$ est bijective, donc $|H|=|xH|$.
		\item On montre que la relation $\mathcal R$ définie par \[x\mathcal R y\ssi y\in xH\]
		est une relation d'équivalence et que la classe d'équivalence de $x\in H$ est $xH$.
	\end{enumerate}
	$G/\mathcal R$ est une partition de $G$ de parts toutes égales, donc on a \[G=\bigcup_{H\in G/\mathcal R}H\]
	d'où que, en passant au cardinal, $G=\sum_{H\in G/\mathcal R}|H|=|G/\mathcal R|\times|H|$.
	
	\begin{center}
		\bfseries\Large\textsc{Matteo Delfour}
	\end{center}
	\subsection*{Morphismes de $\Q$ dans $\Z$}
		\paragraph{Analyse} Soit $f\in\text{Hom}(\Q,\Z)$, alors $f(\Z)$ est un sous-groupe de $(\Z, +)$, ainsi il existe $n\in\N$ tel que $f(\Z)=n\Z$. Il en découle qu'il existe $x\in Q$ tel que $f(x)=n$, d'où que $2f(\frac x2)=n$ donc $f(\frac x2)=\frac n2$ donc $\frac n2\in n\Z$, ceci n'est possible que si $n=0$, donc $f$ est la fonction nulle.
		\paragraph{Syntèse} Bla bla...
	\begin{center}
		\bfseries\Large\textsc{Yanis Grigy}
	\end{center}
	\subsection*{Petit Lemme}
	\paragraph{Méthode 1 :} la récurrence bizarre.\\


	On que tout élément de $G$ est égal à son inverse, donc, lorsque $x,y\in G$, $xy=(xy)^{-1}=y^{-1}x^{-1}=yx$.\\
	On montre à présent par récurrence sur $|G|$ que $|G|$ est une puissance de $2$.\\
	Lorsque $|G|=1$, il n'y a rien à vérifier.
	Lorsque $|G|\geq 2$, on note $H$ le sous-groupe de $G$ maximal pour l'inclusion tel que $G\neq H$. Soit $a\in G\backslash H$, alors $H\cup aH$ est un groupe, mais $H\cap aH=\emptyset$. D'autre part $|H\cup aH|=2|H|$ d'après la propriété de Pierre-Gabriel, ainsi $H\cup aH$ est un sous-groupe de $G$ ayant un cardinal strictement plus grand que celui de $H$, donc $H\cup aH=G$, d'où que $|G|=2|H|$. D'après notre hypothèse de récurrence, $|G|$ est une puissance de 2 puisque $|H|$ est une puissance de $2$. 
	\paragraph{Méthode 2 :} la méthode élégante qui utilise de l'algèbre linéaire.\\


	On remarque que $G$ est un $\Z/2\Z$-ev lorsque que l'on définit la loi de composition externe $\cdot$ telle que $0\cdot x=1_G$ et $1\cdot x = x$; on laissera le soin au lecteur de vérifier les axiomes. On a directement, avec $\dim G$ fini, \[G\simeq (\Z/2\Z)^{\dim G}\]
	Ce qui conclut.

	\newpage
		
	\begin{center}
		\bfseries\Large\textsc{Louis Marchal}
	\end{center}
	\subsection*{Groupes dont l'ensemble des sous-groupes est fini}
		Soit $G$ un groupe. On note $E$ l'ensemble de ses sous-groupes, on suppose qu'il est fini.\\

		Les éléments de $G$ sont tous d'ordre fini : en effet, si $g\in G$ est tel que $\ord(g)=+\infty$, alors on a que $<g>\simeq \Z$ donc $<g>$ a une infinité de sous-groupes, ce qui ne peut arriver puisque $E$ est fini. On note $E'$ l'ensemble des sous-groupes monogènes de $G$, alors $\bigcup_{H\in E'}H=G$. Ainsi, $G$ est fini puisqu'il est une union finie de groupes finis. \\

		Remarques : $<g>$ désigne le plus petit sous-groupe de $G$ contenant $g$, qui est égal à $g\Z$ lorsque la loi de $G$ est notée additivement; on appelle le plus petit sous-groupe de $G$ contenant une partie $X$ de $G$, et on note $\text{Gr}(X)$, l'intersection de tous les sous-groupes de $G$ contenant $X$ : $\text{Gr}(X)$ est le sous-groupe monogène engendré par $X$. Un sous-groupe est dit homogène s'il est un sous-groupe homogène de lui-même.

	\begin{center}
		\bfseries\Large\textsc{Louis Thevenet}
	\end{center}
	\subsection*{Cas particulier du Lemme de Cauchy}
		D'après le théorème de Lagrange, les éléments de $G$ sont soit d'ordre $1$, $2$, $p$ ou $2p$. On suppose par l'absurde qu'il n'existe pas d'éléments d'ordre $p$ : il en découle assez rapidement qu'il n'existe pas non plus d'éléments d'ordre $2p$ (en effet, si $x\in G$ est d'ordre $2p$, $x^2$ est d'ordre $p$). On en déduit que tous les éléments sont d'ordre soit $1$ ou $2$ et $p\geq 3$, d'où \[\forall g \in G,\ g^2=1_G\]
		D'après le Lemme de Yanis, $G$ est une puissance de $2$.
		Or, $|G|=2p$, qui n'est pas une puissance de $2$. Voilà l'absurdité.

	\begin{center}
		\bfseries\Large\textsc{Armand sans nom de famille}
	\end{center}
	\subsection*{Existence d'un idempotent}
	Soit $a\in G$, l'application $f:n\in \N\mapsto a^{2^n}$ ne saurait être injective, puisque $E$ est fini; ainsi, il en découle qu'il existe $p,q\in \N$ deux entiers différents tels que $a^{2^p}=a^{2^q}$, ce qui peut être réécrit : $a^{2^{m+n}}=a^{2^{m}}$. On pose $b=a^{2^{m}}$, on a alors $b^{2^n}=b$, ainsi $b^{2^n-1}$ est idempotent : en effet, $(b^{2^n-1})^2=b^{2^{n}+2^{n}-2}=bb^{2^n-2}=b^{2^n-1}$. 

	\begin{center}
		\bfseries\Large\textsc{Shems}
	\end{center}
	\subsection*{Neutre à droite et inverse à droite}
	On montre que tout élément est inversible : soit $g\in G$, alors $g$ admet un inverse à droite $g'$ qui admet un inverse à droite $g''$ : ainsi, $gg'g''=eg''$ donc  $g'gg'g''=g'eg''$ donc $g'ge=g'g''$ donc $g'g=e$ donc $g'$ est inversible à gauche et $e$ est un neutre à gauche ce qui conclut.
\end{document}