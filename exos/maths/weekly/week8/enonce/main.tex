% Preamble
\documentclass[17pt]{article}

% Math related packages.
\usepackage{amsfonts, amsthm, amsmath, amssymb, amstext}
\usepackage{mathtools}
\usepackage{physics}
\usepackage{cancel, textcomp}
\usepackage[mathscr]{euscript}
\usepackage[nointegrals]{wasysym}
\usepackage[left=3cm, right=3cm, top=1.5cm, bottom=1.5cm]{geometry}

\usepackage{esvect}
\usepackage{IEEEtrantools}

% Font related packages.
\usepackage[french]{babel}
\usepackage[unicode]{hyperref}
\usepackage[fontsize = 10pt]{scrextend}
\usepackage[T1]{fontenc}
\usepackage[utf8]{inputenc}
\usepackage[finemath]{kotex}
\usepackage{dhucs-nanumfont}
\usepackage{mathpazo}
\usepackage{FiraMono}
\usepackage{mathrsfs}

% Other
\usepackage{enumerate}
\usepackage[shortlabels]{enumitem}
\usepackage{tabularx}
\usepackage[object=vectorian]{pgfornament}
\usepackage{pgf}
\usepackage{pgfpages}
\usepackage[european,straightvoltages]{circuitikz}
\usepackage{scalerel}
\usepackage{stackengine}


% A more colorful world.
\usepackage{xcolor}

% Display Source Code
\usepackage{listings}
\lstset{
    language=C,
    basicstyle=\small\ttfamily\mdseries,
    numberstyle=\color{gray},
    stringstyle=\color[HTML]{933797},
    commentstyle=\color[HTML]{228B22},
    emph={[2]from,with,import,as,pass,return,and,or,not},
    emphstyle={[2]\color[HTML]{DD52F0}},
    emph={[3]range,format,enumerate,print},
    emphstyle={[3]\color[HTML]{D17032}},
    emph={[4]if,elif,else,for,while,in,def,lambda,int,float,all,len},
    emphstyle={[4]\color{blue}},
    emph={[5]abs},
    emphstyle={[5]\color{black}},
    showstringspaces=false,
    breaklines=true,
    prebreak=\mbox{{\color{gray}\tiny$\searrow$}},
    numbers=left,
    xleftmargin=15pt
}

% OPTIONS
\everymath\expandafter{\the\everymath\displaystyle}

% COMMANDS
\newcommand{\f}[1]{\texttt{#1}}
\newcommand{\sct}[1]{
	\begin{center}
		\Large\textbf{#1}
	\end{center}
}
\newcommand{\subsct}[1]{
	\begin{center}
		\large\textbf{#1}
	\end{center}
}
\newcommand{\inl}[2]{[\![#1, #2]\!]}
\newcommand{\q}[1]{\textbf{#1.}\quad}
\newcommand{\urlsymbol}{\kern1pt\vbox to .5ex{}\raise.10ex\hbox{\pdfliteral{%
    q .8 0 0 .8 0 0 cm
    2.5 5 m 1 j 1 J .8 w
    1 5 l 0 5 0 4 y 0 1 l 0 0 1 0 y 4 0 l 5 0 5 1 y 5 2.5 l S
    3 3 m 6 6 l S 4 6 m 6 6 l 6 4 l S
Q}}\kern5pt}

\newcounter{iloop}
\newcommand\openbigstar[1][0.7]{%
  \scalerel*{%
    \stackinset{c}{-.125pt}{c}{}{\scalebox{#1}{\color{white}{$\bigstar$}}}{%
      $\bigstar$}%
  }{\bigstar}
}
\newcommand{\Stars}[1]{\ensuremath{%
\pgfmathtruncatemacro{\imax}{ifthenelse(int(#1)==#1,#1-1,#1)}%
\pgfmathsetmacro{\xrest}{0.9*(1-#1+\imax)}%
\setcounter{iloop}{0}%
\loop\stepcounter{iloop}\ifnum\value{iloop}<\the\numexpr\imax+1
\bigstar\repeat
\openbigstar[\xrest]%
\setcounter{iloop}{0}%
\loop\stepcounter{iloop}\ifnum\value{iloop}<\the\numexpr5-\imax\relax
\openbigstar[.9]\repeat}}


\def\N{\mathbb N}
\def\Z{\mathbb Z}
\def\Q{\mathbb Q}
\def\R{\mathbb R}
\def\Rpe{\mathbb R_+^*}
\def\C{\mathbb C}
\def\K{\mathbb K}
\def\L{\mathbb L}

\def\P{\mathscr{P}}

\def\Ker{\text{Ker}}
\def\ord{\text{ord}}

\def\ssi{\Leftrightarrow}
\def\Ssi{\Longleftrightarrow}
\def\implique{\Longrightarrow}

\def\sep{\noindent\makebox[\linewidth]{\rule{\paperwidth}{0.4pt}}}

% CUSTOM TITLE
\makeatletter
\def\@maketitle{%
	\newpage
	%  \null% DELETED
	%  \vskip 2em% DELETED
	\begin{center}%
		\let \footnote \thanks
		{\LARGE\bfseries \@title \par}%
		\vskip 1em%
		{\large
			\lineskip .5em%
			\begin{tabular}[t]{c}%
				\@author
			\end{tabular}\par}%
		\vskip 1em%
		{\large \@date}%
	\end{center}%
	\par
	\vskip -1em}
\makeatother

\title{Semaine 8\\ Exercices}
\author{Amar AHMANE}

\begin{document}
	\maketitle
	\begin{flushleft}
		\textit{– Eh bien... dit Chick, je lui ai demandé si elle aimait Jean-Sol Partre et elle m'a dit qu'elle faisait collection de ses oeuvres... Alors, je lui ai dit : Moi aussi... Et, chaque fois que je lui disais quelque chose, elle répondait : Moi aussi, et vice-versa... Alors à la fin, juste pour faire une expérience existentialiste, je lui ai dit : Je vous aime beaucoup, et elle a dit : Oh !}\\
		\textit{– L'expérience avait raté, dit Colin.}\\
		\textit{– Oui, dit Chick. Mais elle n'est pas partie tout de même. Alors, j'ai dit : Je vais par là, et elle a dit : Pas moi... et elle a ajouté : Moi je vais par là.}\\
		\textit{– Alors j'ai dit : Moi aussi, dit Chick. Et j'ai été partout où elle a été}
	\end{flushleft}
	\begin{flushright}
		\textit{L'Écume des jours}, Boris Vian
	\end{flushright}
	\subsection*{Exercice : c'est le moment de réfléchir. (source : Mathraining) (et franchement c'est du niveau Terminale il faut qu'au moins quelqu'un le fasse ça)}
	Au jeu Euro Millions, il est demandé de donner $5$ "numéros" entre $1$ et $50$ (et pas deux fois le même) et $2$ "étoiles" entre $1$ et $11$ (et pas deux fois la même non plus). Le tirage consiste alors en $5$ numéros et $2$ étoiles tirés au hasard. Un gain est obtenu dès que l'on a $2$ bons numéros ou lorsqu'on a $1$ bon numéro et les $2$ bonnes étoiles. Dans tous les autres cas, on ne remporte rien. Quelle est la probabilité de ne rien gagner après avoir joué une grille ?
	\subsection*{Exercice : Calcul d'inverse. (source : Oraux X-ENS, Algèbre 1, Cassini)}
	Soit $A$ un anneau et $(a,b)\in A^2$. Montrer que si $1-ab$ est inversible alors $1-ba$ l'est aussi.
	\subsection*{Exercice : Des carrés particuliers. (source : École Navale)}
	Déterminer les matrices $M\in\mathcal M_3(\R)$ telles que \[M^2=\begin{pmatrix}0&0&1\\0&0&0\\0&0&0 \end{pmatrix}\]
	\subsection*{Exercice : Questions de divisibilité. (source : Oraux X-ENS, Algèbre 1, Cassini)}
	\begin{enumerate}
		\item Si $n$ et $k$ sont dans $\N$, montrer que $(n!)^k|(nk)!$.
		\item Soit $n\geq 1$. Montrer que si $(m_1,\dots,m_n)\in(\N)^n$, alors il possible de trouver $K\in\mathcal P(\lbrace m_1,\dots,m_n\rbrace)$ tel que $\sum_{k\in K}k$ divise $n$.
		\item Déterminer les eniters $n\in\N^*$ tels que $7$ divise $n^n-3$.
	\end{enumerate}
\end{document}