% Preamble
\documentclass[10pt]{article}

% Math related packages.
\usepackage{amsfonts, amsthm, amsmath, amssymb, amstext}
\usepackage{mathtools}
\usepackage{physics}
\usepackage{cancel, textcomp}
\usepackage[mathscr]{euscript}
\usepackage[nointegrals]{wasysym}
\usepackage[left=3cm, right=3cm, top=1.5cm, bottom=1.5cm]{geometry}

\usepackage{esvect}
\usepackage{IEEEtrantools}

% Font related packages.
\usepackage[french]{babel}
\usepackage[unicode]{hyperref}
\usepackage[fontsize = 10pt]{scrextend}
\usepackage[T1]{fontenc}
\usepackage[utf8]{inputenc}
\usepackage[finemath]{kotex}
\usepackage{dhucs-nanumfont}
\usepackage{mathpazo}
\usepackage{FiraMono}
\usepackage{mathrsfs}

% Other
\usepackage{enumerate}
\usepackage[shortlabels]{enumitem}
\usepackage{tabularx}
\usepackage[object=vectorian]{pgfornament}
\usepackage{pgf}
\usepackage{pgfpages}
\usepackage[european,straightvoltages]{circuitikz}
\usepackage{scalerel}
\usepackage{stackengine}


% A more colorful world.
\usepackage{xcolor}

% Display Source Code
\usepackage{listings}
\lstset{
    language=C,
    basicstyle=\small\ttfamily\mdseries,
    numberstyle=\color{gray},
    stringstyle=\color[HTML]{933797},
    commentstyle=\color[HTML]{228B22},
    emph={[2]from,with,import,as,pass,return,and,or,not},
    emphstyle={[2]\color[HTML]{DD52F0}},
    emph={[3]range,format,enumerate,print},
    emphstyle={[3]\color[HTML]{D17032}},
    emph={[4]if,elif,else,for,while,in,def,lambda,int,float,all,len},
    emphstyle={[4]\color{blue}},
    emph={[5]abs},
    emphstyle={[5]\color{black}},
    showstringspaces=false,
    breaklines=true,
    prebreak=\mbox{{\color{gray}\tiny$\searrow$}},
    numbers=left,
    xleftmargin=15pt
}

% OPTIONS
\everymath\expandafter{\the\everymath\displaystyle}

% COMMANDS
\newcommand{\f}[1]{\texttt{#1}}
\newcommand{\sct}[1]{
	\begin{center}
		\Large\textbf{#1}
	\end{center}
}
\newcommand{\subsct}[1]{
	\begin{center}
		\large\textbf{#1}
	\end{center}
}
\newcommand{\inl}[2]{[\![#1, #2]\!]}
\newcommand{\q}[1]{\textbf{#1.}\quad}
\newcommand{\urlsymbol}{\kern1pt\vbox to .5ex{}\raise.10ex\hbox{\pdfliteral{%
    q .8 0 0 .8 0 0 cm
    2.5 5 m 1 j 1 J .8 w
    1 5 l 0 5 0 4 y 0 1 l 0 0 1 0 y 4 0 l 5 0 5 1 y 5 2.5 l S
    3 3 m 6 6 l S 4 6 m 6 6 l 6 4 l S
Q}}\kern5pt}

\newcounter{iloop}
\newcommand\openbigstar[1][0.7]{%
  \scalerel*{%
    \stackinset{c}{-.125pt}{c}{}{\scalebox{#1}{\color{white}{$\bigstar$}}}{%
      $\bigstar$}%
  }{\bigstar}
}
\newcommand{\Stars}[1]{\ensuremath{%
\pgfmathtruncatemacro{\imax}{ifthenelse(int(#1)==#1,#1-1,#1)}%
\pgfmathsetmacro{\xrest}{0.9*(1-#1+\imax)}%
\setcounter{iloop}{0}%
\loop\stepcounter{iloop}\ifnum\value{iloop}<\the\numexpr\imax+1
\bigstar\repeat
\openbigstar[\xrest]%
\setcounter{iloop}{0}%
\loop\stepcounter{iloop}\ifnum\value{iloop}<\the\numexpr5-\imax\relax
\openbigstar[.9]\repeat}}


\def\N{\mathbb N}
\def\Z{\mathbb Z}
\def\Q{\mathbb Q}
\def\R{\mathbb R}
\def\Rpe{\mathbb R_+^*}
\def\C{\mathbb C}
\def\K{\mathbb K}
\def\L{\mathbb L}

\def\prts{\mathcal{P}}
\def\P{\mathscr{P}}

\def\vide{\emptyset}

\def\ssi{\Leftrightarrow}
\def\Ssi{\Longleftrightarrow}
\def\implique{\Longrightarrow}

\def\sep{\noindent\makebox[\linewidth]{\rule{\paperwidth}{0.4pt}}}

% CUSTOM TITLE
\makeatletter
\def\@maketitle{%
	\newpage
	%  \null% DELETED
	%  \vskip 2em% DELETED
	\begin{center}%
		\let \footnote \thanks
		{\LARGE \@title \par}%
		\vskip 1em%
		{\large
			\lineskip .5em%
			\begin{tabular}[t]{c}%
				\@author
			\end{tabular}\par}%
		\vskip 1em%
		{\large \@date}%
	\end{center}%
	\par
	\vskip -1em}
\makeatother

\title{Semaine 2}

\begin{document}
	\maketitle
	\subsubsection*{\textsc{Pierre-Gabriel Berlureau} \Stars{4}}
	\begin{enumerate}
		\item Le premier exemple est simple, on se concentre sur le cas $\mathcal{P}(X)$. D'abord, il est évident que $\vide\in\prts(X)$ et $X\in\prts(X)$. L'union d'une familles de parties de $X$ est évidemment aussi une partie de $X$, de même pour l'intéresection, ainsi $\prts(X)$ est une topologie sur $X$.
		\item Itérativement.
		\item
		\begin{itemize}
			\item $\emptyset$ est évidemment un ouvert de $\R$, puisque la phrase $\forall x\in\vide\dots$ est toujours vraie.
			\item $\R$ est un ouvert de $\R$, on pourra prendre $\epsilon=1$ pour le montrer.
			\item $[a,b]$ n'est pas un ouvert de $\R$. En effet, $a\in[a,b]$, or, pour tout $\epsilon\in\R_+^*$, $a-\epsilon<a$, d'où $a-\epsilon<a-\frac\epsilon2<a$, d'où l'existence d'un élément $x\in]a-\epsilon,a+\epsilon[$ tel que $x\notin[a,b]$.
			\item $]a,b[$, oui. Soit $x\in]a,b[$, on pose $\epsilon_1=x-a$ et $\epsilon_2=b-x$ et $\epsilon=\frac{\min(\epsilon_1,\epsilon_2)}2$, ainsi $x-\epsilon>a$ et $x+\epsilon<b$ donc $]x-\epsilon,x+\epsilon[\subset]a,b[$.
		\end{itemize}
		\item On pose $\mathcal{O}=\lbrace U\in\prts(\R)|\forall x\in U,\ \exists\epsilon\in\Rpe,\ ]x-\epsilon,x+\epsilon[\subset U\rbrace$. Ainsi, on a que $\mathcal{O}$ est un sous-ensemble de $\prts(\R)$, et $\vide,\R\in\mathcal{O}$. Soit $I$ un ensemble, $(U_i)_{i\in I}$ une famille d'éléments de $\mathcal{O}$. Soit $x\in\bigcup_{i\in I}U_i$, alors $\exists i\in I,\ x\in U_i$, il existe donc $\epsilon\in\Rpe$ tel que $]x-\epsilon,x+\epsilon[\subset U_i$ donc $]x-\epsilon,x+\epsilon[\subset \bigcup_{i\in I}U_i$, ainsi $\bigcup_{i\in I}U_i$ est un ouvert de $\R$. La preuve pour l'interesction est de même nature, il faudra juste prendre le minimum des deux $\epsilon$ que l'on a durant la preuve. Ainsi, on a montré que $\mathcal{O}$ est une topologie sur $\R$. 
		\item Vérifications faciles. L'union ne donne pas toujours des topologies sur $X$, considérer $\mathcal{O}$ et $\lbrace \vide,[a,b],\R$.
		\item Considérer l'intersection des topologies contenant $\mathcal{A}$.
	\end{enumerate}
	\subsubsection*{\textsc{Matteo Delfour} \Stars{2}}
		\paragraph{Analyse} Soit $f\in\R^\R$ telle que \[\forall x,y\in\R,\quad f(x-f(y))=2-x-y\]
		Soit $x\in\R$, alors on a $f(-f(x))=2-x$, donc ce résultat est vrai pour tout $x\in\R$. Soit $x\in\R$, on a alors $f(x-f(2))=-x$ donc $-f(x-f(2))=x$ donc $f(-f(x-f(2)))=f(x)$ donc $2-(x-f(2))=f(x)$ donc $f(x)=2-x+f(2)$. On en déduit que \[\exists\lambda\in\R,\ \forall x\in\R,\ f(x)=2-x+\lambda\]
		Le reste du raisonnement est simple.
		\paragraph{Synthèse} Facile.
	\subsubsection*{\textsc{Yanis Grigy} \Stars{3}}
		Supposons que $\mathbb{P}$ est fini, on note $p_1,\dots,p_n$ les $n\in\N^*$ nombres premiers et on pose $N=p_1\dots p_n +1$. Ainsi, $N$ admet un diviseur premier, que l'on note $p_k$ qui divise donc $N-p_1\dots p_n$, donc $p_k$ divise $1$, ce qui est absurde.
	\subsubsection*{\textsc{Louis Marchal} \Stars{5}}
		On le montre par l'absurde. 
		Soit $n\in\N^*$, on suppose $\left(\frac{3+4i}{5}\right)^n=1$, donc $(3+4i)^n=5^n$, donc \[\sum_{k=0}^n\binom{n}{k}3^{n-k}(4i)^k=5^n\] donc \[\sum_{1\leq 2k+1\leq n}(-1)^k\binom{n}{2k+1}3^{n-(2k+1)}4^{2k+1}=0\] donc \[\sum_{3\leq 2k+1\leq n}(-1)^k\binom{n}{2k+1}3^{n-(2k+1)}4^{2k}+n=0\] donc $4$ divise $n$ (on vient juste d'éliminer tous les eniters non congrus à $0$ mod $4$, reste à chercher une absurdité avec ceux-là).\\

		Soit à présent $m\in\N^*$, on a 
		\begin{align*}
			(3+4i)^{4m} &=\sum_{k=0}^4m\binom{4m}k3^{4m-k}(4i)^k\\
						&=\sum_{k=0}^{2m}(-1)^k\binom{4m}{2k}3^{4m-2k}4^{2k}+i\sum_{k=0}^{2m-1}(-1)^k\binom{4m}{2k+1}3^{4m-(2k+1)}4^{2k+1}
		\end{align*}
		Mais aussi $z:=(3+4i)^{4m}=5^{4m}$.\\
		 On a \[\Re(z)=\sum_{k=0}^{2m}(-1)^k\binom{4m}{2k}3^{4m-2k}4^{2k}\equiv \sum_{k=0}^{2m}(-1)^k\binom{4m}{2k}(-1)^{2m-k}\equiv\sum_{k=0}^{2m}\binom{4m}{2k}[5]\]
		et 
		\begin{align*}
			\Im(z)=\sum_{k=0}^{2m-1}(-1)^k\binom{4m}{2k+1}3^{4m-(2k+1)}4^{2k+1}&=\sum_{k=0}^{2m-1}(-1)^k\binom{4m}{2k+1}3^{4m-(2k)}4^{2k}\\&\equiv\sum_{k=0}^{2m-1}(-1)^k\binom{4m}{2k+1}(-1)^{2m-k}\\&\equiv \sum_{k=0}^{2m-1}\binom{4m}{2k+1}[5]
		\end{align*}
		Donc $\Re(z)+\Im(z)\equiv\sum_{k=0}^{4m}\binom{4m}{k}[5]$, mais $\Re(z)+\Im(z)=5^{4m}$, donc $5^{4m}\equiv 2^{4m}[5]$, donc $2^{4m}\equiv0[5]$, ce qui est absurde.
	\subsubsection*{\textsc{Shems} \Stars{3}}
		Soit $m\in\N$,
		\begin{align*}
			\int_0^\pi \sin^{2m}x\cos(2mx)\dd x &=\Re\left(\int_0^\pi \sin^{2m}xe^{i2mx}\dd x\right)\\
												&=\Re\left(\int_0^\pi \left(\sin(x)e^{ix}\right)^{2m}\dd x\right)\\
												&=\frac{(-1)^m}{4^m}\Re\left(\int_0^\pi \left(2i\sin(x)e^{ix}\right)^{2m}\dd x\right)\\
												&=\frac{(-1)^m\pi}{4^m}
		\end{align*}
		La relation de récurrence pour la seconde est, pour $m>2$ : \[I_m=\frac{m-1}{m-2}I_{m-2}\]

\end{document}