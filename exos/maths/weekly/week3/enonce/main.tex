% Preamble
\documentclass[10pt]{article}

% Math related packages.
\usepackage{amsfonts, amsthm, amsmath, amssymb, amstext}
\usepackage{mathtools}
\usepackage{physics}
\usepackage{cancel, textcomp}
\usepackage[mathscr]{euscript}
\usepackage[nointegrals]{wasysym}
\usepackage[left=3cm, right=3cm, top=1.5cm, bottom=1.5cm]{geometry}

\usepackage{esvect}
\usepackage{IEEEtrantools}

% Font related packages.
\usepackage[french]{babel}
\usepackage[unicode]{hyperref}
\usepackage[fontsize = 10pt]{scrextend}
\usepackage[T1]{fontenc}
\usepackage[utf8]{inputenc}
\usepackage[finemath]{kotex}
\usepackage{dhucs-nanumfont}
\usepackage{mathpazo}
\usepackage{FiraMono}
\usepackage{mathrsfs}

% Other
\usepackage{enumerate}
\usepackage[shortlabels]{enumitem}
\usepackage{tabularx}
\usepackage[object=vectorian]{pgfornament}
\usepackage{pgf}
\usepackage{pgfpages}
\usepackage[european,straightvoltages]{circuitikz}
\usepackage{scalerel}
\usepackage{stackengine}


% A more colorful world.
\usepackage{xcolor}

% Display Source Code
\usepackage{listings}
\lstset{
    language=C,
    basicstyle=\small\ttfamily\mdseries,
    numberstyle=\color{gray},
    stringstyle=\color[HTML]{933797},
    commentstyle=\color[HTML]{228B22},
    emph={[2]from,with,import,as,pass,return,and,or,not},
    emphstyle={[2]\color[HTML]{DD52F0}},
    emph={[3]range,format,enumerate,print},
    emphstyle={[3]\color[HTML]{D17032}},
    emph={[4]if,elif,else,for,while,in,def,lambda,int,float,all,len},
    emphstyle={[4]\color{blue}},
    emph={[5]abs},
    emphstyle={[5]\color{black}},
    showstringspaces=false,
    breaklines=true,
    prebreak=\mbox{{\color{gray}\tiny$\searrow$}},
    numbers=left,
    xleftmargin=15pt
}

% OPTIONS
\everymath\expandafter{\the\everymath\displaystyle}

% COMMANDS
\newcommand{\f}[1]{\texttt{#1}}
\newcommand{\sct}[1]{
	\begin{center}
		\Large\textbf{#1}
	\end{center}
}
\newcommand{\subsct}[1]{
	\begin{center}
		\large\textbf{#1}
	\end{center}
}
\newcommand{\inl}[2]{[\![#1, #2]\!]}
\newcommand{\q}[1]{\textbf{#1.}\quad}
\newcommand{\urlsymbol}{\kern1pt\vbox to .5ex{}\raise.10ex\hbox{\pdfliteral{%
    q .8 0 0 .8 0 0 cm
    2.5 5 m 1 j 1 J .8 w
    1 5 l 0 5 0 4 y 0 1 l 0 0 1 0 y 4 0 l 5 0 5 1 y 5 2.5 l S
    3 3 m 6 6 l S 4 6 m 6 6 l 6 4 l S
Q}}\kern5pt}

\newcounter{iloop}
\newcommand\openbigstar[1][0.7]{%
  \scalerel*{%
    \stackinset{c}{-.125pt}{c}{}{\scalebox{#1}{\color{white}{$\bigstar$}}}{%
      $\bigstar$}%
  }{\bigstar}
}
\newcommand{\Stars}[1]{\ensuremath{%
\pgfmathtruncatemacro{\imax}{ifthenelse(int(#1)==#1,#1-1,#1)}%
\pgfmathsetmacro{\xrest}{0.9*(1-#1+\imax)}%
\setcounter{iloop}{0}%
\loop\stepcounter{iloop}\ifnum\value{iloop}<\the\numexpr\imax+1
\bigstar\repeat
\openbigstar[\xrest]%
\setcounter{iloop}{0}%
\loop\stepcounter{iloop}\ifnum\value{iloop}<\the\numexpr5-\imax\relax
\openbigstar[.9]\repeat}}


\def\N{\mathbb N}
\def\Z{\mathbb Z}
\def\Q{\mathbb Q}
\def\R{\mathbb R}
\def\Rpe{\mathbb R_+^*}
\def\C{\mathbb C}
\def\K{\mathbb K}
\def\L{\mathbb L}

\def\P{\mathscr{P}}

\def\ssi{\Leftrightarrow}
\def\Ssi{\Longleftrightarrow}
\def\implique{\Longrightarrow}

\def\sep{\noindent\makebox[\linewidth]{\rule{\paperwidth}{0.4pt}}}

% CUSTOM TITLE
\makeatletter
\def\@maketitle{%
	\newpage
	%  \null% DELETED
	%  \vskip 2em% DELETED
	\begin{center}%
		\let \footnote \thanks
		{\LARGE \@title \par}%
		\vskip 1em%
		{\large
			\lineskip .5em%
			\begin{tabular}[t]{c}%
				\@author
			\end{tabular}\par}%
		\vskip 1em%
		{\large \@date}%
	\end{center}%
	\par
	\vskip -1em}
\makeatother

\title{Semaine 2}

\begin{document}
	\maketitle
	\subsubsection*{\textsc{Pierre-Gabriel Berlureau} \Stars{4}}
	Lorsque $X$ est un ensemble, si $\mathcal T$ est un sous-ensemble de $\mathcal P(X)$, on dit que $\mathcal T$ est une topologie sur $X$ si et seulement si 
	\begin{itemize}
		\item $\emptyset\in\mathcal T$ et $X\in\mathcal T$;
		\item Pour tout ensemble $I$, pour toute famille $(U_i)_{i\in I}$ d'éléments de $\mathcal T$, $\bigcup_{i\in I}U_i\in\mathcal T$;
		\item Pour tout $U,V\in\mathcal T$, $U\cap V\in\mathcal T$.
	\end{itemize}
	\begin{enumerate}
		\item Soit $X$ un ensemble. Montrer que $\lbrace\emptyset,X\rbrace$ et $\mathcal P(X)$ sont des topologies sur X.
		\item On suppose que $\mathcal T$ est une topologie sur $X$.\\
		Soit $n\in\N^*$ et $U_1,\dots,U_n$ $n$ éléments de $\mathcal T$. Montrer que $U_1\cap\dots\cap U_n\in\mathcal T$.
		\item Si $U$ est une partie de $\R$, on dit que $U$ est un ouvert de $R$ si et seulement si \[\forall x\in u,\ \exists \varepsilon\in\R_+^*,\ ]x-\varepsilon,x+\varepsilon[\subset U\]
		\item Les ensembles suivants sont-ils des ouverts de $\R$ ?
		\begin{itemize}
			\item $\emptyset$
			\item $\R$
			\item $[a,b]$, où $a,b\in\R$ avec $a<b$
			\item $]a,b[$, où $a,b\in\R$ avec $a<b$
		\end{itemize}
		\item Montrer que l'ensemble des ouverts de $\R$ est une topologie sur $\R$.
		\item Soit $X$ et $I$ deux ensembles avec $I\neq \emptyset$. Soit $(\mathcal T_i)_{i\in  I}$ une famille de topolgies sur $X$. Montrer que $\bigcap_{i\in I}\mathcal T_i$ est une topologie sur $X$. Qu'en est-il de $\bigcup_{i\in I}\mathcal T_i$ ?
		\item Soit $X$ un ensemble avec $\mathcal A$ un sous-ensemble de $\mathcal P(X)$.\\
		Montrer que l'on peut définir la plus petite topologie sur $X$ contenant $\mathcal A$.
	\end{enumerate}
	\subsubsection*{\textsc{Matteo Delfour} \Stars{2}}
		Déterminer toutes les fonctions de $R$ dans $R$ telles que \[\forall x,y\in\R,\quad f(x-f(y))=2-x-y\]
	\subsubsection*{\textsc{Yanis Grigy} \Stars{3}}
		Montrer que l'ensemble des nombres premier est infini.
	\subsubsection*{\textsc{Louis Marchal} \Stars{5}}
		Montrer que pour tout $n\in\N^*$, le nombre $\frac{3+4i}{5}$ n'est pas une racine $n^{\text{ème}}$ de $1$.
	\subsubsection*{\textsc{Shems} \Stars{3}}
		Calculer, pour tout $m\in\N$, \(\int_0^\frac\pi4\sin^{2m}t\cos(2mt)\dd t\) et $\int_0^\frac\pi4\cos^m(t)\dd t$.

\end{document}