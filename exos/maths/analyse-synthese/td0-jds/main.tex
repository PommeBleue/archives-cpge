%! suppress = Unicode
\documentclass[10pt]{article}

% Math related packages.
\usepackage{amsfonts, amsthm, amsmath, amssymb, amstext}
\usepackage{mathtools}
\usepackage{physics}
\usepackage{cancel, textcomp}
\usepackage[mathscr]{euscript}
\usepackage[nointegrals]{wasysym}
\usepackage[a4paper, left=1.5cm, right=1.7cm, top=2.3cm, bottom=2cm]{geometry}

\usepackage{esvect}
\usepackage{IEEEtrantools}

% Font related packages.
\usepackage[french]{babel}
\usepackage[unicode]{hyperref}
\usepackage[fontsize = 10pt]{scrextend}
\usepackage[T1]{fontenc}
\usepackage[utf8]{inputenc}
\usepackage[finemath]{kotex}
\usepackage{dhucs-nanumfont}
\usepackage{mathpazo}
\usepackage{FiraMono}
\usepackage{mathrsfs}

% A more colorful world.
\usepackage{xcolor}

% Display Source Code
\usepackage{listings}
\lstset{
    language=Python,
    basicstyle=\small\ttfamily\mdseries,
    numberstyle=\color{gray},
    stringstyle=\color[HTML]{933797},
    commentstyle=\color[HTML]{228B22},
    emph={[2]from,with,import,as,pass,return,and,or,not},
    emphstyle={[2]\color[HTML]{DD52F0}},
    emph={[3]range,format,enumerate,print},
    emphstyle={[3]\color[HTML]{D17032}},
    emph={[4]if,elif,else,for,while,in,def,lambda,int,float,all,len},
    emphstyle={[4]\color{blue}},
    emph={[5]abs},
    emphstyle={[5]\color{black}},
    showstringspaces=false,
    breaklines=true,
    prebreak=\mbox{{\color{gray}\tiny$\searrow$}},
    numbers=left,
    xleftmargin=15pt
}

% OPTIONS
\everymath{\displaystyle}

% COMMANDS
\newcommand{\f}[1]{\texttt{#1}}
\newcommand{\chpt}[1]{\newpage\begin{flushright}\huge\textbf{#1}\end{flushright}}
\newcommand{\urlsymbol}{\kern1pt\vbox to .5ex{}\raise.10ex\hbox{\pdfliteral{%
    q .8 0 0 .8 0 0 cm
    2.5 5 m 1 j 1 J .8 w
    1 5 l 0 5 0 4 y 0 1 l 0 0 1 0 y 4 0 l 5 0 5 1 y 5 2.5 l S
    3 3 m 6 6 l S 4 6 m 6 6 l 6 4 l S
Q}}\kern5pt}

\def\N{\mathbb N}
\def\Z{\mathbb Z}
\def\Q{\mathbb Q}
\def\R{\mathbb R}
\def\C{\mathbb C}
\def\K{\mathbb K}

\title{\huge\textbf{Analyse-Synthèse}}
\author{Amar AHMANE\\ MP2I}
\date{}

\begin{document}
    \maketitle
    \subsubsection*{10}
    \paragraph{Énoncé} Soit $f:[0,1]\to \R$ une fonction continue.
    \begin{enumerate}
        \item Montrer que $f$ s'écrit, de manière unique, sous la forme $f=a+g$, où \[a\in\R\quad\text{et}\quad\int_0^1g(t)\dd t=0\]
        \item Montrer que $f$ s'écrit, de manière unique, sous la forme $f=P+g$, où \[P\text{ est affine}\quad\text{et}\quad\int_0^1g(t)\dd t=\int_0^1tg(t)\dd t=0\]
    \end{enumerate}

    \paragraph{Solution proposée} Soit $f:[0,1]\to \R$ une fonction continue.
    \begin{enumerate}
        \item On raisonne par analyse synthèse.


        \paragraph{Analyse} Supposons qu'il existe un réel $a$ et une fonction $g$ vérifiant \[\int_0^1g(t)\dd t=0\] tels que $f=a+g$. 


        La continuité de $f$ nous permet d'écrire que
        \begin{align*}
            \int_0^1f(t)\dd t &= \int_0^1(a+g(t))\dd t\\
                              &= \int_0^1a\dd t+\int_0^1g(t)\dd t\quad \text{par linéarité de l'intégrale}\\
                              &= \int_0^1a\dd t\\
                              &= \Big[at\Big]_0^1\\
                              &= a
        \end{align*}
        Ceci ayant été établi, on a également que $\forall x\in[0,1],\quad g(x)=f(x)-\int_0^1f(t)\dd t$.


        \paragraph{Synthèse} Posons $a=\int_0^1f(t)\dd t$ et $g=f-a$, vérifions alors que le couple $(a,g)$ convient.


        Soit $x\in[0, 1]$, \[g(x)+a=f(x)-a+a=f(x)\]
        Ensuite
        \begin{align*}
            \int_0^1g(t)\dd t &= \int_0^1(f(t)-a)\dd t\\
                              &= \int_0^1f(t)\dd t-\int_0^1a\dd t\\
                              &= a - \Big[at\Big]_0^1\\
                              &= 0
        \end{align*}
        Et, comme $f$ est à valeurs dans $\R$, on a que $a\in\R$. Ce qui conclut notre raisonnement.

        \item On raisonne par analyse synthèse.

        \paragraph{Analyse} Supposons qu'il existe une fonction $P$ affine et une fonction $g$ vérifiant \[\int_0^1g(t)\dd t = \int_0^1tg(t)\dd t\] telles que $f=P+g$. Il existe alors $(a,b)\in\R^2$ tel que $\forall x\in[0,1],\quad P(x)=ax+b$.


        En ce cas, la continuité de $f$ et des fonctions affines (et donc de $P$) nous permet d'écrire que

        \begin{align*}
            \int_0^1f(t)\dd t &= \int_0^1(P(t)+g(t))\dd t\\
                              &= \int_0^1P(t)\dd t +\int_0^1g(t)\dd t\quad \text{par linéarité de l'intégrale}\\
                              &= \int_0^1(at+b)\dd t\\
                              &=\Big[\frac{at^2}2+bt\Big]_0^1\\
                              &=\frac a2+b
        \end{align*}
        Donc \[\frac a2+b=\int_0^1f(t)\dd t \tag{1}\]
        Or, \[\forall x\in[0,1], xf(x)=xP(x)+xg(x)\]
        Donc, comme le produit de fonctions continues est une fonction continue, on a que

        \begin{align*}
            \int_0^1tf(t)\dd t &= \int_0^1(tP(t)+tg(t))\dd t\\
                              &= \int_0^1tP(t)\dd t +\int_0^1tg(t)\dd t\quad \text{par linéarité de l'intégrale}\\
                              &= \int_0^1(at^2+bt)\dd t\\
                              &=\Big[\frac{at^3}3+\frac{bt^2}2\Big]_0^1\\
                              &=\frac a3+\frac b2
        \end{align*}
        Donc \[\frac a3+\frac b2=\int_0^1tf(t)\dd t\tag{2}\]
        De (1) on a que \[b=\int_0^1f(t)\dd t - \frac a2\]
        En remplaçant dans (2) on obtient \[\frac {2a}3=2\int_0^1tf(t)\dd t - \left(\int_0^1f(t)\dd t - \frac a2\right)\]
        Donc 
        \begin{alignat*}{3}
            &\frac {2a}3-\frac a2 & &= 2\int_0^1tf(t)\dd t - \int_0^1f(t)\dd t\\
            \Leftrightarrow & \qquad\ \ \frac a6  & &= \int_0^1f(t)(2t-1)\dd t\\
            \Leftrightarrow &\qquad\ \ \ a  & &= 6\int_0^1f(t)(2t-1)\dd t
        \end{alignat*}

        De même, \[b=2\int_0^1f(t)(2-3t)\dd t\]

        Et, enfin, \[g=f-P\]
    \end{enumerate}
    \paragraph{Synthèse} Posons $a=6\int_0^1f(t)(2t-1)\dd t$, $b=2\int_0^1f(t)(2-3t)\dd t$, $P:x\mapsto ax+b$ et $g=f-P$. On vérifie aisément que $f=P+g$; $P$ est affine et
    \begin{align*}
        \int_0^1g(t)\dd t &= \int_0^1(f(t)-P(t))\dd t\\
                          &= \int_0^1f(t)\dd t - \int_0^1P(t)\dd t\\
                          &= \int_0^1f(t)\dd t - \int_0^1(at+b)\dd t\\
                          &= \int_0^1f(t)\dd t - \Big[\frac {at^2}2+bt\Big]_0^1\\
                          &= \int_0^1f(t)\dd t - \left(\frac a2 +b\right)
    \end{align*}
    Or \[\frac a2 +b=3\int_0^1f(t)(2t-1)\dd+2\int_0^1f(t)(2-3t)\dd t=\int_0^1f(t)(6t-3+4-6t)\dd t=\int_0^1f(t)\dd t\]
    Donc \[\int_0^1g(t)\dd t=0\]
    De même 
    \begin{align*}
        \int_0^1tg(t)\dd t &= \int_0^1(tf(t)-tP(t))\dd t\\
                          &= \int_0^1tf(t)\dd t - \int_0^1tP(t)\dd t\\
                          &= \int_0^1tf(t)\dd t - \int_0^1(at^2+bt)\dd t\\
                          &= \int_0^1tf(t)\dd t - \Big[\frac {at^3}3+\frac{bt^2}2\Big]_0^1\\
                          &= \int_0^1tf(t)\dd t - \left(\frac a3 +\frac b2\right)
    \end{align*}
    Or \[\frac a3 +\frac b2=2\int_0^1f(t)(2t-1)\dd+\int_0^1f(t)(2-3t)\dd t=\int_0^1f(t)(4t-2+2-3t)\dd t=\int_0^1tf(t)\dd t\]
    Donc \[\int_0^1tg(t)\dd t=0\]
    Le couple $(P,g)$ convient. Ceci conclut notre raisonnement.
\end{document}