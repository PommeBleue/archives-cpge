% Preamble
\documentclass[7pt, twocolumn]{extarticle}

% Math related packages.
\usepackage{amsfonts, amsthm, amsmath, amssymb, amstext}
\usepackage{mathtools}
\usepackage{physics}
\usepackage{cancel, textcomp}
\usepackage[mathscr]{euscript}
\usepackage[nointegrals]{wasysym}
\usepackage[left=2mm, right=2mm, top=1.5cm, bottom=1.5cm]{geometry}

\usepackage{esvect}
\usepackage{IEEEtrantools}

% Font related packages.
\usepackage[french]{babel}
\usepackage[unicode]{hyperref}
\usepackage[fontsize = 10pt]{scrextend}
\usepackage[T1]{fontenc}
\usepackage[utf8]{inputenc}
\usepackage[finemath]{kotex}
\usepackage{dhucs-nanumfont}
\usepackage{mathpazo}
\usepackage{FiraMono}
\usepackage{mathrsfs}

% Other
\usepackage{enumerate}
\usepackage[shortlabels]{enumitem}
\usepackage{tabularx}
\usepackage[object=vectorian]{pgfornament}
\usepackage{pgf}
\usepackage{pgfpages}

% Columns
\usepackage{multicolrule}
\setlength\columnsep{1cm}

\SetMCRule{width=0.4pt, line-style=solid-circles,
  width=1pt, extend-fill, extend-top=1ex}

% A more colorful world.
\usepackage{xcolor}

% Display Source Code
\usepackage{listings}
\lstset{
    language=C,
    basicstyle=\small\ttfamily\mdseries,
    numberstyle=\color{gray},
    stringstyle=\color[HTML]{933797},
    commentstyle=\color[HTML]{228B22},
    emph={[2]from,with,import,as,pass,return,and,or,not},
    emphstyle={[2]\color[HTML]{DD52F0}},
    emph={[3]range,format,enumerate,print},
    emphstyle={[3]\color[HTML]{D17032}},
    emph={[4]if,elif,else,for,while,in,def,lambda,int,float,all,len},
    emphstyle={[4]\color{blue}},
    emph={[5]abs},
    emphstyle={[5]\color{black}},
    showstringspaces=false,
    breaklines=true,
    prebreak=\mbox{{\color{gray}\tiny$\searrow$}},
    numbers=left,
    xleftmargin=15pt
}

% OPTIONS
\everymath\expandafter{\the\everymath\displaystyle}

% COMMANDS
\newcommand{\f}[1]{\texttt{#1}}
\newcommand{\sct}[1]{
  \begin{center}
    \Large\textbf{#1}
  \end{center}
}
\newcommand{\subsct}[1]{
  \begin{center}
    \large\textbf{#1}
  \end{center}
}
\newcommand{\q}[1]{\textbf{#1.}\quad}
\newcommand{\urlsymbol}{\kern1pt\vbox to .5ex{}\raise.10ex\hbox{\pdfliteral{%
    q .8 0 0 .8 0 0 cm
    2.5 5 m 1 j 1 J .8 w
    1 5 l 0 5 0 4 y 0 1 l 0 0 1 0 y 4 0 l 5 0 5 1 y 5 2.5 l S
    3 3 m 6 6 l S 4 6 m 6 6 l 6 4 l S
Q}}\kern5pt}


\def\N{\mathbb N}
\def\Z{\mathbb Z}
\def\Q{\mathbb Q}
\def\R{\mathbb R}
\def\Rpe{\mathbb R_+^*}
\def\C{\mathbb C}
\def\K{\mathbb K}

\def\P{\mathscr{P}}

\def\ssi{\Leftrightarrow}
\def\Ssi{\Longleftrightarrow}
\def\implique{\Longrightarrow}

% CUSTOM TITLE
\makeatletter
\def\@maketitle{%
  \newpage
  %  \null% DELETED
  %  \vskip 2em% DELETED
  \begin{center}%
    \let \footnote \thanks
    {\LARGE \@title \par}%
    \vskip 1em%
    {\large
      \lineskip .5em%
      \begin{tabular}[t]{c}%
        \@author
      \end{tabular}\par}%
    \vskip 1em%
    {\large \@date}%
  \end{center}%
  \par
  \vskip -1em}
\makeatother

\title{\textbf{Calcul intégral}}
\date{}


\begin{document}
  \subsection*{Exercice 9.4}
  \begin{enumerate}
    \item Soit $a\in \R_+$. On a, en posant le changement de variable $t=-x$ \[\int_{-a}^af(x)\dd x=\int_a^{-a}f(-t)-\dd t=\int_a^{-a}f(t)\dd t=-\int_{-a}^a\dd t\]
    D'où \[\int_{-a}^af(x)\dd x = 0\]
    \item On pose $f:x\mapsto \ln\left(\frac{1+e^{\arctan(x)}}{1+e^{-\arctan(x)}}\right)$. On a 
    \begin{align*}
      \mathcal{D}_f &= \lbrace x\in\R | f(x) \text{ est défini} \rbrace\\
                    &= \left\lbrace x\in\R | \frac{1+e^{\arctan(x)}}{1+e^{-\arctan(x)}} > 0 \right\rbrace\\
                    &= \R
    \end{align*}
    On montre que $f$ est impaire. Soit $x\in\R$, on a
    \begin{align*}
      f(-x) &= \ln\left(\frac{1+e^{\arctan(x)}}{1+e^{-\arctan(x)}}\right) \\
            &= \ln\left(\frac{1+e^{\arctan(-x)}}{1+e^{-\arctan(-x)}}\right) \\ 
            &= \ln\left(\frac{1+e^{-\arctan(x)}}{1+e^{\arctan(x)}}\right) \\
            &= -\ln\left(\frac{1+e^{\arctan(x)}}{1+e^{-\arctan(x)}}\right) \\
            &= -f(x)
    \end{align*}
    $f$ étant continue, $\int_{-666}^{666}f(x)\dd x$ est définie et, d'après la question 1, on a $a=666$ donc $\int_{-a}^af(x)\dd x = 0$
  \end{enumerate}
  \subsection*{Exercice 9.7}
  \begin{enumerate}
    \item $\alpha = \ln(1+\sqrt{2})$.
    \item On pose le changement de variable $x=\sinh(t)$, d'où $\dd x=\cosh(t)\dd t$. On a alors 
    \begin{align*}
      J=\int_0^1\frac1{\sqrt{1+x^2}}\dd x &=\int_0^\alpha \frac1{\sqrt{1+\sinh^2(t)}}\cosh(t)\dd t\\
                                          &=\int_0^\alpha \frac1{\cosh(t)}\cosh(t)\dd t\\
                                          &=\int_0^\alpha \dd t\\
                                          &=\alpha
    \end{align*}
    \item On a 
    \begin{align*}
      I=\int_0^1\sqrt{1+x^2}\dd x &= [x\sqrt{1+x^2}]^1_0-\int_0^1x\frac{2x}{2\sqrt{1+x^2}}\\
                                  &= [x\sqrt{1+x^2}]^1_0-\int_0^1\frac{x^2}{\sqrt{1+x^2}}\\
                                  &= [x\sqrt{1+x^2}]^1_0-\int_0^1\frac{x^2+1-1}{\sqrt{1+x^2}}\\
                                  &= [x\sqrt{1+x^2}]^1_0-\int_0^1\frac{x^2+1-1}{\sqrt{1+x^2}}\\
                                  &= \sqrt2-(I-J)\\
                                  &= \sqrt2-I+J
    \end{align*}
    D'où $I=\frac{\sqrt2+J}2$.
    \item $I=\frac{\sqrt2+\alpha}2$.
  \end{enumerate}
  \subsection*{Exercice 9.8}
  En posant $t=x^3$, d'où $\dd t = 3x^2\dd x$
  \subsection*{Exercice 9.9}
  En posant $x=\frac\pi4 - u$, d'où $\dd x=-\dd u$. 
  \begin{align*}
    I &= \int_0^{\frac\pi4}\ln(1+\tan(x))\dd x\\
      &= \int_{\frac\pi4}^0-\ln(1+\tan(\frac\pi4-u))\dd u\\
      &= \int_0^{\frac\pi4}\ln(1+\frac{1-\tan(u)}{1+\tan(u)})\dd u\\
      &= \int_0^{\frac\pi4}\ln(\frac2{1+\tan(u)}) \dd u\\
      &= \int_0^{\frac\pi4} \ln2\dd u-I\\
      &= \ln2\frac\pi4 - I
  \end{align*}
  D'où $2I=\ln2\frac\pi4\Ssi I=\ln2\frac\pi8$.
\end{document}