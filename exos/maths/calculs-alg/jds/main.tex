% Preamble
\documentclass[7pt, twocolumn]{extarticle}

% Math related packages.
\usepackage{amsfonts, amsthm, amsmath, amssymb, amstext}
\usepackage{mathtools}
\usepackage{physics}
\usepackage{cancel, textcomp}
\usepackage[mathscr]{euscript}
\usepackage[nointegrals]{wasysym}
\usepackage[left=2mm, right=2mm, top=1.5cm, bottom=1.5cm]{geometry}

\usepackage{esvect}
\usepackage{IEEEtrantools}

% Font related packages.
\usepackage[french]{babel}
\usepackage[unicode]{hyperref}
\usepackage[fontsize = 10pt]{scrextend}
\usepackage[T1]{fontenc}
\usepackage[utf8]{inputenc}
\usepackage[finemath]{kotex}
\usepackage{dhucs-nanumfont}
\usepackage{mathpazo}
\usepackage{FiraMono}
\usepackage{mathrsfs}

% Other
\usepackage{enumerate}
\usepackage[shortlabels]{enumitem}
\usepackage{tabularx}
\usepackage[object=vectorian]{pgfornament}
\usepackage{pgf}
\usepackage{pgfpages}

% Columns
\usepackage{multicolrule}
\setlength\columnsep{1cm}

\SetMCRule{width=0.4pt, line-style=solid-circles,
	width=1pt, extend-fill, extend-top=1ex}

% A more colorful world.
\usepackage{xcolor}

% Display Source Code
\usepackage{listings}
\lstset{
    language=C,
    basicstyle=\small\ttfamily\mdseries,
    numberstyle=\color{gray},
    stringstyle=\color[HTML]{933797},
    commentstyle=\color[HTML]{228B22},
    emph={[2]from,with,import,as,pass,return,and,or,not},
    emphstyle={[2]\color[HTML]{DD52F0}},
    emph={[3]range,format,enumerate,print},
    emphstyle={[3]\color[HTML]{D17032}},
    emph={[4]if,elif,else,for,while,in,def,lambda,int,float,all,len},
    emphstyle={[4]\color{blue}},
    emph={[5]abs},
    emphstyle={[5]\color{black}},
    showstringspaces=false,
    breaklines=true,
    prebreak=\mbox{{\color{gray}\tiny$\searrow$}},
    numbers=left,
    xleftmargin=15pt
}

% OPTIONS
\everymath\expandafter{\the\everymath\displaystyle}

% COMMANDS
\newcommand{\f}[1]{\texttt{#1}}
\newcommand{\sct}[1]{
	\begin{center}
		\Large\textbf{#1}
	\end{center}
}
\newcommand{\subsct}[1]{
	\begin{center}
		\large\textbf{#1}
	\end{center}
}
\newcommand{\q}[1]{\textbf{#1.}\quad}
\newcommand{\urlsymbol}{\kern1pt\vbox to .5ex{}\raise.10ex\hbox{\pdfliteral{%
    q .8 0 0 .8 0 0 cm
    2.5 5 m 1 j 1 J .8 w
    1 5 l 0 5 0 4 y 0 1 l 0 0 1 0 y 4 0 l 5 0 5 1 y 5 2.5 l S
    3 3 m 6 6 l S 4 6 m 6 6 l 6 4 l S
Q}}\kern5pt}


\def\N{\mathbb N}
\def\Z{\mathbb Z}
\def\Q{\mathbb Q}
\def\R{\mathbb R}
\def\Rpe{\mathbb R_+^*}
\def\C{\mathbb C}
\def\K{\mathbb K}

\def\P{\mathscr{P}}

\def\ssi{\Leftrightarrow}
\def\Ssi{\Longleftrightarrow}
\def\implique{\Longrightarrow}

% CUSTOM TITLE
\makeatletter
\def\@maketitle{%
	\newpage
	%  \null% DELETED
	%  \vskip 2em% DELETED
	\begin{center}%
		\let \footnote \thanks
		{\LARGE \@title \par}%
		\vskip 1em%
		{\large
			\lineskip .5em%
			\begin{tabular}[t]{c}%
				\@author
			\end{tabular}\par}%
		\vskip 1em%
		{\large \@date}%
	\end{center}%
	\par
	\vskip -1em}
\makeatother

\title{\textbf{Calculs Algébriques}}
\date{}

% PGF BOXED
\pgfpagesdeclarelayout{boxed}
{
	\edef\pgfpageoptionborder{0pt}
}
{
	\pgfpagesphysicalpageoptions
	{%
		logical pages=1,%
	}
	\pgfpageslogicalpageoptions{1}
	{
		border code=\pgfsetlinewidth{0.5pt}\pgfstroke,%
		border shrink=\pgfpageoptionborder,%
		resized width=.95\pgfphysicalwidth,%
		resized height=.95\pgfphysicalheight,%
		center=\pgfpoint{.5\pgfphysicalwidth}{.5\pgfphysicalheight}%
	}%
}


% DOCUMENT
\pgfpagesuselayout{boxed}
\begin{document}
	\maketitle
	Dans tout ce qui va suivre, pour $n\in\N^*$, on notera $I_n=[\![1,n]\!]$.
	\sct{Sommes}
	\subsct{Sommes télescopiques}

	\vspace{2mm}

	\begin{enumerate}[start=1,label={\bfseries \arabic*}]
		\item Calculer la somme \(\sum_{k=1}^n\sqrt{1+\frac1{k^2}+\frac1{(k+1)^2}}\).
		\item La suite de Fibonacci est définie comme suit \[F_0=1;\ F_1=1;\quad \forall n\in\N,\ F_{n+2}=F_{n+1}+F_n\]
		Calculer 
		\begin{tabular}{c c c}
			\q1 \(\sum_{k=1}^nF_k\) & \q2 \(\sum_{k=1}^nF_k^2\) & \q3 \(\sum_{k=2}^n\frac1{F_{k-1}F_{k+1}}\)
		\end{tabular}
		\item 
		\begin{enumerate}[start=1, label={\bfseries \arabic*.}]
			\item Calculer les sommes suivantes
				\begin{enumerate}[(a)]
					\item \(\sum_{k=0}^n{\binom{k}{p}}\) avec $p,n\in\N$;
					\item \(\sum_{k=0}^p(-1)^k{\binom{n}{k}}\) avec $p\in\N$ et $n\in\N^*$.
				\end{enumerate}
			\item Soit $n\in\N^*$
				\begin{enumerate}[(a)]
					\item Simplifier la somme $S_{n,p}=\sum_{k=1}^n\prod_{l=0}^p(k+l)$ pour tout $p\in\N$.
					\item Retrouver la valeur de $\sum_{k=1}^nk^2$.
					\item Calculer la somme $\sum_{k=0}^n\frac{\binom{n}{k}}{\binom{2n-1}{k}}$.
				\end{enumerate}
		\end{enumerate}
	\end{enumerate}

	\subsct{Calcul de sommes}

	\vspace{2mm}
	\begin{enumerate}[start=4,label={\bfseries \arabic*}]
		\item Soit $n\in\N^*$. Montrer que pour toute famille $(x_i)_{i\in I_n}$ de nombres complexes :
		\begin{enumerate}[start=1, label={\bfseries \arabic*.}]
			\item \(\sum_{i=1}^n\left(x_i-\frac1n\sum_{j=1}^nx_j\right)=0\).
			\item \(\frac1n\sum_{i=1}\left(x_i-\frac1n\sum_{j=1}^nx_j\right)^2=\frac1n\sum_{i=1}^nx_i^2-\left(\frac1n\sum_{i=1}^nx_i\right)^2\).
		\end{enumerate}
		\item Calculer, pour tout $n\in\N$, les sommes suivantes.
		\begin{tabular}{c c}
			\q{1} \(A_n=\sum_{0\leq i\leq j\leq n}\frac i{j+1}\). & \q{2} \(B_n=\sum_{k=1}^n(-1)^kk^2\).\\
			\q{3} \(C_n=\sum_{k=0}^n\binom{2n+1}k\).
		\end{tabular}
		\item Soit $N$ un nombre entier de $n$ chiffres. Soit $s$ la somme de ses chiffres, et $t$ la somme de tous les nombres obtenus en combinant $2$ quelconques des chiffres de $N$ de rangs distincts (si $2$ chiffres sont égaux, un même nombre peut apparaître plusieurs fois dans la somme).\\
		Exprimer $t$ en fonction de $s$.
		\item Calculer, pour $n\in\N$, les sommes suivantes 
		\begin{tabular}{c c}
			\q{1} \(A_n=\sum_{1\leq i\leq j\leq n}(i+j)\). & \q{2} \(B_n=\sum_{1\leq i,j\leq n}\min(i,j)\).
		\end{tabular}
		\item Calculer de deux façons les sommes suivantes\\
		\begin{tabular}{c c}
			\q{1} \(A_n=\sum_{1\leq i,j\leq n}ij\). & \q{2} \(B_n=\sum_{1\leq i,j\leq n}i^2j\).
		\end{tabular}\\
		Retrouver la valeur de $\sum_{k=1}^nk^3$ et calculer $\sum_{k=1}^nk^4$.
		\item On pose, pour tout $n\in\N$, $S_n=\sum_{k=0}^n(-1)^{\lfloor \sqrt{k}\rfloor}$.
		\begin{enumerate}[start=1, label={\bfseries \arabic*.}]
			\item Calculer $S_{(2m)^2-1}$ pour tout $m\in\N$.
			\item Montrer que, pour tout $n\in\N$, on a \[|S_n|\leq \sqrt{n+1}\]
			Préciser le cas de l'égalité.
		\end{enumerate}
		\item Soit $E$ un ensemble fini de cardinal $n$.\\
		En considérant l'involution $X\mapsto \overline{X}$ de $\mathscr{P}(E)$, montrer les formules suivantes
		\begin{enumerate}[start=1, label={\bfseries \arabic*.}]
			\item \(\sum_{X\in\mathscr{P}(E)}|X|=n2^{n-1}\).
			\item
			\begin{enumerate}[(a)]
				\item \(\sum_{(X,Y)\in\P(E)^2}|X\cap Y|=n4^{n-1}\).
				\item \(\sum_{(X,Y)\in\P(E)^2}|X\cup Y|=3n4^{n-1}\).
			\end{enumerate}
		\end{enumerate}
	\end{enumerate}

	\subsct{Formule du binôme}
	\begin{enumerate}[start=11,label={\bfseries \arabic*}]
		\item Soit $n$ un entier naturel.
		\begin{enumerate}[start=1, label={\bfseries \arabic*.}]
			\item Montrer qu'il existe deux entiers $a_n$ et $b_n$ tels que \[(1+\sqrt{2})^n=a_n+b_n\sqrt{2}\quad \text{et}\quad (1-\sqrt{2})^n=a_n-b_n\sqrt{2}.\]
			\item Calculer $a_n^2-2b_n^2$ et déduire qu'il existe un entier $p-n$ tel que \[(1+\sqrt{2})^n=\sqrt{p_n}+\sqrt{p_n+1}.\]
		\end{enumerate}
		\item Calculer, pour tout $n\in\N^*$, les sommes suivantes\\
		\begin{tabular}{p{4cm} p{4cm}}
			\q{1} \(A_n=\sum_{k=0}^nk\binom nk\) & \q{2} \(B_n=\sum_{k=0}^n(-1)^k\binom nk\)\\
			\q{3} \(C_n=\sum_{k=0}^nk^2\binom nk\) & \q{4} \(D_n=\sum_{k=0}^n\frac{\binom nk}{k+1}\)
		\end{tabular}
		\item Calculer, pour tous $p,q\in\N$, la somme \[\sum_{k=0}^p\binom{p+q}k\binom{p+q-k}{p-k}\]
		\item 
		\begin{enumerate}[start=1, label={\bfseries \arabic*.}]
			\item Soit $(n,p,q)\in\N^3$.
			\begin{enumerate}[(a)]
				\item En utilisant l'identité polynomiale \[(1+x)^p(1+x)^q=(1+x)^{p+q},\]
				montrer que $\sum_{k=0}^n\binom pk\binom q{n-k}=\binom{p+q}n$.
				\item Calculer la somme suivante \[A_{n,p,q}=\sum_{k=0}^nk\binom pk\binom p{n-k}.\]
			\end{enumerate}
			\item Montrer que, pour tout $n\in\N$, \[\sum_{k=0}^{\lfloor n/2\rfloor}\left(\binom nk - \binom n{k-1}\right)^2=\frac1{n+1}\binom{2n}n\]
		\end{enumerate}
		\item Calculer les sommes suivantes
		\begin{enumerate}[start=1, label={\bfseries \arabic*.}]
			\item \(\sum_{k=0}^n(-1)^k\binom nk^2\), pour tout $n\in\N$.
			\item \(\sum_{k=1}^pk\binom n{p-k}\binom nk\), pour tous $n,p\in\N^*$.
		\end{enumerate}
		\item Soit $n\in\N^*$.
		\begin{enumerate}[start=1, label={\bfseries \arabic*.}]
			\item Montrer que \(\sum_{j=0}^nj\binom{2n}{n-j}=n\binom{2n-1}n\).
			\item En déduire la valeur des somme suivantes
			\begin{align*}
				A_n &=\sum_{\begin{smallmatrix}0\leq k\leq n\\ 0\leq \ell \leq n\end{smallmatrix}}\min(k,\ell)\binom nk\binom n\ell,\\
				B_n &=\sum_{\begin{smallmatrix}0\leq k\leq n\\ 0\leq \ell \leq n\end{smallmatrix}}\max(k,\ell)\binom nk\binom n\ell.
			\end{align*}
		\end{enumerate}
		\item 
		\begin{enumerate}[start=1, label={\bfseries \arabic*.}]
			\item Montrer que, pour tout $p,q,n\in\N$, on a \[\sum_{k=0}^n\binom{n-k}p\binom kq=\binom{n+1}{p+q+1}.\]
			\item En déduire que, pour tout $n\in\N$, on a \[\sum_{k=0}^n2^k\binom{2n-k}n=4^n\]
		\end{enumerate}
	\end{enumerate}
\end{document}