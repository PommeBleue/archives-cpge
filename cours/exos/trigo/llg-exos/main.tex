%! suppress = Unicode
\documentclass[14pt]{article}

% Math related packages.
\usepackage{amsfonts, amsthm, amsmath, amssymb, amstext}
\usepackage{mathtools}
\usepackage{physics}
\usepackage{cancel, textcomp}
\usepackage[mathscr]{euscript}
\usepackage[nointegrals]{wasysym}
\usepackage[a4paper, left=1.5cm, right=1.7cm, top=2.3cm, bottom=2cm]{geometry}

\usepackage{esvect}
\usepackage{IEEEtrantools}

% Font related packages.
\usepackage[french]{babel}
\usepackage[unicode]{hyperref}
\usepackage[fontsize = 10pt]{scrextend}
\usepackage[T1]{fontenc}
\usepackage[utf8]{inputenc}
\usepackage[finemath]{kotex}
\usepackage{dhucs-nanumfont}
\usepackage{mathpazo}
\usepackage{FiraMono}
\usepackage{mathrsfs}

% A more colorful world.
\usepackage{xcolor}

% Display Source Code
\usepackage{listings}
\lstset{
    language=Python,
    basicstyle=\small\ttfamily\mdseries,
    numberstyle=\color{gray},
    stringstyle=\color[HTML]{933797},
    commentstyle=\color[HTML]{228B22},
    emph={[2]from,with,import,as,pass,return,and,or,not},
    emphstyle={[2]\color[HTML]{DD52F0}},
    emph={[3]range,format,enumerate,print},
    emphstyle={[3]\color[HTML]{D17032}},
    emph={[4]if,elif,else,for,while,in,def,lambda,int,float,all,len},
    emphstyle={[4]\color{blue}},
    emph={[5]abs},
    emphstyle={[5]\color{black}},
    showstringspaces=false,
    breaklines=true,
    prebreak=\mbox{{\color{gray}\tiny$\searrow$}},
    numbers=left,
    xleftmargin=15pt
}

% OPTIONS
\everymath{\displaystyle}

% COMMANDS
\newcommand{\pr}[1]{\left(#1\right)}
\newcommand{\f}[1]{\texttt{#1}}
\newcommand{\chpt}[1]{\newpage\begin{flushright}\huge\textbf{#1}\end{flushright}}
\newcommand{\urlsymbol}{\kern1pt\vbox to .5ex{}\raise.10ex\hbox{\pdfliteral{%
    q .8 0 0 .8 0 0 cm
    2.5 5 m 1 j 1 J .8 w
    1 5 l 0 5 0 4 y 0 1 l 0 0 1 0 y 4 0 l 5 0 5 1 y 5 2.5 l S
    3 3 m 6 6 l S 4 6 m 6 6 l 6 4 l S
Q}}\kern5pt}

\def\N{\mathbb N}
\def\Z{\mathbb Z}
\def\Q{\mathbb Q}
\def\R{\mathbb R}
\def\C{\mathbb C}
\def\K{\mathbb K}

\def\ssi{\Leftrightarrow}
\def\Ssi{\Longleftrightarrow}

\def\lb{\lbrace}
\def\rb{\rbrace}

\title{\huge\textbf{Trigonométrie}}
\author{Amar AHMANE\\ MP2I}
\date{}

\begin{document}
    \maketitle
    \subsubsection*{Exercice 1}
    \paragraph{Énoncé} Calculer $\sin\pr{\frac{45\pi}{12}}$ et $\cos\pr{\frac{45\pi}{12}}$
    \paragraph{Solution proposée} On a
    \begin{align*}
        \sin\pr{\frac{45\pi}{12}} &= \sin\pr{\frac{3^2\times 5\times\pi}{2^2\times 3}}\\
                                  &= \sin\pr{\frac{15\pi}{4}}\\
                                  &= \sin\pr{4\pi-\frac{\pi}{4}}\\
                                  &= -\sin\pr{\frac\pi4}\\
                                  &= -\frac{\sqrt{2}}2
    \end{align*}
    et
    \begin{align*}
        \cos\pr{\frac{45\pi}{12}} &= \cos\pr{4\pi-\frac{\pi}{4}}\\
                                  &= \cos\pr{\frac\pi4}\\
                                  &= \frac{\sqrt{2}}2
    \end{align*}

    \subsubsection*{Exercice 2}
    \paragraph{Énoncé} Résoudre dans $\R$ l'équation $\sin\pr{x+\frac{3\pi}4}=\cos\pr{2x-\frac\pi3}$.
    \paragraph{Solution proposée} Soit $x\in\R$, \[\sin\pr{x+\frac{3\pi}4}=\cos\pr{2x-\frac\pi3}\ssi \sin\pr{x+\frac{3\pi}4}=\sin\pr{2x-\frac\pi3+\frac\pi2}\]
    Donc
    \begin{align*}
        &\begin{cases}x+\frac{3\pi}4\equiv 2x+\frac\pi6[2\pi]\\ \text{ou}\\x+\frac{3\pi}4\equiv\pi- 2x-\frac\pi6[2\pi] \end{cases}\\
        \Ssi&\begin{cases}-x\equiv\frac{2\pi}{12}-\frac{9\pi}{12}[2\pi]\\ \text{ou}\\3x\equiv\frac{12\pi}12-\frac{9\pi}{12}-\frac{2\pi}{12}[2\pi] \end{cases}\\
    \end{align*}
    \newpage
    \begin{align*}
        \Ssi&\begin{cases}x\equiv\frac{7\pi}{12}[2\pi]\\ \text{ou}\\x\equiv\frac\pi{36}\left[\frac{2\pi}3\right] \end{cases}\\
    \end{align*}

    \subsubsection*{Exercice 3}
    \paragraph{Énoncé} Résoudre dans $\R$ l'équation $\sin\pr{5x+\frac\pi2}=\sin(2x)$.
    \paragraph{Solution proposée} Soit $x\in\R$,
    Donc
    \begin{align*}
        &\sin\pr{5x+\frac\pi2}=\sin(2x)\\
        \Ssi&\begin{cases}5x+\frac\pi2\equiv 2x[2\pi]\\ \text{ou}\\5x+\frac\pi2\equiv\pi- 2x[2\pi] \end{cases}\\
        \Ssi&\begin{cases}3x\equiv-\frac\pi2[2\pi]\\ \text{ou}\\7x\equiv\pi-\frac\pi2[2\pi] \end{cases}\\
        \Ssi&\begin{cases}x\equiv-\frac\pi6\left[\frac{2\pi}3\right]\\\text{ou}\\x\equiv\frac\pi{14}\left[\frac{2\pi}7\right] \end{cases}
    \end{align*}

    \subsubsection*{Exercice 10}
    \paragraph{Énoncé} 
    \begin{enumerate}
        \item Exprimer $\tan(a+b+c)$ en fonction de $\tan(a),\ \tan(b)$ et $\tan(c)$.
        \item Généraliser à $\tan\left(\sum_{k=1}^na_k\right)$.
    \end{enumerate}
    

    \paragraph{Solution proposée} 
    \begin{enumerate}
        \item Soit $(a,b,c)\in\R^3$ tel que $\tan(a),\ \tan(b)$ et $\tan(c)$ soient définis.\\

        \begin{align*}
            \tan(a+b+c) &=\frac{\tan(a+b)+\tan(c)}{1-\tan(a+b)\tan(c)}\\
                        &=\frac{\frac{\tan(a)+\tan(b)}{1-\tan(a)\tan(b)}+\tan(c)}{1-\frac{\tan(a)+\tan(b)}{1-\tan(a)\tan(b)}\tan(c)}\\
                        &=\frac{\tan(a)+\tan(b)+\tan(c)(1-\tan(a)\tan(b))}{1-\tan(a)\tan(b)-\tan(c)(\tan(a)+\tan(b))}\\
                        &=\frac{\tan(a)+\tan(b)+\tan(c)-\tan(a)\tan(b)\tan(c)}{1-\tan(a)\tan(b)-\tan(a)\tan(c)-\tan(c)\tan(b)}
        \end{align*}

        \item Soit $(a_1,a_2,a_3,a_4)\in\R^4$ tel que $\forall i\in\lb1,2,3,4\rb,\ \tan(a_i)$ est défini.\\

        \begin{align*}
            \tan\left(\sum_{k=1}^4a_k\right) &=\frac{\tan\left(\sum_{k=1}^3a_k\right)+\tan(a_4)}{1-\tan\left(\sum_{k=1}^3a_k\right)\tan(a_4)}\\
                          &=\frac{
                                \frac{
                                    \sum_{k=1}^3\tan(a_k)-\prod_{k=1}^3\tan(a_k)
                                    }
                                    {
                                    1-\sum_{k=1}^3\prod_{i=1;\ i\neq k}^3a_k
                                    }
                                    +\tan(a_4)
                                }
                                {
                                1-\left(
                                    \frac
                                    {
                                    \sum_{k=1}^3\tan(a_k)-\prod_{k=1}^3\tan(a_k)
                                    }
                                    {
                                    1-\sum_{k=1}^3\prod_{i=1;\ i\neq k}^3a_k
                                    }\right)\tan(a_4)
                                }\\
                          &=\frac{
                            \sum_{k=1}^3\tan(a_k)-\prod_{k=1}^3\tan(a_k)+\tan(a_4)\left(1-\sum_{k=1}^3\prod_{i=1;\ i\neq k}^3\tan(a_k)\right)
                          }
                          {
                            1-\sum_{k=1}^3\prod_{i=1;\ i\neq k}^3\tan(a_k)-\left(\sum_{k=1}^3\tan(a_k)-\prod_{k=1}^3\tan(a_k)\right)\tan(a_4)
                          }\\
                          &=\frac{
                            \sum_{k=1}^4\tan(a_k)-\prod_{k=1}^3\tan(a_k)-\sum_{k=1}^3\prod_{i=1;\ i\neq k}^4\tan(a_k)
                          }
                          {
                            1-\sum_{k=1}^3\prod_{i=1;\ i\neq k}^3\tan(a_k)-\tan(a_4)\sum_{k=1}^3\tan(a_k)+\prod_{k=1}^4\tan(a_k)
                          }
        \end{align*}
    \end{enumerate}

\end{document}