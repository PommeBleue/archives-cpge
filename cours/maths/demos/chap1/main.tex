%! suppress = Unicode
\documentclass[10pt]{article}

% Math related packages.
\usepackage{amsfonts, amsthm, amsmath, amssymb, amstext}
\usepackage{mathtools}
\usepackage{physics}
\usepackage{cancel, textcomp}
\usepackage[mathscr]{euscript}
\usepackage[nointegrals]{wasysym}
\usepackage[a4paper, left=1.5cm, right=1.7cm, top=2.3cm, bottom=2cm]{geometry}

\usepackage{esvect}
\usepackage{IEEEtrantools}

% Font related packages.
\usepackage[french]{babel}
\usepackage[unicode]{hyperref}
\usepackage[fontsize = 10pt]{scrextend}
\usepackage[T1]{fontenc}
\usepackage[utf8]{inputenc}
\usepackage[finemath]{kotex}
\usepackage{dhucs-nanumfont}
\usepackage{mathpazo}
\usepackage{FiraMono}
\usepackage{mathrsfs}

% A more colorful world.
\usepackage{xcolor}

% Display Source Code
\usepackage{listings}
\lstset{
    language=Python,
    basicstyle=\small\ttfamily\mdseries,
    numberstyle=\color{gray},
    stringstyle=\color[HTML]{933797},
    commentstyle=\color[HTML]{228B22},
    emph={[2]from,with,import,as,pass,return,and,or,not},
    emphstyle={[2]\color[HTML]{DD52F0}},
    emph={[3]range,format,enumerate,print},
    emphstyle={[3]\color[HTML]{D17032}},
    emph={[4]if,elif,else,for,while,in,def,lambda,int,float,all,len},
    emphstyle={[4]\color{blue}},
    emph={[5]abs},
    emphstyle={[5]\color{black}},
    showstringspaces=false,
    breaklines=true,
    prebreak=\mbox{{\color{gray}\tiny$\searrow$}},
    numbers=left,
    xleftmargin=15pt
}

% OPTIONS
\everymath{\displaystyle}

% COMMANDS
\newcommand{\f}[1]{\texttt{#1}}
\newcommand{\chpt}[1]{\newpage\begin{flushright}\huge\textbf{#1}\end{flushright}}
\newcommand{\urlsymbol}{\kern1pt\vbox to .5ex{}\raise.10ex\hbox{\pdfliteral{%
    q .8 0 0 .8 0 0 cm
    2.5 5 m 1 j 1 J .8 w
    1 5 l 0 5 0 4 y 0 1 l 0 0 1 0 y 4 0 l 5 0 5 1 y 5 2.5 l S
    3 3 m 6 6 l S 4 6 m 6 6 l 6 4 l S
Q}}\kern5pt}

\def\N{\mathbb N}
\def\Z{\mathbb Z}
\def\Q{\mathbb Q}
\def\R{\mathbb R}
\def\C{\mathbb C}
\def\K{\mathbb K}

\title{\huge\textbf{Exponentielle et fonctions associées.\\ Quelques preuves du cours.}}
\date{}

\begin{document}
    \maketitle
    \subsubsection*{Théorème 4}
    \begin{equation}
        \boxed{\forall x\in \mathbb{R},\ \forall y\in\mathbb{R},\quad  \exp(x+y)=\exp(x)\exp(y)}
    \end{equation}
    Il suffit de considérer la fonction $f:x\mapsto \frac{\exp(x+a)}{\exp(a)}$.\\
    Reste à prouver ce qui suit :
    \begin{itemize}
        \item \(\forall x\in\mathbb{R},\quad \exp(-x)=\frac1{\exp(x)}\).
        \item \(\forall x\in \mathbb{R},\ \forall y\in\mathbb{R},\quad \exp(x-y)=\frac{\exp(x)}{\exp(y)}\).
        \item \(\forall x\in\mathbb{R},\ \forall p\in\mathbb{Z},\quad \exp(px)=\exp(x)^p\).
    \end{itemize}
    Les preuves dans l'ordre :
    \begin{itemize}
        \item Soit $x\in\mathbb{R}$,
        \begin{align*}
            1=\exp(0)&=\exp(x-x)\\
                     &=\exp(x)\exp(-x)
        \end{align*}
        D'où il vient que $\exp(x)=\frac1{\exp(x)}$.\\
        Ceci étant vrai pour tout $x\in\mathbb{R}$, on a prouvé la propriété voulue.
        \item Soient $x,y\in\mathbb{R}$,
        \begin{align*}
            \exp(x-y)&=\exp(x+(-y))\\
                     &=\exp(x)\exp(-y)\\
                     &=\frac{\exp(x)}{\exp(y)}
        \end{align*}
        Ceci étant vrai pour tout $x,y\in\mathbb{R}$, on a prouvé la propriété voulue.
        \item Soit $x\in\mathbb{R},\ n\in\mathbb{N}$,\
        \begin{align*}
            \exp(nx)&=\exp\left(\sum_{k=1}^nx\right)\\
                    &=^*\prod_{k=1}^n\exp(x)\\
                    &=\exp(x)^n
        \end{align*}
        On justifie * par la propriété de morphisme (1) prouvée plus haut.
        Ainsi, ceci étant vrai pour tout $x\in\mathbb{R}$ et pour tout $n\in\mathbb{N}$, on a montré ce qui suit : 
        \begin{equation*}
            \forall x\in\mathbb{R},\ \forall n\in\mathbb{N},\quad \exp(nx)=\exp(x)^n
        \end{equation*}
        Soit à présent $x\in\mathbb{R}$ et $n\in\mathbb{Z}\backslash\mathbb{N}$.
        \begin{align*}
            \exp(nx)&=\exp(-(-n)x)\\
                    &=\frac1{\exp((-n)x)}\\
                    &=\frac1{\exp(x)^{-n}}\\
                    &=\exp(x)^n
        \end{align*}
        Ceci montre la même propriété pour les entiers négatifs.
    \end{itemize}

    \subsubsection*{Proposition 7}
    On sait que 
    \begin{equation}
        \forall x\in\R,\quad \exp(x)\geq x + 1
    \end{equation}
    Montrons que 
    \begin{equation}
        \forall x\in\R^*_+,\quad \ln(x)\leq x - 1
    \end{equation}
    Soit $x\in\R^*_+$, en ce cas $\ln(x)\in\R$ et 
    \begin{alignat*}{3}
        &\exp(\ln(x)) &&\geq \ln(x) + 1\\
        \Leftrightarrow& \quad x-1 &&\geq \ln(x)
    \end{alignat*}
    Ceci étant vrai pour tout $x\in\R^*_+$, on a montré (3).

    \subsubsection*{Proposition 8}
    On montre que 
    \begin{equation}
        \forall x,y\in\R^*_+,\quad \ln(xy)=\ln(x)+\ln(y)
    \end{equation}
    On tire profit de la propriété de morphisme de $\exp$.
\end{document}