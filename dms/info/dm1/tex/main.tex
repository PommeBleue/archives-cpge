% Preamble
\documentclass[10pt]{article}

% Math related packages.
\usepackage{amsfonts, amsthm, amsmath, amssymb, amstext}
\usepackage{mathtools}
\usepackage{physics}
\usepackage{cancel, textcomp}
\usepackage[mathscr]{euscript}
\usepackage[nointegrals]{wasysym}
\usepackage[left=3cm, right=3cm, top=1.5cm, bottom=1.5cm]{geometry}

\usepackage{esvect}
\usepackage{IEEEtrantools}

% Font related packages.
\usepackage[french]{babel}
\usepackage[unicode]{hyperref}
\usepackage[fontsize = 10pt]{scrextend}
\usepackage[T1]{fontenc}
\usepackage[utf8]{inputenc}
\usepackage[finemath]{kotex}
\usepackage{dhucs-nanumfont}
\usepackage{mathpazo}
\usepackage{FiraMono}
\usepackage{mathrsfs}

% Other
\usepackage{enumerate}
\usepackage[shortlabels]{enumitem}
\usepackage{tabularx}
\usepackage[object=vectorian]{pgfornament}
\usepackage{pgf}
\usepackage{pgfpages}
\usepackage[european,straightvoltages]{circuitikz}


% A more colorful world.
\usepackage{xcolor}

% Display Source Code
\usepackage{listings}
\lstset{
    language=C,
    basicstyle=\small\ttfamily\mdseries,
    numberstyle=\color{gray},
    stringstyle=\color[HTML]{933797},
    commentstyle=\color[HTML]{228B22},
    emph={[2]from,with,import,as,pass,return,and,or,not},
    emphstyle={[2]\color[HTML]{DD52F0}},
    emph={[3]range,format,enumerate,print},
    emphstyle={[3]\color[HTML]{D17032}},
    emph={[4]if,elif,else,for,while,in,def,lambda,int,float,all,len},
    emphstyle={[4]\color{blue}},
    emph={[5]abs},
    emphstyle={[5]\color{black}},
    showstringspaces=false,
    breaklines=true,
    prebreak=\mbox{{\color{gray}\tiny$\searrow$}},
    numbers=left,
    xleftmargin=15pt
}

% OPTIONS
\everymath\expandafter{\the\everymath\displaystyle}

% COMMANDS
\newcommand{\f}[1]{\texttt{#1}}
\newcommand{\sct}[1]{
	\begin{center}
		\Large\textbf{#1}
	\end{center}
}
\newcommand{\subsct}[1]{
	\begin{center}
		\large\textbf{#1}
	\end{center}
}
\newcommand{\inl}[2]{[\![#1, #2]\!]}
\newcommand{\q}[1]{\textbf{#1.}\quad}
\newcommand{\urlsymbol}{\kern1pt\vbox to .5ex{}\raise.10ex\hbox{\pdfliteral{%
    q .8 0 0 .8 0 0 cm
    2.5 5 m 1 j 1 J .8 w
    1 5 l 0 5 0 4 y 0 1 l 0 0 1 0 y 4 0 l 5 0 5 1 y 5 2.5 l S
    3 3 m 6 6 l S 4 6 m 6 6 l 6 4 l S
Q}}\kern5pt}


\def\N{\mathbb N}
\def\Z{\mathbb Z}
\def\Q{\mathbb Q}
\def\R{\mathbb R}
\def\Rpe{\mathbb R_+^*}
\def\C{\mathbb C}
\def\K{\mathbb K}
\def\L{\mathbb L}

\def\P{\mathscr{P}}

\def\ssi{\Leftrightarrow}
\def\Ssi{\Longleftrightarrow}
\def\implique{\Longrightarrow}


% CUSTOM TITLE
\makeatletter
\def\@maketitle{%
	\newpage
	%  \null% DELETED
	%  \vskip 2em% DELETED
	\begin{center}%
		\let \footnote \thanks
		{\LARGE \@title \par}%
		\vskip 1em%
		{\large
			\lineskip .5em%
			\begin{tabular}[t]{c}%
				\@author
			\end{tabular}\par}%
		\vskip 1em%
		{\large \@date}%
	\end{center}%
	\par
	\vskip -1em}
\makeatother


\title{DM 1 – Nuage de points}
\author{Amar AHMANE\\ MP2I.}
\date{}

\begin{document}
	\maketitle
	\subsection*{1 – Approche exhaustive naïve}
	\begin{enumerate}[start=1,label={\bfseries Question \arabic*}]
		\item Définition de la fonction \f{distance}.
		\begin{lstlisting}
		/*
		Input: 
			p1: parameter of type struct point containing the coordinates of a point in an eucledian sapce
			p2: parameter of type struct point containing the coordinates of a point in an eucledian sapce
		Output: eucledian distance between these two points.
		*/
		double distance(struct point p1, struct point p2){
			return sqrt((p2.x-p1.x)*(p2.x-p1.x)+(p2.y-p1.y)*(p2.y-p1.y));
		}
		\end{lstlisting}
		\item Définition de la fonction \f{plus\_proches}.
		\begin{lstlisting}
		/*
			Input:
			Output: Void. Changes the values of p and q setting them to the indexes of the two closest points in the cloud.
		*/
		void plus_proches(struct nuage* n, int* p, int* q){
			if(n == NULL || p == NULL || q == NULL){
				return;
			}
			assert(n->taille > 1);
			double ref = distance(n->P[0], n->P[1]);
			*p = 0, *q = 0;
			for(int i = 0; i < n->taille; i++){
				for(int j = 0; j < n->taille; j++){
					if(distance(n->P[i], n->P[j]) < ref){
						*p = i, *q = j;
					}
				}
			}
		}
		\end{lstlisting}
		\item En supposant que les opérations s'effectuant dans la deuxième boucle sont faites en un temps $\Theta(1)$, alors la complexité est en $\sum_{i=1}^n\sum_{j=1}^n\Theta(1)=\Theta(n^2)$.
	\end{enumerate}
	\subsection*{2 – Méthode sophistiquée}
	\begin{enumerate}[start=4,label={\bfseries Question \arabic*}]
		\item Cette fonction trie le tableau \f{tab}. En effet, si l'on choisit la phrase suivante comme invariant, pour une itération $i$ donnée : \[\text{« le tableau \f{tab[:i]} est trié »}\]
		On utilise ici la notation python pour le saucissonnage de \f{tab}.
		\begin{enumerate}[i)]
			\item Il est clair que lorsque $i=0$, \f{tab[:0]} est trié puisqu'il est vide.
			\item On suppose que pour un $i$ donné (une itération donnée), \f{tab[:i]} est trié. Ainsi, à l'itération suivante, le $i$ème élément sera placé dans \f{tab[:i+1]} de sorte que tous les élements à droite soient plus grands et que tous les éléments à gauche soient plus petits : en effet, ceci est contrôlé par la condition \f{tab[pos]<tab[pos - 1]}. Le tableau \f{tab[:i]} étant déjà trié par hypothèse de récurrence, le tableau \f{tab[:i + 1]} l'est aussi.
			\item par le principe de récurrence, la phrase déclarée plus haut est bien un invariant de boucle. Ceci prouve la correction de l'algorithme.
		\end{enumerate} 
		\item En supposant que les opérations faites à l'intérieur de la boucle \f{while} soient effectuées en temps constant, on a alors une complexité en $\sum_{i=0}^{n - 1}\sum_{j=1}^i\Theta(1)=\sum_{i=1}^{n-1}\Theta(j)=\Theta\left(\frac{n(n-1)}2\right)=\Theta(n^2)$.
		\item On programme la fonction \f{tri\_cluster} qui comporte les modifications voulues.
		\begin{lstlisting}
			void tri_cluster(struct nuage* n){
				int pos;
				for(int i = 0; i < n->taille; i++){
					pos = i;
					while(pos > 0 && n->P[pos].x < n->P[pos - 1].x){
						struct point tmp = n->P[pos];
						n->P[pos] = n->P[pos - 1];
						n->P[pos - 1] = tmp;
						pos = pos - 1;
					}
				}
			}
		\end{lstlisting}
		\item L'algorithme de tri par tas est un algorithme de tri de complexité $\mathcal O(n\log n)$ au pire des cas.
		\item On programme la fonction \f{sous\_cluster}.
		\begin{lstlisting}
			/*
				Input:	
					c: parametre de type pointeur sur struct cluster; cluster dont on veut former un sous-cluster
					min: parametre de type double; abscisse minimale
					max: parametre de type double; abscisse maximale
				Output: Sous-cluster du cluster c dont les abscisses sont comprises entre min et max.
			**/
			struct cluster* sous_cluster(struct cluster* c, double min, double max){
				assert(!(c == NULL))
				struct cluster* result;
				struct nuage* N;
				int a, b;
				for(int i = 1; i < c->taille; i++){
					if(c->N->P[c->abs[i]] >= min){
						a = i;
					}
				}
				for(int i = c->taille - 1; i >= 0; i--){
					if(c->N->P[c->abs[i]] <= max){
						b = i;
					}
				}
				result = (struct cluster*) malloc((b - a + 1)*sizeof(struct cluster));
				N = (struct nuage*) malloc((b - a + 1)*sizeof(struct nuage));
				result->abs = (int*) malloc((b - a +1)*sizeof(int));
				result->ord = (int*) malloc((b - a +1)*sizeof(int));
				N->taille = b - a + 1;
				result->taille = N->taille;
				for(int i = 0; i + a <= b; i++){
					result->abs[i] = c->abs[i + a];
					result->ord[i] = c->abs[i + a];
					N->P[i] = c->N->P[result->abs[i]];
				}
				result->N = N;
				return result;
			}
		\end{lstlisting}
		\item On programme la fonction \f{mediane}.
		\begin{lstlisting}
			/*
				Input:	
					c: parametre de type pointeur sur struct cluster; cluster dont on veut recuperer la mediane
				Output: mediane du cluster fourni en parametre
			**/
			double mediane(struct cluster* c){
				assert(!(c == NULL));
				return c->N->P[c->abs[c->taille/2]].x;
			}
		\end{lstlisting}
		\item On programme la fonction \f{gauche}.
		\begin{lstlisting}
			/*
				Input:	
					c: parametre de type pointeur sur struct cluster; cluster dont on veut recuperer la partie gauche
				Output: Partie gauche du cluster fourni en parametre
			**/
			struct cluster* gauche(struct cluster *c){
				return sous_cluster(c, c->N->P[c->abs[0]].x, mediane(c));
			}
		\end{lstlisting}
		\item Si l'on choisit un premier point en dehors de l'intervalle $[x_0-\delta,x_0+\delta]$, la distance entre celui-ci et un point quelconque de l'autre moitié du cluster sera forcément strictement supérieure à $\delta$; on se rend alors compte que si on choisit deux points de deux clusters à une distance $\delta$ l'un de l'autre, on les trouvera forcément dans $[x_0-\delta,x_0+\delta]$.
		\item On programme la fonction \f{bande\_centrale}.
		\begin{lstlisting}
			/*
				Input:	
					c: parametre de type pointeur sur struct cluster; cluster dont on veut recuperer un sous-clusters dont les points sont a une distance au plus d0 de la mediane.
					d0: parametre de type double; distance maximale de la mediane.
				Output: Bande centrale de largeur 2*d0.
			**/
			struct cluster* bande_centrale(struct cluster* c, double d0){
				return sous_cluster(c, mediane(c) - d0, mediane(c) + d0);
			}
		\end{lstlisting}
		\item Voir le fichier \f{part2.c}.
		\item Voir le fichier \f{part2.c}.
		\item Lors du premier appel de la fonction, on effectue plusieurs opérations : 
		\begin{enumerate}[i)]
			\item l'appel récursif de cette même fonction, deux fois pour un cluster de taille deux fois moindre (donc $n/2$).
			\item un appel à la fonction fusion avec une entrée de taille $n$.
		\end{enumerate}
		En notant $C(n)$ le nombre d'opérations effectuées pour une entrée de taille $n$, on a bien $C(n)=2C(n/2)+\mathcal{O}(n)$ (\f{fusion} ayant une complexité linéaire).
		\item On suppose qu'il existe $k\in\N$ tel que $n=2^k$, on pourra ainsi note $C_k=C(2^k)$. D'où \[C_k=2C_{k-1}+\mathcal{O}(n)\]
		Par récurrence, on obtient que 
		\begin{align*}
			C_k &= 2^kC_0+\sum_{i=0}^{k-1}2^i\mathcal{O}(2^{k-i})\\
				&= 2^kC_0+\sum_{i=0}^{k-1}\mathcal{O}(2^k)\\
				&= 2^kC_0+\mathcal{O}(k2^k)\\
		\end{align*}
		Finalement \[C(n)=nC_0+\mathcal{O}(n\log_2(n)\]
		Or, $C_0$ est une constante, donc $C(n)=\mathcal{O}(n\log n)$.
	\end{enumerate}

\end{document}